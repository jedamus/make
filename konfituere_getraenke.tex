  \mynewchapter{Getränke}

  \mynewsection{Holunderblütensirup}\label{holunderbluetensirup}

    % aus Wohnen BHW 2/2010

    \begin{zutaten}
      12 & große Dolden\index{Holunderdolden} oder \\
      2 Handvoll & \myindex{Holunderblüten} \\
      1 l & abgekochtes kaltes \myindex{Wasser} \\
      1 kg & \myindex{Zucker} \\
      30 g & \myindex{Zitronensäure} (aus der Apotheke) \\
      2 & unbehandelte \myindex{Zitrone}n \\
    \end{zutaten}

    \bemerkung{reicht für 1,8 Liter}

    \begin{zubereitung}
      Holunderblüten nach Tierchen absuchen, diese gegebenenfalls entfernen.
      Nicht zu sehr schütteln, damit der feine Staub an den Blüten bleibt. \\
      Wasser, Zucker und Zitronensäure in ein großes Gefäß füllen, Blüten
      hineingeben und in die Flüssigkeit drücken. Zitronen in feine Scheiben
      schneiden, Oberfläche damit belegen. Mit einem Teller beschweren, damit
      alles von Flüssigkeit bedeckt ist. \\
      Mit Folie abdecken, an einem kühlen Ort zwei Wochen durchziehen lassen.
      Zwischendurch immer wieder umrühren --- der Zucker soll am Ende ganz
      aufgelöst sein. \\
      Den Sirup durch ein feines Sieb abgießen und in heiß ausgespülte Flaschen
      füllen. Er ist kühl aufbewahrt mehrere Monate haltbar. \\
      Tip: Mit Mineralwasser verdünnt, wird aus dem Sirup eine erfrischende
      Limonade. Ein Glas Sekt duftet mit einem kleinen Schuß Sirup wunderbar
      nach frischen Blüten. Obstsalat oder frischen Beeren verleiht der Saft
      ein ganz besonderes Aroma. \\
    \end{zubereitung}

  \mynewsection{Holundersaft}

    \begin{zutaten}
      1 kg & \myindex{Holunderbeeren}\index{Beeren>Holunder-} \\
      1 l & \myindex{Wasser} \\
      300 g & \myindex{Zucker} \\
    \end{zutaten}

    \begin{zubereitung}
      Beeren mit dem Wasser aufkochen, bis die Beeren platzen. Den Saft durch
      ein Tuch abtropfen lassen, Zucker zugeben und erneut aufkochen. Schaum
      abschöpfen. Den noch heißen Holundersaft in keimfreie Flaschen abfüllen.
      \\
    \end{zubereitung}

  \mynewsection{Schlehenlikör}

    \begin{zutaten}
      1 kg & reife \myindex{Schlehenbeeren}\index{Beeren>Schlehen-}
             \footnote{Beeren sollten den ersten Frost abbekommen haben oder
	     die Beeren einfrieren}\\
      1 l & \myindex{Weingeist} \\
      250 g & \myindex{Zucker} \\
      1 & \myindex{Vanillestange} \\
    \end{zutaten}

    \begin{zubereitung}
      Schlehen waschen, gut verlesen und jede Beere mit einem Dorn zweimal
      anstechen. Beeren zusammen mit dem Weingeist in eine Flasche mit weitem
      Hals füllen und 6--8~Wochen an einem schattigen Platz stehen lassen. \\
      Danach die Flüssigkeit filtern. Den Zucker und die Vanille mit einem
      Liter Wasser ca. eine Viertelstunde kochen und den Sirup mit dem Alkohol
      vermischen. Das Ganze in gut verschließbare Flaschen abfüllen und vor der
      ersten Kostprobe noch einige Monate lagern. \\
    \end{zubereitung}

  \mynewsection{Ingwerlimonade}

    \begin{zutaten}
      & \myindex{Eiswürfel} \\
      4--5 & \myindex{Minze}blätter \\
      1 Teelöffel & \myindex{Ingwer} \\
      1 Dose & \myindex{Zitronenlimonade}\index{Limonade>Zitronen-} \\
      & \myindex{Mineralwasser}\index{Wasser>Mineral-} \\
    \end{zutaten}

    \begin{zubereitung}
      Glaskrug mit Eiswürfeln füllen, Minzeblätter quetschen und zerreissen.
      2~Zentimeter Ingwer raspeln, Zitronenlimonade dazu und mit Mineralwasser
      auffüllen, umrühren. \\
    \end{zubereitung}

  \mynewsection{Geeiste Himbeer-Limonade}

    \begin{zutaten}
      \breh{} Beutel & \myindex{Eiswürfel} \\
      1 Schälchen & frische \myindex{Himbeeren}\index{Beeren>Him-} (oder TK) \\
      100 ml & \myindex{Holunderblütensirup} (siehe Seite
               \pageref{holunderbluetensirup}) \\
      5--6 & \myindex{Minze}blätter \\
      & \myindex{Mineralwasser}\index{Wasser>Mineral-} \\
    \end{zutaten}

    \begin{zubereitung}
      Außer Mineralwasser alles in den Mixer, bei laufendem Motor Mineralwasser
      dazu. Inhalt in einen Glaskrug, mit Mineralwasser auffüllen. \\
    \end{zubereitung}

  \mynewsection{Hibiskus-Eistee}

    \begin{zutaten}
      2--3 Teebeutel & \myindex{Hibiskus} \\
      750 ml & kochendes \myindex{Wasser} \\
      1 Eßlöffel & \myindex{Zucker} \\
      & Schale einer \myindex{Clementine} oder \myindex{Orange} oder
        \myindex{Limette} \\
      & \myindex{Eiswürfel} \\
      2 Stengel & \myindex{Minze} \\
    \end{zutaten}

    \begin{zubereitung}
      Tee brühen, Zucker rein, Mit Sparschäler Schale der Clementine oder
      Orange/Limette abnehmen und in das Getränk. In abgekühlten Tee
      Eiswürfel und Minze geben. \\
    \end{zubereitung}

  \mynewsection{St. Clement's Drink}

    \begin{zutaten}
      & \myindex{Eiswürfel} \\
      3 Zweige & \myindex{Minze} \\
      1 & \myindex{Zitrone} \\
      1 & \myindex{Orange} \\
      & eventuell Zucker \\
      & \myindex{Mineralwasser}\index{Wasser>Mineral-} \\
    \end{zutaten}

    \begin{zubereitung}
      Glaskrug mit Eiswürfeln füllen, Zitrone und Orange pressen, Zucker in den
      Krug, Mineralwasser dazu, umrühren. \\
    \end{zubereitung}

  % \mynewsection{Text}

    % \begin{zutaten}
    % \end{zutaten}

    % \begin{zubereitung}
    % \end{zubereitung}
