  \mynewchapter{Marmeladen und Gelees}

  % aus Marmeladen und Gelees, GU

    \mynewsection{Ananas-Konfitüre}\index{Rezept>Konfitüre>Ananas-}

      \begin{zutaten}
        1 & \myindex{Ananas} (1\breh{} kg, geputzt 800 g) \\
	200 g & \myindex{Rohrzucker} (ersatzweise Haushaltszucker) \\
	2 & Bio-\myindex{Limette}n \\
	5 & \myindex{Ingwerpflaumen} in Sirup (aus dem Asia-Laden) \\
	2 Teelöffel & \myindex{Agar-Agar} \\
	\breh{} Teelöffel & \myindex{Ingwer}pulver \\
      \end{zutaten}

      \reichtfuer{2}{300 ml}
      \haelt{2 Monate}
      %         kcal EW   F    KH
      \glaskcal{690}{2 g}{1 g}{162 g}

      \begin{zubereitung}
        Von der Ananas das Grün abschneiden. Die Ananas längs vierteln,
	schälen und den harten Strunk herausschneiden. Die Augen großzügig
	entfernen. Das Fruchtfleisch in kleine Würfel schneiden. Die Hälfte
	der Ananasstückchen im Mixer pürieren und mit den Ananasstückchen und
	dem Zucker in einem Topf vermischen. \\
	Die Limetten heiß waschen und trockentupfen. Die Schale abreiben, den
	Saft auspressen. Die Ingwerpflaumen in feine Streifen schneiden.
	Ingwerstreifen, Limettenschale und -saft zu der Ananasmasse geben und
	zugedeckt 2~Stunden ziehen lassen. \\
	Das Agar-Agar mit 2~Eßlöffel Wasser verrühren. Die Fruchtmasse unter
	Rühren zum Kochen bringen. Agar-Agar unterrühren und alles noch
	5~Minuten sprudelnd kochen lassen. Ingwerpulver unterrühren. Die
	Konfitüre in heiß ausgespülte Gläser füllen und verschließen. \\
      \end{zubereitung}

    \mynewsection{Bananen-Kiwi-Konfitüre}\index{Rezept>Konfitüre>Bananen-Kiwi-}

      \begin{zutaten}
        600 g & reife \myindex{Kiwi} (6 Stück) \\
	4 & \myindex{Banane}n (geschält 400 g) \\
	1 & Bio-\myindex{Limette} \\
	250 ml & \myindex{Maracujanektar} \\
	250 g & \myindex{Gelierzucker}\index{Zucker>Gelier-} 3:1 \\
      \end{zutaten}

      \reichtfuer{2}{500 ml}
      \haelt{6 Monate}
      %         kcal EW   F    KH
      \glaskcal{895}{5 g}{2 g}{208 g}

      \begin{zubereitung}
        Die Kiwis schälen und klein würfeln. Die Bananen schälen, längs
	halbieren, die Hälften in feine Scheiben schneiden. Die Limette
	heiß waschen und trockentupfen. Die Schale abreiben, den Saft
	auspressen. \\
	Die Früchte mit Limettensaft und -schale, Maracujanektar und
	Gelierzucker in einem Topf gut vermischen. Zugedeckt 2~Stunden ziehen
	lassen. \\
	Die Fruchtmischung unter Rühren aufkochen und 4~Minuten sprudelnd
	kochen lassen, dabei ständig weiterrühren. Gelierprobe nicht
	vergessen. Die Konfitüre in heiß ausgespülte Gläser füllen und
	verschließen. \\
      \end{zubereitung}

    \mynewsection{Erdbeer-Konfitüre}\index{Rezept>Konfitüre>Erdbeer-}

      \begin{zutaten}
        1 kg & kleine \myindex{Erdbeeren}\index{Beeren>Erd-} \\
	1 kg & \myindex{Gelierzucker}\index{Zucker>Gelier-} 1:1 \\
	1 & \myindex{Apfel} \\
	1 Teelöffel & abgeriebene Bio-\myindex{Orange}nschale \\
	2 Eßlöffel & \myindex{Orangenlikör}\index{Likör>Orangen-} \\
      \end{zutaten}

      \reichtfuer{4}{250 ml}
      \haelt{12 Monate}
      %         kcal EW   F    KH
      \glaskcal{618}{2 g}{1 g}{149 g}

      \begin{zubereitung}
        Die Erdbeeren putzen, waschen. Größere Früchte halbieren. Erdbeeren
	mit Gelierzucker mischen und zugedeckt 12~Stunden ziehen lassen. Die
	Erdbeeren grob zerdrücken, beiseite stellen. \\
	Den Apfel schälen, halbieren, das Kerngehäuse entfernen, Apfel fein
	reiben, mit der Orangenschale in den Topf geben, unter Rühren
	aufkochen und 2~Minuten sprudelnd kochen lassen. Erdbeeren dazugeben
	und noch weitere 2~Minuten kochen lassen, dabei immer wieder umrühren.
	Gelierprobe nicht vergessen. Orangenlikör nach Belieben unterrühren. \\
      \end{zubereitung}

    \mynewsection{Erdbeer-Rhabarber-Konfitüre}%
      \index{Rezept>Konfitüre>Erdbeer-Rhabarber-}

      \begin{zutaten}
        500 g & \myindex{Erdbeeren}\index{Beeren>Erd-} \\
	700 g & roter \myindex{Rhabarber} (geputzt 500 g) \\
	3 Eßlöffel & \myindex{Agavendicksaft}\index{Saft>Agavendick-} 
	             (aus dem Reformhaus oder aus dem Bioladen) \\
	500 g & \myindex{Gelierzucker}\index{Zucker>Gelier-} 2:1 \\
	1 & \myindex{Banane} \\
      \end{zutaten}

      \reichtfuer{4}{250 ml}
      \haelt{12 Monate}
      %         kcal  EW   F    KH
      \glaskcal{1120}{2 g}{1 g}{269 g}

      \begin{zubereitung}
        Die Erdbeeren waschen, putzen und klein würfeln. Rhabarber putzen, wenn
	nötig, entfädeln und in kleine Würfel schneiden. Alles in einen Topf
	geben, mit Agavendicksaft und Gelierzucker vermischen. Zugedeckt
	2~Stunden ziehen lassen. \\
	Die Banane schälen, in dünne Scheiben schneiden, zur Fruchtmischung
	geben. Unter Rühren aufkochen und 4~Minuten sprudelnd kochen lassen.
	Gelierprobe nicht vergessen. Die Konfitüre in heiß ausgespülte Gläser
	füllen und verschließen. \\
      \end{zubereitung}

    \mynewsection{Erdbeer-Kirsch-Konfitüre}%
      \index{Rezept>Konfitüre>Erdbeer-Kirsch-}

      \begin{zutaten}
        750 g & \myindex{Erdbeeren}\index{Beeren>Erd-} \\
	450 g & \myindex{Kirschen} (geputzt 250 g) \\
	1 & \myindex{Vanilleschote} \\
	500 g & \myindex{Gelierzucker}\index{Zucker>Gelier-} 2:1 \\
	1 Eßlöffel & \myindex{Kirschlikör}\index{Likör>Kirsch-}
	             (nach Belieben) \\
      \end{zutaten}

      \reichtfuer{4}{250 ml}
      \haelt{12 Monate}
      %         kcal EW   F    KH
      \glaskcal{605}{2 g}{1 g}{145 g}

      \begin{zubereitung}
        Die Erdbeeren putzen, waschen und klein würfeln. Die Kirschen waschen,
	entkernen und halbieren. Vanilleschote halbieren, längs aufschlitzen,
	das Mark herauskratzen und mit der Schote, dem Gelierzucker und den
	Früchten in einem Topf vermischen. Zugedeckt 2~Stunden ziehen lassen.
	\\
	Die Fruchtmischung unter Rühren aufkochen und 4~Minuten sprudelnd
	kochen lassen. Gelierprobe nicht vergessen. Likör nach Belieben
	unterrühren. Konfitüre in heiß ausgespülte Gläser füllen, jeweils
	1~Stückchen Vanilleschote mit ins Glas geben. Gläser sofort
	verschließen. \\
      \end{zubereitung}

    \mynewsection{Pfirsich-Konfitüre mit Rosmarin}%
      \index{Rezept>Konfitüre>Pfirsich- mit Rosmarin}

      \begin{zutaten}
        1,2 kg & gelbe \myindex{Pfirsiche} (entsteint 1 kg) \\
	1 kleiner Zweig & frischer \myindex{Rosmarin} \\
	1 & Bio-\myindex{Zitrone} \\
	500 g & \myindex{Gelierzucker}\index{Zucker>Gelier-} 2:1 \\
	1 Eßlöffel & \myindex{Vanillelikör}\index{Likör>Vanille-}
	             (nach Belieben) \\
      \end{zutaten}

      \reichtfuer{4}{250 ml}
      \haelt{12 Monate}
      %         kcal EW   F    KH
      \glaskcal{618}{2 g}{0 g}{150 g}

      \begin{zubereitung}
        Die Pfirsiche waschen, halbieren und entsteinen. Die Hälften in
	schmale Spalten schneiden. Rosmarin waschen, die Nadeln abzupfen und
	fein hacken. Die Zitrone waschen und trockentupfen. 1~Teelöffel Schale
	abreiben, Saft auspressen. \\
	Pfirsiche, Rosmarin, Zitronenschale und -saft mit dem Zucker in einem
	Topf vermischen. Zugedeckt 3~Stunden ziehen lassen. \\
	Die Fruchtmischung unter Rühren zum Kochen bringen und 4~Minuten
	sprudelnd kochen lassen. Gelierprobe nicht vergessen. Vanillelikör
	nach Belieben unterrühren. Die Konfitüre in heiß ausgespülte Gläser
	füllen und verschließen. \\
      \end{zubereitung}

    \mynewsection{Nektarinen-Pfirsich-Konfitüre}%
      \index{Rezept>Konfitüre>Nektarinen-Pfirsich-}

      \begin{zutaten}
        1 kg & \myindex{Nektarinen} (entsteint 750 g) \\
	700 g & \myindex{Pfirsiche} (entsteint 500 g) \\
	2 Eßlöffel & \myindex{Zitrone}nsaft\index{Saft>Zitronen-} \\
	250 ml & \myindex{Holunderblütensirup} (siehe Seite
	         \pageref{holunderbluetensirup}) \\
	500 g & \myindex{Gelierzucker}\index{Zucker>Gelier-} 3:1 \\
      \end{zutaten}

      \reichtfuer{4}{250 ml}
      \haelt{12 Monate}
      %         kcal EW   F    KH
      \glaskcal{850}{3 g}{0 g}{208 g}

      \begin{zubereitung}
        Die Nektarinen und Pfirsiche waschen, halbieren und entsteinen. Die
	Hälften der Früchte in kleine Würfel schneiden. \\
	Die Früchte mit Zitronensaft, Holunderblütensirup und Gelierzucker
	mischen. Zugedeckt 2~Stunden ziehen lassen. Ab und zu umrühren. \\
	Die Fruchtmasse unter ständigem Rühren zum Kochen bringen und 4~Minuten
	sprudelnd kochen lassen. Gelierprobe nicht vergessen. Die Konfitüre in
	heiß ausgespülte Gläser füllen und sofort verschließen. \\
      \end{zubereitung}

    \mynewsection{Brombeergelee mit Zitronengras}%
      \index{Rezept>Gelee>Brombeer- mit Zitronengras}

      \begin{zutaten}
        1,2 kg & \myindex{Brombeeren}\index{Beeren>Brom-} \\
	3 Stängel & \myindex{Zitronengras} \\
	ca. 200 ml & \myindex{Fruchtsaft}\index{Saft>Frucht-} (z.B.
	             naturtrüber \myindex{Birnensaft}\index{Saft>Birnen-}) \\
        1 & \myindex{Vanilleschote} \\
	1 Päckchen & \myindex{Zitronensäure} (5 g) \\
	1 kg & \myindex{Gelierzucker}\index{Zucker>Gelier-} 1:1 \\
      \end{zutaten}

      \reichtfuer{6}{250 ml}
      \haelt{12 Monate}
      %         kcal EW   F    KH
      \glaskcal{775}{3 g}{2 g}{183 g}

      \begin{zubereitung}
        Die Brombeeren verlesen und vorsichtig abbrausen. Mit 300~ml Wasser in
	einem Topf geben, zum Kochen bringen und die Brombeeren zugedeckt
	4--6~Minuten kochen lassen. \\
	Das Zitronengras waschen. Die Enden abschneiden. Die inneren mittleren
	Teile zuerst in Ringe schneiden, danach sehr fein hacken. \\
	Ein Sieb mit einem Tuch über einen Topf hängen. Beeren hineingießen
	und den Saft abtropfen lassen. Das Tuch zusammendrehen und auspressen,
	bis kein Saft mehr austritt. Brombeersaft mit Fruchtsaft, Zitronengras,
	Vanilleschote, Zitronensäure und dem Gelierzucker in einen Topf geben.
	\\
	Die Fruchtmasse aufkochen und unter Rühren 4~Minuten sprudelnd
	kochen lassen. Gelierprobe nicht vergessen. Vanillestange entfernen.
	Das Gelee in heiß ausgespülte Gläser füllen und gut verschließen. \\
      \end{zubereitung}

    \mynewsection{Traubengelee}\index{Rezept>Gelee>Trauben-}

      \begin{zutaten}
        1\breh{} kg & grüne \myindex{Weintrauben} \\
	2 Eßlöffel & \myindex{Zitrone}nsaft \\
	200 ml & trockener \myindex{Weißwein}\index{Wein>weiß} \\
	1 kg & \myindex{Gelierzucker}\index{Zucker>Gelier-} 1:1 \\
      \end{zutaten}

      \reichtfuer{5}{200 ml}
      \haelt{12 Monate}
      %         kcal  EW   F    KH
      \glaskcal{1010}{2 g}{1 g}{251 g}

      \begin{zubereitung}
        Die Weintrauben waschen, von den Stielen zupfen, in einen Topf geben,
	ganz kurz pürieren oder zerdrücken und mit 100~ml Wasser kurz
	aufkochen und 8--10~Minuten köcheln lassen. Ein Sieb mit einem
	sauberen Tuch auslegen. Die Fruchtmasse hineingießen und das Tuch so
	lange zusammendrehen, bis kein Saft mehr austritt. \\
	800~ml Saft abmessen, mit Zitronensaft, Weißwein und Gelierzucker in
	einen Topf geben, zum Kochen bringen und 4~Minuten sprudelnd kochen
	lassen. Gelierprobe nicht vergessen. Gelee in heiß ausgespülte Gläser
	füllen und verschließen. \\
      \end{zubereitung}

    \mynewsection{Zwetschgenmus}\index{Rezept>Mus>Zwetschgen-}

      \begin{zutaten}
        2,2 kg & \myindex{Zwetschgen} oder \myindex{Pflaumen} (entsteint 2 kg)
	         \\
        500 g & \myindex{Krümelkandis}\index{Zucker>Krümelkandis} \\
	1 & \myindex{Zimtstange} \\
	3 & \myindex{Nelken} \\
	3 Eßlöffel & \myindex{Zwetschgenwasser} (nach Belieben) \\
      \end{zutaten}

      \reichtfuer{3}{300 ml}
      \haelt{12 Monate}
      %         kcal  EW   F    KH
      \glaskcal{1020}{4 g}{1 g}{235 g}

      \begin{zubereitung}
        Die Zwetschgen waschen, halbieren und entsteinen, mit dem Krümelkandis
	mischen und zugedeckt 3~Stunden ziehen lassen. \\
	Die Zwetschgen zum Kochen bringen und bei schwacher Hitze 30~Minuten
	köcheln lassen. Dabei immer wieder umrühren. Abkühlen lassen und durch
	ein Sieb streichen. \\
	Den Backofen auf \grad{160} vorheizen. Das Zwetschgenpüree mit
	Zimtstange und Nelken in eine große ofenfeste Form füllen und im Ofen
	(Mitte) in 6~Stunden dicklich einkochen lassen. Dabei immer wieder
	umrühren. Die Zimtstange und Nelken entfernen. Das Mus in heiß
	ausgespülte Gläser füllen. Zwetschgenwasser nach Belieben darüber
	träufeln und verschließen. \\
      \end{zubereitung}

    \mynewsection{Apfelgelee}\index{Rezept>Gelee>Apfel-}

      \begin{zutaten}
        1\breh{} kg & säuerliche, nicht zu reife Äpfel\index{Aepfel=Äpfel} (z.B. Cox
	              Orange) \\
        ca. 200 ml & \myindex{Cidre} (ersatzweise Wasser) \\
	250 g & \myindex{Gelierzucker}\index{Zucker>Gelier-} 3:1 \\
	1 & \myindex{Zitrone} \\
	1 & \myindex{Vanilleschote} \\
      \end{zutaten}

      \reichtfuer{4}{250 ml}
      \haelt{12 Monate}
      %         kcal EW   F    KH
      \glaskcal{490}{1 g}{1 g}{119 g}

      \begin{zubereitung}
        Die Äpfel waschen und mit dem Kerngehäuse vierteln. Die Äpfel mit
	500~ml Wasser zugedeckt bei schwacher Hitze in 20~Minuten weich
	kochen. \\
	Ein Sieb mit einem Tuch auslegen und über einen Topf hängen. Äpfel
	hineingießen, den Saft 12~Stunden abtropfen lassen. Das Tuch
	zusammendrehen und auspressen. Mit Cidre oder Wasser auf 1~Liter auffüllen
	und mit Gelierzucker in einem Topf mischen. Die Zitrone auspressen.
	Zitronensaft und Vanilleschote zur Apfelsaftmischung geben. \\
	Die Mischung unter Rühren aufkochen und 4~Minuten sprudelnd kochen
	lassen. Gelierprobe nicht vergessen. Vanilleschote entfernen und
	vierteln. Das Gelee in heiß ausgespülte Gläser füllen, jeweils 1 Stück
	Vanilleschote ins Glas geben und verschließen. \\
      \end{zubereitung}

    \mynewsection{Quittengelee}\index{Rezept>Gelee>Quitten-}

      \begin{zutaten}
        1\breh{} kg & \myindex{Quitten} \\
	1 & \myindex{Zitrone} \\
	2 & \myindex{Orange}n \\
	1 kg & \myindex{Gelierzucker}\index{Zucker>Gelier-} 1:1 \\
      \end{zutaten}

      \reichtfuer{4}{250 ml}
      \haelt{12 Monate}
      %         kcal EW   F    KH
      \glaskcal{1160}{2 g}{2 g}{280 g}

      \begin{zubereitung}
        Die Quitten mit einem Tuch abreiben. Blüten und Stielansatz entfernen.
	Die Quitten mit der Schale und dem Kerngehäuse zuerst in Viertel,
	danach nochmals quer durchschneiden. \\
	Die Zitrone auspressen, Saft mit den Quittenstücken und ca.
	1\brev{}~Liter Wasser in einen Topf geben. Zugedeckt in 45~Minuten
	weich kochen. Ein Sieb mit einem Tuch auslegen und über einen Topf
	hängen. Die Quitten hineingießen, den Saft 12~Stunden (über Nacht)
	abtropfen lassen. Das Tuch zusammendrehen und auspressen. \\
	Die Orangen auspressen und den Quittensaft mit dem Orangensaft auf
	1~Liter auffüllen. Zusammen mit dem Gelierzucker in einem Topf
	vermischen. Die Mischung unter Rühren aufkochen und 4~Minuten sprudelnd
	kochen lassen. Gelierprobe nicht vergessen. Das Quittengelee in heiß
	ausgespülte Gläser füllen und verschließen. \\
      \end{zubereitung}

    \mynewsection{Winterkonfitüre}\index{Rezept>Konfitüre>Winter-}

      \begin{zutaten}
        100 g & getrocknete \myindex{Feigen} \\
	100 g & getrocknete \myindex{Datteln} \\
	100 g & getrocknete \myindex{Aprikosen} \\
	3 Eßlöffel & \myindex{Agavendicksaft}\index{Saft>Agavendick-}
	             (aus dem Reformhaus), ersatzweise 2 Eßlöffel Honig
		     oder Ahornsirup \\
        150 ml & naturtrüber \myindex{Birnensaft}\index{Saft>Birnen-}
	         (ersatzweise \myindex{Apfelsaft}\index{Saft>Apfel-}) \\
        700 g & Äpfel\index{Aepfel=Äpfel} (geputzt 500 g) \\
        700 g & \myindex{Birne}n (geputzt 500 g) \\
	500 g & \myindex{Gelierzucker}\index{Zucker>Gelier-} 2:1 \\
      \end{zutaten}

      \reichtfuer{4}{250 ml}
      \haelt{12 Monate}
      %         kcal EW   F    KH
      \glaskcal{890}{4 g}{1 g}{218 g}

      \begin{zubereitung}
        Die Feigen, Datteln und Aprikosen klein würfeln. Früchte im
	Agavendicksaft und Birnensaft in einem Topf erwärmen und bei mittlerer
	Hitze ca. 10~Minuten ziehen lassen. \\
	Die Äpfel und Birnen schälen, vierteln, dabei das Kerngehäuse
	entfernen. Die Viertel quer in Scheiben schneiden und mit dem
	Gelierzucker zur Trockenfruchtmischung geben. Zugedeckt 2~Stunden
	ziehen lassen. \\
	Die Fruchtmischung unter Rühren zum Kochen bringen und 4~Minuten
	sprudelnd kochen lassen. Nach der Gelierprobe in heiß ausgepülte
	Gläser füllen und verschließen. Kühl aufbewahren. \\
	Gut als Füllung für Bratäpfel. \\
      \end{zubereitung}
