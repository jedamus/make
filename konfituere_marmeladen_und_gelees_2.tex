  \mynewchapter{Marmeladen und Gelees II}

  % aus Marmeladen und Gelees, Früchtchen auf Vorrat, GU

    \mynewsection{Pannenhilfe}

      \begin{einleitung}
        Gegen trübes Gelee: Gelee wird trüb, wenn Sie naturtrüben Saft
	verwenden oder die Früchte zu stark auspressen. Wer das vermeiden will,
	muß die Geduld aufbringen, die Früchte wirklich über Nacht abtropfen
	zu lassen und dann nur leicht auszupressen. \\
	Wenn's nicht fest wird: Konfitüren aus Früchten mit wenig natürlichem
	Pektingehalt, z.B. Melonen, brauchen etwas länger, um fest zu werden.
	Außerdem dauert das Festwerden im Glas ohnehin mehrere Stunden. Ist die
	Konfitüre im Glas allerdings auch nach einigen Tagen noch flüssig,
	hilft nur eins: wieder in den Topf damit, etwas Gelierzucker dazu und
	noch einmal kochen lassen. \\
	Wenn's zu fest wird: Viel können Sie nicht mehr machen, streichen läßt
	sich die Konfitüre trotzdem noch. Einzige Möglichkeit: die Konfitüre,
	die Sie sofort essen möchten, in einem Schälchen mit etwas Saft
	verrühren und verdünnt auf den Frühstückstisch stellen. \\
	Einfüllen: Egal, ob Konfitüre oder Gelee, eingefüllt wird beides immer
	kochend heiß. Dann die Gläser sofort verschließen und kurz auf den
	Kopf stellen. Dadurch bildet sich ein Vakuum in den Gläsern. Und so
	verteilen sich enthaltene Fruchtstückchen schön gleichmäßig. \\
	Lagern: Alle Gläser sollten Sie für den besseren Überblick beschriften:
	Inhalt und Einfülldatum draufschreiben. Nach dem Abkühlen der Gläser
	zum Aufbewahren in den kühlen und dunklen Keller stellen. Die meisten
	Konfitüren halten sich ein Jahr. Abweichende Haltbarkeitshinweise
	finden Sie bei den Rezepten. \\
	Schimmel: Wurde die Konfitüre, auf der sich Schimmel gebildet hat, mit
	viel Zucker gemacht oder mit Gelierzucker und Konservierungsmitteln,
	können Sie die obere Schicht großzügig abheben und den Rest rasch
	verbrauchen. Wenn die Konfitüre wenig Zucker enthält, muß der ganze
	Glasinhalt in den Abfall --- der Gesundheit zuliebe. \\
      \end{einleitung}

    \mynewsection{Konfitüre aus TK-Beeren}

      \begin{zutaten}
        1 Paket & \myindex{TK-Beeren}\index{Beeren} (gemischt; 300 g) \\
	\brea{} l & trockener \myindex{Rotwein}\index{Wein>rot} \\
	215 g & \myindex{Gelierzucker}\index{Zucker>Gelier-} (2 Teile Frucht,
	        1 Teil Zucker) \\
        1 Eßlöffel & \myindex{Zitrone}nsaft \\
      \end{zutaten}

      \reichtfuer{2}{300 ml}
      \haelt{1 Jahr}

      \begin{zubereitung}
        Die Beeren mit Rotwein in einem Topf bei mittlerer Hitze auftauen
	lassen. \\
	Gelierzucker und Zitronensaft untermischen, alles bei mittlerer Hitze
	offen 4~Minuten kochen lassen. \\
	Die Gelierprobe machen. Die Konfitüre in heiß ausgespülte Gläser mit
	Twist-Off-Deckeln füllen und sofort verschließen. Kühl aufbewahren. \\
      \end{zubereitung}

    \mynewsection{Rohe Himbeerkonfitüre}

      \begin{zutaten}
        200 g & frische aromatische \myindex{Himbeeren}\index{Beeren>Him-} \\
	200 g & \myindex{Gelierzucker}\index{Zucker>Gelier-} (1:1) \\
	2--3 Zweige & \myindex{Zitronenmelisse} \\
      \end{zutaten}

      \reichtfuer{2}{200 ml}
      \haelt{2--4 Wochen}

      \begin{zubereitung}
        Die Himbeeren nicht waschen, sondern nur verlesen. Mit Gelierzucker
	15~Minuten durchmixen, bis die Masse fester wird. \\
	Inzwischen die Melisse waschen und trockenschwenken und die Blättchen
	grob hacken. Zur Konfitüre geben und noch kurz weitermixen. \\
	Die Konfitüre in heiß ausgespülte Gläser mit Twist-Off-Deckeln füllen
	und sofort verschließen. Im Kühlschrank aufbewahren. \\
      \end{zubereitung}

    \mynewsection{Erdbeer-Konfitüre II}

      \begin{zutaten}
        1 kg & \myindex{Erdbeeren}\index{Beeren>Erd-} \\
	500 g & \myindex{Gelierzucker}\index{Zucker>Gelier-} (2:1) \\
	2 Eßlöffel & \myindex{Zitrone}nsaft \\
	2 Eßlöffel & \myindex{Orangenlikör} (z.B. ,,Grand Marnier'') \\
      \end{zutaten}

      \reichtfuer{6}{\brev{} l}
      \haelt{1 Jahr}

      \begin{zubereitung}
        Die Erdbeeren vorsichtig waschen und mit Küchenpapier trockentupfen.
	Die Kelchblätter herauszupfen oder -schneiden und die Beeren in sehr
	kleine Stücke schneiden. \\
	Die Früchte in einem Topf mit dem Gelierzucker mischen und 2~Stunden
	lang Saft ziehen lassen. Dabei immer mal wieder durchrühren. Danach
	die Früchte mit dem Kartoffelstampfer etwas zerdrücken. \\
	Die Masse mit Zitronensaft und Likör mischen und unter Rühren bei
	starker Hitze zum Kochen bringen. Die Konfitüre bei mittlerer Hitze
	unter Rühren 4~Minuten kochen lassen, bis sie geliert. \\
	Die Gelierprobe machen. Die Konfitüre in heiß ausgespülte Gläser mit
	Twist-Off-Deckeln füllen und verschließen. Kühl aufbewahren. \\
      \end{zubereitung}

    \mynewsection{Erdbeer-Rhabarber-Konfitüre II}

      \begin{zutaten}
        700 g & \myindex{Erdbeeren}\index{Beeren>Erd-} \\
	300 g & \myindex{Rhabarber} \\
	500 g & \myindex{Gelierzucker}\index{Zucker>Gelier-} (2:1) \\
      \end{zutaten}

      \reichtfuer{6}{\brev{} l}
      \haelt{1 Jahr}

      \begin{zubereitung}
        Die Erdbeeren vorsichtig waschen und mit Küchenpapier trockentupfen.
	Die Kelchblätter herauszupfen oder -schneiden und die Beeren in kleine
	Stücke schneiden. Den Rhabarber waschen, die Enden abschneiden. Wenn
	sich Fäden lösen, diese gleich mit abziehen. Die Rhabarberstangen in
	dünne Scheiben schneiden. \\
	Die Früchte in einem Topf gut mit dem Gelierzucker mischen und
	2~Stunden lang Saft ziehen lassen. Dabei immer mal wieder durchrühren.
	Danach die Früchte mit einem Kartoffelstampfer etwas zerdrücken. \\
	Die Masse unter Rühren bei starker Hitze zum Kochen bringen. Die
	Konfitüre bei mittlerer Hitze unter Rühren 4~Minuten kochen lassen,
	bis sie geliert. \\
	Die Gelierprobe machen. Die Konfitüre in heiß ausgespülte Gläser mit
	Twist-Off-Deckeln füllen und verschließen. Kühl aufbewahren. \\
      \end{zubereitung}

    \mynewsection{Johannisbeergelee mit Cassis}

      \begin{zutaten}
        1 kg & \myindex{Johannisbeeren}\index{Beeren>Johannis-} (am besten
	       schwarz, rot und weiß gemischt) \\
        100 ml & \myindex{Cassis} (Likör aus schwarzen Johannisbeeren) \\
	450 g & \myindex{Gelierzucker}\index{Zucker>Gelier-} (2:1) \\
      \end{zutaten}

      \reichtfuer{6}{\brev{} l}
      \haelt{1 Jahr}

      \begin{zubereitung}
        Die Johannisbeeren waschen, mit den Rispen in einen Topf füllen und
	300~ml Wasser angießen. \\
	Das Wasser zum Kochen bringen, die Beeren zugedeckt 5~Minuten kochen
	lassen. Ein Sieb mit einem frischen Tuch auslegen und über eine
	Schüssel hängen. Die Beeren hineingießen und den Saft abtropfen lassen.
	Das Tuch zusammendrehen und kräftig ausdrücken, bis kein Saft mehr
	austritt. \\
	Den Saft mit Cassis mischen, mit Wasser auf 1~Liter auffüllen und
	wieder in den Topf gießen. Den Gelierzucker untermischen. Alles unter
	Rühren bei starker Hitze zum Kochen bringen. Die Masse unter Rühren
	4~Minuten kochen lassen, bis sie geliert. \\
	Die Gelierprobe machen. Das Johannisbeergelee in heiß ausgespülte
	Gläser mit Twist-Off-Deckeln füllen und sofort verschließen. Kühl
	aufbewahren. \\
      \end{zubereitung}

    \mynewsection{Holundergelee mit Krokant}

      \begin{zutaten}
        1 kg & reife \myindex{Holunderbeeren}\index{Beeren>Holunder-} \\
	100 g & gehäutete \myindex{Mandel}n \\
	50 g & \myindex{Zucker} \\
	500 g & \myindex{Gelierzucker}\index{Zucker>Gelier-} (2:1) \\
	2 Eßlöffel & \myindex{Zitrone}nsaft \\
      \end{zutaten}

      \reichtfuer{6}{\brev{} l}
      \haelt{1 Jahr}

      \begin{zubereitung}
        Die Holunderbeeren waschen, mit 300~ml Wasser zum Kochen bringen, die
	Beeren zugedeckt 5~Minuten kochen, dann etwas abkühlen lassen. Ein
	Sieb mit einem frischen Tuch auslegen und über eine Schüssel hängen.
	Die Beeren hineingießen und den Saft abtropfen lassen. Das Tuch
	zusammendrehen, bis kein Saft mehr austritt. \\
	Die Mandeln fein hacken. Den Zucker in einem Topf bei mittlerer Hitze
	schmelzen lassen, bis er flüssig und leicht gebräunt ist. Die Mandeln
	untermischen und goldbraun werden lassen. Auf einem Teller beiseite
	legen. \\
	Holundersaft mit Wasser auf 1~Liter auffüllen und wieder in den Topf
	gießen. Gelierzucker und Zitronensaft untermischen. Alles unter
	Rühren bei starker Hitze zum Kochen bringen. Das Gelee unter Rühren
	4~Minuten kochen lassen, Mandeln unterziehen. \\
	Die Gelierprobe machen. Das Gelee in heiß ausgespülte Gläser mit
	Twist-Off-Deckeln füllen, sofort verschließen. Kühl aufbewahren. \\
      \end{zubereitung}

    \mynewsection{Stachelbeer-Holunder-Konfitüre}

      \begin{zutaten}
        1 kg & \myindex{Stachelbeeren}\index{Beeren>Stachel-} \\
	1 Eßlöffel & \myindex{Zitrone}nsaft (nur, wenn die Beeren sehr süß
	             sind) \\
        \breh{} l & \myindex{Holunderblütensirup} (siehe Seite
	            \pageref{holunderbluetensirup}) \\
	ca. 150 g & \myindex{Zucker} (je nachdem, wie sauer oder süß die
	            Beeren sind) \\
        2 Teelöffel & \myindex{Apfelpektin} (aus dem Reformhaus) \\
      \end{zutaten}

      \reichtfuer{6}{\brev{} l}
      \haelt{6 Monate}

      \begin{zubereitung}
        Die Stachelbeeren vorsichtig waschen, dann die Blüten- und
	Stielansätze abknipsen. Eventuell den Zitronensaft mit Stachelbeeren
	und Holunderblütensirup, Zucker und Pektin in einem Topf bei mittlerer
	Hitze zum Kochen bringen. \\
	Die Beeren zugedeckt bei mittlerer Hitze 15~Minuten kochen lassen, bis
	sie schön weich sind. Mit einem Holzlöffel durchrühren, damit sie
	etwas zerfallen. \\
	Die Gelierprobe machen. Die Stachelbeerkonfitüre in heiß ausgespülte
	Gläser mit Twist-Off-Deckeln füllen und sofort verschließen. \\
      \end{zubereitung}

    \mynewsection{Brombeer-Konfitüre mit Rosmarin}

      \begin{zutaten}
        1 kg & \myindex{Brombeeren}\index{Beeren>Brom-} \\
	500 g & \myindex{Gelierzucker}\index{Zucker>Gelier-} (2:1) \\
	1 & unbehandelte \myindex{Zitrone} (ersatzweise \myindex{Orange}) \\
	2 Zweige & frischer \myindex{Rosmarin} \\
      \end{zutaten}

      \reichtfuer{6}{\brev{} l}
      \haelt{1 Jahr}

      \begin{zubereitung}
        Die Brombeeren vorsichtig waschen und mit Küchenpapier trockentupfen.
	Die Früchte in einem Topf mit dem Gelierzucker mischen und 2~Stunden
	lang Saft ziehen lassen. Dabei immer mal wieder durchrühren. \\
	Die Brombeeren mit dem Kartoffelstampfer etwas zerdrücken. Die Zitrone
	heiß waschen und abtrocknen. Die Schale fein abreiben, den Saft
	auspressen. Den Rosmarin waschen, trockenschwenken und die Nadeln grob
	hacken. Mit dem Zitronensaft und der -schale zu den Brombeeren geben. \\
	Die Masse unter Rühren bei starker Hitze zum Kochen bringen. Die
	Konfitüre bei mittlerer Hitze unter Rühren 4~Minuten kochen lassen, bis
	sie geliert. \\
	Die Gelierprobe machen. Die Konfitüre in heiß ausgespülte Gläser mit
	Twist-Off-Deckeln füllen und sofort verschließen. Die Konfitüre kühl
	aufbewahren. \\
      \end{zubereitung}

    \mynewsection{Apfelsaftgelee mit Minze}

      \begin{zutaten}
        350 ml & naturtrüber \myindex{Apfelsaft} \\
	2 Eßlöffel & \myindex{Calvados} (nach Belieben) \\
	150 g & \myindex{Gelierzucker}\index{Zucker>Gelier-} (2:1) \\
	ein paar & Zweige frische \myindex{Minze} \\
      \end{zutaten}

      \reichtfuer{2}{\brev{} l}

      \begin{zubereitung}
        Den Apfelsaft nach Belieben mit Calvados und Gelierzucker in einem
	Topf mischen und unter Rühren bei starker Hitze zum Kochen bringen. \\
	Die Minze in feine Streifen schneiden. Apfelsaft unter Rühren 4~Minuten
	kochen lassen. Die Gelierprobe machen. \\
	Die Minze untermischen, alles aufkochen lassen. Das Gelee sofort in
	heiß ausgespülte Gläser mit Twist-Off-Deckeln füllen und sofort
	verschließen. \\
      \end{zubereitung}

    \mynewsection{Birnensaftgelee mit Mohn}

      \begin{zutaten}
        2 Eßlöffel & \myindex{Mohn}samen \\
	50 ml & \myindex{Williams Birnengeist} (ersatzweise Birnensaft) \\
	350 ml & naturtrüber \myindex{Birnensaft} \\
	150 g & \myindex{Gelierzucker}\index{Zucker>Gelier-} (2:1) \\
      \end{zutaten}

      \reichtfuer{2}{\brev{} l}

      \begin{zubereitung}
        Den Mohn im Blitzhacker mahlen. Mit 50 ml Wasser und Birnengeist zum
	Kochen bringen. Auf der abgeschalteten Kochfläche zugedeckt quellen
	lassen. \\
	Den Birnensaft mit Gelierzucker bei starker Hitze zum Kochen bringen.
	Den Mohn untermischen, bei mittlerer Hitze 4~Minuten kochen lassen,
	bis alles geliert. \\
	Die Gelierprobe machen. Das Gelee sofort in Gläser füllen und
	verschließen. \\
      \end{zubereitung}

    \mynewsection{Birnen-Rotwein-Konfitüre}

      \begin{zutaten}
        900 g & saftige \myindex{Birne}n \\
	1 große & \myindex{Zitrone} \\
	500 g & \myindex{Gelierzucker}\index{Zucker>Gelier-} (2:1) \\
	\brev{} l & trockener \myindex{Rotwein}\index{Wein>rot} \\
	1 kräftige Prise & \myindex{Zimt} gemahlen \\
	1 kräftige Prise & \myindex{Nelken} gemahlen \\
	1 kräftige Prise & \myindex{Piment} gemahlen \\
      \end{zutaten}

      \reichtfuer{6}{\brev{} l}
      \haelt{1 Jahr}

      \begin{zubereitung}
        Die Birnen schälen, vierteln, vom Kerngehäuse befreien und sehr klein
	würfeln. Den Saft der Zitrone auspressen, mit Birnen und Zucker mischen
	und über Nacht Saft ziehen lassen. \\
	Am nächsten Tag den Wein dazugießen. Zimt, Nelken und Piment
	untermischen. Alles bei starker Hitze unter Rühren zum Kochen bringen,
	bei mittlerer Hitze 4~Minuten kochen lassen. \\
	Die Gelierprobe machen. Die Konfitüre in heiß ausgespülte Gläser mit
	Twist-Off-Deckeln füllen und sofort verschließen. \\
      \end{zubereitung}

    \mynewsection{Quittengelee II}

      \begin{zutaten}
        2 kg & \myindex{Quitten} \\
	2 & unbehandelte \myindex{Zitrone}n \\
	500 g & \myindex{Gelierzucker}\index{Zucker>Gelier-} (2:1) \\
      \end{zutaten}

      \reichtfuer{6}{\brev{} l}
      \haelt{1 Jahr}

      \begin{zubereitung}
        Die Quitten mit einem feuchten Tuch abreiben, um den Flaum zu
	entfernen. Die Quitten mit den Kerngehäusen in Stücke schneiden und in
	einen Topf legen. Die Zitronen heiß waschen und die Schale spiralförmig
	abschneiden. Mit 1~Liter Wasser zu den Quitten geben und zum Kochen
	bringen. \\
	Die Quitten bei schwacher Hitze zugedeckt 45~Minuten musig kochen
	lassen. \\
	Ein Sieb mit einem sauberen Tuch auskleiden und über eine Schüssel
	hängen. Die Quittenmasse hineinfüllen und den Saft über Nacht
	abtropfen lassen. Am nächsten Tag das Tuch zusammendrehen; nicht zu
	fest, sonst wird das Gelee trüb. \\
	Den Saft der Zitronen auspressen. Mit dem Quittensaft mischen und mit
	Wasser auf 1~Liter auffüllen. Den Gelierzucker untermischen und alles
	zum Kochen bringen. Bei mittlerer Hitze 4~Minuten kochen lassen. \\
	Die Gelierprobe machen. In heiß ausgespülte Gläser füllen und
	verschließen. \\
      \end{zubereitung}

    \mynewsection{Quittenmus}

      \begin{zutaten}
        2 kg & \myindex{Quitten} \\
	1 & \myindex{Zitrone} \\
	1 Prise & \myindex{Zimt} gemahlen \\
	1 Prise & \myindex{Nelken} gemahlen \\
	400 g & \myindex{Gelierzucker}\index{Zucker>Gelier-} (2:1) \\
      \end{zutaten}

      \reichtfuer{6}{\brev{} l}
      \haelt{1 Jahr}

      \begin{zubereitung}
        Die Quitten mit einem feuchten Tuch abreiben, um den Flaum zu
	entfernen. Die Quitten vierteln, schälen und von den Kerngehäusen
	befreien. Einige Kerne herauslösen und in ein kleines Tuch binden. \\
	Mit 300~ml Wasser zu den Quitten geben und erhitzen. Die Quitten bei
	schwacher Hitze zugedeckt 1\breh{}~Stunden garen, bis sie schön weich
	sind. Zwischendurch vorsichtig umrühren, damit nichts anbrennt. \\
	Das Tuch mit den Kernen aus dem Topf fischen. Den Saft der Zitrone
	auspressen. Die Quitten mit dem Pürierstab pürieren. Zitronensaft,
	Zimt, Gewürznelken und Gelierzucker zum Mus geben und alles
	4~Minuten kochen lassen. \\
	Die Gelierprobe machen. Das Quittenmus in heiß ausgespülte Gläser
	mit Twist-Off-Deckeln füllen und sofort verschließen. \\
      \end{zubereitung}

    \mynewsection{Apfelgelee mit Limettenschale}

      \begin{zutaten}
        2 kg & säuerliche Äpfel\index{Aepfel=Äpfel} \\
	3 & \myindex{Limette}n \\
	450 g & \myindex{Gelierzucker}\index{Zucker>Gelier-} (2:1) \\
      \end{zutaten}

      \reichtfuer{6}{\brev{} l}
      \haelt{1 Jahr}

      \begin{zubereitung}
        Die Äpfel waschen, trockenreiben und mit dem Kerngehäuse in Achtel
	schneiden. \\
	Die Äpfel mit 1~Liter Wasser zum Kochen bringen. Zugedeckt bei
	schwacher Hitze 20~Minuten garen, bis die Äpfel wie Mus sind. \\
	Ein Sieb mit einem Tuch auskleiden und über eine Schüssel hängen. Die
	Apfelmasse darin über Nacht abtropfen lassen. Am nächsten Tag das Tuch
	zusammendrehen; nicht zu fest. \\
	Limetten heiß waschen und abtrocknen. Die Schale dünn abschneiden und
	in Streifen schneiden, den Saft auspressen. Apfel- und Limettensaft
	mischen und mit Wasser auf 1~Liter auffüllen. \\
	Saft und Limettenschale mit Gelierzucker mischen und unter Rühren bei
	starker Hitze zum Kochen bringen. Den Saft unter Rühren 4~Minuten
	kochen lassen. \\
	Die Gelierprobe machen. Das Gelee in heiß ausgespülte Gläser mit
	Twist-Off-Deckeln füllen und sofort verschließen. \\
      \end{zubereitung}

    \mynewsection{Apfel-Zwetschgen-Konfitüre}

      \begin{zutaten}
        600 g & säuerliche Äpfel\index{Aepfel=Äpfel} \\
	400 g & \myindex{Zwetschgen} \\
	1 große & \myindex{Zitrone} \\
	500 g & \myindex{Gelierzucker}\index{Zucker>Gelier-} (2:1) \\
	\brea{} l & \myindex{Cidre} (ersatzweise naturtrüber Apfelsaft) \\
	2 Eßlöffel & \myindex{Ahornsirup} \\
      \end{zutaten}

      \reichtfuer{6}{\brev{} l}
      \haelt{1 Jahr}

      \begin{zubereitung}
        Die Äpfel schälen, vierteln, vom Kerngehäuse befreien und sehr klein
	würfeln. Die Zwetschgen waschen, entsteinen und ebenfalls klein
	schneiden. Den Saft der Zitrone auspressen. Die Früchte mit
	Zitronensaft und Gelierzucker mischen und über Nacht Saft ziehen
	lassen. \\
	Am nächsten Tag den Cidre und den Ahornsirup dazugeben. Alles bei
	starker Hitze zum Kochen bringen, bei mittlerer Hitze 4~Minuten
	kochen lassen. \\
	Die Gelierprobe machen. Die Konfitüre in heiß ausgespülte Gläser mit
	Twist-Off-Deckeln füllen und sofort verschließen. \\
      \end{zubereitung}

    \mynewsection{Birnenkraut}

      \begin{zutaten}
        2\breh{} kg & \myindex{Birne}n \\
	\breh{} l & naturtrüber \myindex{Apfelsaft} \\
	1 Prise & \myindex{Zimt} gemahlen \\
	1 Prise & \myindex{Kardamom} gemahlen \\
      \end{zutaten}

      \reichtfuer{2}{200 ml}
      \haelt{3 Monate}

      \begin{zubereitung}
        Die Birnen waschen und abtrocknen, alle schlechten Stellen entfernen.
	Die Birnen mit den Kerngehäusen in Stücke schneiden. \\
	Die Birnenstücke mit Apfelsaft, Zimt und Kadamom in einem Topf mischen
	und zum Kochen bringen. Offen bei schwacher Hitze 1~Stunde köcheln
	lassen, bis sie musig geworden sind. Dabei immer mal wieder
	durchrühren. \\
	Ein Sieb mit einem frischen Tuch auskleiden und über eine Schüssel
	hängen, die Birnenmasse hineingeben und über Nacht abtropfen lassen.
	Dann sehr gut ausdrücken. \\
	Den Birnensaft in einen Topf gießen. Den Saft zum Kochen bringen und
	offen bei mittlerer Hitze in 1\breh{}~Stunden einkochen lassen, bis er
	dickflüssig ist. Es bleibt nur etwa ein Drittel übrig. \\
	Das Birnenkraut in heiß ausgespülte Gläser mit Twist-Off-Deckeln
	füllen und sofort verschließen. \\
      \end{zubereitung}

    \mynewsection{Einfache Aprikosen-Konfitüre}

      \begin{zutaten}
        100 g & getrocknete Soft-\myindex{Aprikosen} \\
	1,1 kg & vollreife \myindex{Aprikosen} \\
	1 große & \myindex{Zitrone} (nur den Saft) \\
	500 g & \myindex{Gelierzucker}\index{Zucker>Gelier-} (2:1) \\
      \end{zutaten}

      \reichtfuer{6}{\brev{} l}
      \haelt{1 Jahr}

      \begin{zubereitung}
        Getrocknete Aprikosen sehr klein würfeln. Frische Aprikosen waschen,
	entsteinen und ebenfalls klein schneiden. Beide mit Zitronensaft und
	Gelierzucker mischen und kurz Saft ziehen lassen. \\
	Alles bei starker Hitze unter Rühren zum Kochen bringen und bei
	mittlerer Hitze 4~Minuten kochen lassen. Die Gelierprobe machen. In
	heiß ausgespülte Gläser füllen und verschließen. Kühl aufbewahren. \\
      \end{zubereitung}

    \mynewsection{Kirschkonfitüre mit Sherry}

      \begin{zutaten}
        1 kg & \myindex{Kirschen} (am besten Sauerkirschen) \\
	1 & \myindex{Vanilleschote} \\
	500 g & \myindex{Gelierzucker}\index{Zucker>Gelier-} (2:1) \\
	100 ml & trockener \myindex{Sherry} \\
      \end{zutaten}

      \reichtfuer{4}{\brev{} l}
      \haelt{1 Jahr}

      \begin{zubereitung}
        Die Kirschen waschen, entstielen, entsteinen und in einen Topf geben.
	Die Vanilleschote der Länge nach aufschlitzen. Das Mark mit einem
	Messer herauskratzen, mit der Schote zu den Kirschen geben. Den
	Gelierzucker untermischen und die Kirschen 2--3~Stunden oder über
	Nacht Saft ziehen lassen. \\
	Die Kirschen mit dem Kartoffelstampfer grob zerstampfen. Sherry
	dazugießen und alles bei starker Hitze unter Rühren zum Kochen bringen.
	Die Konfitüre bei mittlerer Hitze 4~Minuten kochen lassen. \\
	Die Gelierprobe machen. Die Vanilleschote aus der Konfitüre fischen.
	Die Konfitüre in heiß ausgespülte Gläser mit Twist-Off-Deckeln füllen
	und verschließen. \\
      \end{zubereitung}

    \mynewsection{Zwetschgenmus II}

      \begin{zutaten}
        2 kg & reife \myindex{Zwetschgen} \\
	150 g & \myindex{Zucker} \\
	4 Eßlöffel & \myindex{Zwetschgenwasser} (nach Belieben) \\
	1 Prise & \myindex{Zimt} \\
	1 Prise & \myindex{schwarzer Pfeffer}\index{Pfeffer>schwarz} \\
      \end{zutaten}

      \reichtfuer{4}{\brev{} l}
      \haelt{2 Monate}

      \begin{zubereitung}
        Die Zwetschgen waschen, halbieren und entsteinen. \\
	Die Zwetschgen in einem Topf zum Kochen bringen. Dann zugedeckt bei
	schwacher Hitze 2~Stunden garen, bis sie sehr weich sind. Dabei immer
	mal wieder durchrühren, damit nichts anbrennt. \\
	Die Zwetschgen mit dem Pürierstab glatt pürieren. Zucker und
	Zwetschgenwasser, Zimt und Pfeffer dazugeben und offen 30~Minuten
	garen, bis die Masse dickflüssig ist. \\
	Das Zwetschgenmus in heiß ausgespülte Gläser mit Twist-Off-Deckeln
	füllen und sofort verschließen. \\
      \end{zubereitung}

    \mynewsection{Pfirsich-Zitronen-Konfitüre}

      \begin{zutaten}
        1,2 kg & \myindex{Pfirsiche} (etwa 10 Stück, geputzt knapp 1 kg) \\
	500 g & \myindex{Gelierzucker}\index{Zucker>Gelier-} (2:1) \\
	2 & unbehandelte \myindex{Zitrone}n \\
	2 Eßlöffel & \myindex{Limoncello} (Zitronenlikör; nach Belieben) \\
      \end{zutaten}

      \reichtfuer{6}{\brev{} l}
      \haelt{1 Jahr}

      \begin{zubereitung}
        Die Pfirsiche waschen und häuten. Bei ganz reifen läßt sich die Haut
	problemlos abziehen, andere kurz überbrühen, abschrecken und häuten.
	Die Pfirsiche halbieren, entsteinen und in kleine Würfel schneiden. Mit
	dem Zucker mischen und über Nacht Saft ziehen lassen. \\
	Am nächsten Tag die Zitronen heiß waschen und abtrocknen, die Schale
	fein abreiben. 1~Zitrone so schälen, daß auch die weiße Haut mit
	entfernt wird. Das Zitronenfleisch aus den Trennwänden herauslösen und
	klein schneiden. Dabei alle Kerne aussortieren. \\
	Zitronenfleisch und die -schale sowie eventuell den Limoncello zu den
	Pfirsichen geben und alles verrühren. Die Mischung bei starker Hitze
	unter Rühren zum Kochen bringen. Dann bei mittlerer Hitze 4~Minuten
	kochen lassen. \\
	Die Gelierprobe machen. Die Konfitüre in heiß ausgespülte Gläser mit
	Twist-Off-Deckeln füllen und sofort verschließen. \\
      \end{zubereitung}

    \mynewsection{Reineclauden-Konfitüre}

      \begin{zutaten}
        1,2 kg & \myindex{Reineclauden} \\
	500 g & \myindex{Gelierzucker}\index{Zucker>Gelier-} (2:1) \\
	3 Eßlöffel & \myindex{Zitrone}nsaft \\
      \end{zutaten}

      \reichtfuer{6}{\brev{} l}
      \haelt{1 Jahr}

      \begin{zubereitung}
        Die Reineclauden waschen, halbieren, entsteinen und in kleine Stücke
	schneiden. Mit dem Gelierzucker mischen und über Nacht Saft ziehen
	lassen. \\
	Am nächsten Tag mit dem Zitronensaft mischen und bei starker Hitze
	unter Rühren zum Kochen bringen. Bei mittlerer Hitze 4~Minuten kochen
	lassen, dabei immer mal wieder umrühren. \\
	Die Gelierprobe machen. Die Konfitüre in heiß ausgespülte Gläser mit
	Twist-Off-Deckeln füllen und sofort verschließen. Die Konfitüre kühl
	aufbewahren. \\
      \end{zubereitung}

    \mynewsection{Vier-Frucht-Konfitüre}

      \begin{zutaten}
        3 & \myindex{Pfirsiche} (ca. 300 g) \\
	4 & \myindex{Kiwi}s (ca. 340 g) \\
	1 & \myindex{Banane} (ca. 180 g) \\
	4 & frische \myindex{Feigen} (ca. 200 g) \\
	1 & \myindex{Limette} \\
	1 & unbehandelte \myindex{Orange} \\
	500 g & \myindex{Gelierzucker}\index{Zucker>Gelier-} (2:1) \\
      \end{zutaten}

      \reichtfuer{6}{\brev{} l}
      \haelt{1 Jahr}

      \begin{zubereitung}
        Die Pfirsiche waschen oder häuten, halbieren und entsteinen. Die Kiwis
	schälen, halbieren und den harten Strunk am Stielansatz abschneiden.
	Die Banane schälen. Die Feigen waschen, vom Stielansatz befreien,
	vierteln und nur dann schälen, wenn die Schale sehr dick ist. \\
	Alle Früchte in kleine Stücke schneiden. Die Früchte sollten zusammen
	etwa 900~Gramm wiegen. Die Limette und Orange heiß waschen und
	abtrocknen. Die Schale dünn abschneiden und fein hacken. Mit den
	Früchten und dem Gelierzucker mischen und über Nacht Saft ziehen
	lassen. \\
	Am nächsten Tag den Saft von Limette und Orange auspressen, unter die
	Fruchtmischung mengen. Alles bei starker Hitze unter Rühren zum
	Kochen bringen. Bei mittlerer Hitze 4~Minuten kochen lassen. Dabei
	immer mal wieder umrühren. \\
	Die Gelierprobe machen. Die Konfitüre in heiß ausgespülte Gläser mit
	Twist-Off-Deckeln füllen und sofort verschließen. Die Konfitüre kühl
	aufbewahren. \\
      \end{zubereitung}

    \mynewsection{Feigen-Rhabarber-Konfitüre}

      \begin{zutaten}
        150 g & getrocknete ungeschwefelte \myindex{Feigen} \\
	500 g & \myindex{Rhabarber} \\
	500 g & \myindex{Zucker} \\
	2 Eßlöffel & \myindex{Zitrone}nsaft \\
	2 Eßlöffel & \myindex{Rum} oder \myindex{Whisky} (nach Belieben) \\
      \end{zutaten}

      \reichtfuer{5}{\brev{} l}
      \haelt{1 Jahr}

      \begin{zubereitung}
        Die Feigen waschen und in kleine Stücke schneiden. Den Rhabarber
	waschen, die Enden abschneiden. Wenn sich dabei Fäden lösen, diese
	gleich mit abziehen. Die Rhabarberstangen in schmale Scheiben
	schneiden. \\
	Feigen und Rhabarber mit dem Zucker mischen und über Nacht Saft ziehen
	lassen. \\
	Die Mischung in einem Topf füllen, Zitronensaft und eventuell Rum oder
	Whisky dazugeben. Alles zum Kochen bringen und bei mittlerer Hitze
	15~Minuten kochen lassen, bis die Masse fester wird. \\
	Die Gelierprobe machen. Die Konfitüre in heiß ausgespülte Gläser mit
	Twist-Off-Deckeln füllen und sofort verschließen. \\
      \end{zubereitung}

