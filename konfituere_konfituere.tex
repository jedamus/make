  \mynewchapter{Konfitüre}

    \mynewsection{Holunder-Konfitüre}\index{Rezept>Konfitüre>Holunder-}

      \begin{zutaten}
	1 kg & reife \myindex{Holunderbeeren}\index{Beeren>Holunder-} \\
	1 kg & \myindex{Gelierzucker}\index{Zucker>Gelier-} \\
	1 & unbehandelte \myindex{Zitrone} (Saft und Schale) \\
	3 Likörgläschen & ,,\myindex{Schwarzer Kater}``
                          (Likör)\index{Likör>Schwarzer Kater} \\
      \end{zutaten}

      \garzeit{4}

      \begin{zubereitung}
	Holunder waschen, abtropfen und ,,entbeeren``. Mit Zitronensaft,
	Zitronenschale und Zucker zum Kochen bringen. 4~Minuten sprudelnd
	kochen lassen, Likör unterrühren und in 6~Gläser abfüllen. \\
      \end{zubereitung}

    \mynewsection{Holunder-Konfitüre mit Äpfeln}\index{Rezept>Konfitüre>Holunder-}

      \begin{zutaten}
	1600 g & reife \myindex{Holunderbeeren}\index{Beeren>Holunder-} \\
	1600 g & \myindex{Gelierzucker}\index{Zucker>Gelier-} \\
	1 & unbehandelte \myindex{Zitrone} (Saft und Schale) \\
	400 g & säuerliche Äpfel\index{Aepfel=Äpfel} \\
      \end{zutaten}

      \garzeit{4}

      \begin{zubereitung}
	Holunder waschen, abtropfen und ,,entbeeren``. Äpfel schälen, klein
	schneiden. Alles mit Zitronensaft, Zitronenschale und Zucker zum Kochen
	bringen. 4~Minuten sprudelnd kochen lassen, in Gläser abfüllen. \\
      \end{zubereitung}

    \mynewsection{Holunder-Konfitüre mit Birnen}\index{Rezept>Konfitüre>Holunder-}

      \begin{zutaten}
	2600 g & reife \myindex{Holunderbeeren}\index{Beeren>Holunder-} \\
	2600 g & \myindex{Gelierzucker}\index{Zucker>Gelier-} \\
	1 & unbehandelte \myindex{Zitrone} (Saft und Schale) \\
	400 g & \myindex{Birne}n \\
      \end{zutaten}

      \garzeit{4}

      \begin{zubereitung}
	Holunder waschen, abtropfen und ,,entbeeren``. Birnen schälen, klein
	schneiden. Alles mit Zitronensaft, Zitronenschale und Zucker zum Kochen
	bringen. 4~Minuten sprudelnd kochen lassen, in Gläser abfüllen. \\
      \end{zubereitung}

    \mynewsection{Holunder-Konfitüre mit Apfel / Birne / Brombeersaft}\index{Rezept>Konfitüre>Holunder-}

      \begin{zutaten}
	3750 g & reife \myindex{Holunderbeeren}\index{Beeren>Holunder-} \\
	3750 g & \myindex{Gelierzucker}\index{Zucker>Gelier-} \\
	\brev{} l & \myindex{Apfelsaft}\index{Saft>Apfel-} \\
	\brev{} l & \myindex{Birnensaft}\index{Saft>Birnen-} \\
	\brev{} l & \myindex{Brombeersaft}\index{Saft>Brombeer-} \\
      \end{zutaten}

      \garzeit{4}

      \begin{zubereitung}
	Holunder waschen, abtropfen und ,,entbeeren``. Säfte mit Zucker zum
	Kochen bringen. 4~Minuten sprudelnd kochen lassen, in Gläser abfüllen.
	\\
      \end{zubereitung}

    \mynewsection{Holunderblütengelee}

      \begin{zutaten}
	10--15 & \myindex{Holunderblüten} \\
	1 l & \myindex{Apfelsaft}\index{Saft>Apfel-} \\
	& \myindex{Gelierzucker}\index{Zucker>Gelier-} \\
      \end{zutaten}

      \begin{zubereitung}
	Holunderblüten in Apfelsaft 24h einlegen, abgießen und Gelee
	mit Gelierzucker kochen. \\
      \end{zubereitung}

    \mynewsection{Rhabarber-Bananenkonfitüre}%
      \index{Rezept>Konfitüre>Rhabarber-Bananen-}

      \begin{zutaten}
	700 g & \myindex{Rhabarber} in kleinen Stücken \\
	250 g & \myindex{Banane} ohne Schale zerdrückt \\
	1 & unbehandelte \myindex{Zitrone} (Saft und Schale) \\
	1000 g & \myindex{Gelierzucker}\index{Zucker>Gelier-} \\
      \end{zutaten}

      \garzeit{4}

      \begin{zubereitung}
	Rhabarber, Banane, Zitrone, Zucker mischen. Alles zum Kochen bringen.
	4~Minuten sprudelnd kochen lassen, in Gläser abfüllen. \\
	Variante: Statt Rhabarber Stachelbeeren oder Aprikosen nehmen. \\
      \end{zubereitung}

    \mynewsection{Aprikosenkonfitüre}

      \begin{zutaten}
	2 kg & \myindex{Aprikosen} \\
	1\breh{} kg & \myindex{Zucker} \\
	1 & \myindex{Zitrone} \\
      \end{zutaten}

      \begin{zubereitung}
	Aprikosen gelieren gut, man kann sich den Gelierzucker sparen. Die
	entsteinten Früchte mit der Hälfte des Zuckers mischen --- wir rechnen
	bei Aprikosen insgesamt auf 1~kg entsteintes Obst nur 750~g Zucker.
	Wenn sich Saft gebildet hat, aufkochen, erst dann den restlichen Zucker
	hinzufügen und etwa 10~Minuten köcheln, bis die richtige Konsistenz
	erreicht ist. Dabei immer wieder rühren, am besten mit einem
	hitzebeständigen Schaber, mit dem man dabei den Rand des Topfs schön
	sauber wischen kann. \\
	Nach der halben Zeit Zitronensaft hinzufügen (auf 2~kg Früchte eine
	ganze Zitrone). Die Masse nach Belieben mit dem Pürierstab etwas
	zerkleinern, aber nicht glatt zermusen. \\
	Sobald die Gelierprobe die richtige Konsistenz anzeigt, (siehe ,,Tip``)
	in Gläser füllen. Am besten sind Schraubgläser geeignet. Gut
	verschließen und für kurze Zeit auf den Kopf stellen, dann ist
	garantiert der Inhalt und der Luftraum darüber steril.\\
	Tip: Die Gelierprobe: Einen Kleks auf ein Metalltellerchen geben, das
	eine Zeitlang im Kühlschrank gekühlt wurde. Er sollte nach einer
	Minute, sobald die Konfitüre abgekühlt ist, angenehm fest sein. \\
	Aprikosensoße --- oder ,,Marillenröster``, wie man in Österreich und
	Südtirol sagt --- ist noch schneller gemacht: nur 250~g Zucker auf
	1~kg Obst rechnen. Ansonsten wie bei der Konfitüre vorgehen und so
	lange kochen lassen, bis alles weich ist. Zum Aufbewahren entweder in
	Schraubgläser füllen und sterilisieren oder einfrieren. \\
      \end{zubereitung}

    \mynewsection{Rhabarberkonfitüre}

      \begin{zutaten}
	1 kg & \myindex{Rhabarber} \\
	300 g & getrocknete, ungeschwefelte \myindex{Aprikosen} \\
	1\breh{} kg & \myindex{Gelierzucker}\index{Zucker>Gelier-} 1:1 \\
      \end{zutaten}

      \bemerkung{Vorbereitungszeit: \textbf{8 Stunden}}

      \begin{zubereitung}
	Aprikosen fein würfeln und 4~Stunden in \breh{}~l warmem Wasser
	einweichen. Dann 10~Minuten in der Einweichflüssigkeit auf kleiner
	Flamme köcheln lassen. Inzwischen den Rhabarber putzen, waschen und
	in 1~cm dicke Scheiben schneiden und zu den Aprikosen geben. Die
	gesamte Obstmenge abwiegen und die gleiche Menge Gelierzucker
	dazugeben. 3--4~Stunden durchziehen lassen. In einem großen Topf
	auf mittlerer Flamme zum Kochen bringen und 4~Minuten sprudelnd
	kochen lassen. Gläser nochmal heiß spülen (Spülmaschine). Danach sofort
	in vorbereitete Twist-Off-Gläser füllen und verschließen. \\
	1,276 kg Rhabarber \'a 1,75 Euro gekauft, geputzt 900 g.
	Aprikosen: 300 g, zusammen 1,2 kg. \\
	Gelierzucker 1:1 \'a 0,99 Euro/kg 1,2 kg dazugegeben. \\
	Reicht für 8 Gläser. \\
      \end{zubereitung}

  \mynewsection{Pflaumenkonfitüre}

    \begin{zutaten}
      1 großer Eimer & \myindex{Pflaumen} (ergibt 23 Gläser ca.) \\
      & \myindex{Gelierzucker}\index{Zucker>Gelier-} 1:1 (besser 2:1) \\
      & \myindex{Zitrone}nschale \\
      & \myindex{Zitrone}nsaft\index{Saft>Zitronen-} \\
      & \myindex{Zimt} \\
    \end{zutaten}

    \begin{zubereitung}
      Pflaumen waschen, entsteinen und wiegen. Danach den Gelierzucker
      ausrechnen (1 kg Pflaumen = 1 kg Gelierzucker 1:1).
      In einem großen Topf sollte man maximal 1 kg Pflaumen verarbeiten
      (kocht sonst über und spritzt weit). \\
      Dazu 1 kg Gelierzucker 1:1 oder 500 g Gelierzucker 2:1, die
      Zitronenschale, den Zitronensaft und Zimt geben. 4 Minuten sprudelnd
      kochen lassen. Ergibt etwa 5 Gläser. \\
    \end{zubereitung}

  \mynewsection{Ananas-Minze-Gelee}

    % aus Marmeladen & Eingemachtes ISBN 978-1-4075-1274-7

    \begin{zutaten}
      1 & große reife \myindex{Ananas}, etwa 800 g ohne Schale \\
      2 (500 g) & \myindex{Kochäpfel} (Boskop oder eine andere pektinreiche
                  Sorte) \\
      1 l & \myindex{Wasser} \\
      einige Zweige & frische \myindex{Minze} \\
      ca. 700 g & \myindex{Zucker} (je nach Saftmenge) \\
      2 Eßlöffel & gehackte frische \myindex{Minze} \\
      & \myindex{grüne Lebensmittelfarbe} (nach Wunsch) \\
    \end{zutaten}

    \bemerkung{ergibt ca. 1 kg}

    \begin{zubereitung}
      Die Ananas schälen und längs vierteln. Das Fruchtfleisch inklusive des
      Mittelteils in kleine Stücke schneiden und in einen großen Topf geben. Die
      Äpfel in Stücke schneiden (mit Schale und Kerngehäuse!) und mit Wasser und
      Minzezweigen zur Ananas geben. Aufkochen, dann die Hitze reduzieren und
      1~Stunde köcheln, bzw. bis die Fruchtstücke sehr weich sind. Etwas
      abkühlen lassen, dann durch ein Seihtuch entsaften. \\
      Wenn der Saft vollständig durchgetropft ist, die Menge abmessen und zurück
      in den ausgespülten Topf gießen. Auf 500 ml Flüssigkeit 450 g Zucker geben.
      Bei geringer Hitze ständig rühren, bis der Zucker sich vollständig
      aufgelöst hat. Aufkochen und 10--15~Minuten sprudelnd kochen, bzw. bis die
      Masse eindickt. \\
      Vom Herd nehmen und mindestens 5~Minuten abkühlen lassen. Bei Bedarf
      Schaum abschöpfen, dann die gehackte Minze und die Lebensmittelfarbe
      einrühren. In warme, sterilisierte Gläser füllen und mit Pergamentpapier
      abdecken. Nach dem Abkühlen mit Zellophan oder Schraubdeckeln verschließen,
      beschriften und kühl aufbewahren. \\
      Tip: Die Minze können Sie durch einige Zweige frischen Rosnarin oder
      Basilikum ersetzen. Paßt perfekt zu Fleischgerichten. \\
    \end{zubereitung}

  \mynewsection{Erdbeer-Rhabarber-Gelee}

    % aus Marmeladen & Eingemachtes ISBN 978-1-4075-1274-7

    \begin{zutaten}
      700 g & frischer \myindex{Rhabarber}, geputzt, gewaschen und in kurze \\
      & Stücke geschnitten \\
      1 l & \myindex{Wasser} \\
      1 kg & \myindex{Erdbeeren}\index{Beeren>Erd-}, geputzt und gewaschen \\
      5 cm & Stück frische \myindex{Ingwer}wurzel, geschält und gewürfelt \\
      ca. 1 kg & \myindex{Zucker} (je nach Saftmenge) \\
    \end{zutaten}

    \bemerkung{ergibt ca. 2 kg}

    \begin{zubereitung}
      Den Rhabarber mit Wasser, Erdbeeren und Ingwer in einem großen Topf
      1~Stunde bei geringer Hitze köcheln, bzw. bis die Früchte sehr weich und
      breiig sind. Durch ein Seihtuch entsaften. \\
      Wenn der Saft vollständig durchgetropft ist, die Menge abmessen und zurück
      in den ausgespülten Topf gießen. Auf 500~ml Flüssigkeit 450~g Zucker
      geben. Bei geringer Hitze ständig rühren, bis der Zucker sich vollständig
      aufgelöst hat. Aufkochen und 10--15~Minuten sprudelnd kochen, bzw. bis die
      Masse eindickt. \\
      Leicht abkühlen lassen und bei Bedarf Schaum abschöpfen, dann in warme,
      sterilisierte Gläser füllen und mit Pergamentpapier abdecken. Nach dem
      Abkühlen mit Zellophan oder Schraubdeckeln verschließen, beschriften und
      kühl aufbewahren. \\
      Tip: Wenn die Fruchtmasse frisch abgefüllt wurde, vermeiden Sie das
      Bewegen oder gar Schütteln der Gläser, denn das kann den Geliervorgang
      stören. \\
    \end{zubereitung}

  \mynewsection{Rotweingelee}

    % aus Marmeladen & Eingemachtes ISBN 978-1-4075-1274-7

    \begin{zutaten}
      500 g & \myindex{Kochäpfel}\index{Aepfel=Äpfel>Koch-} (Boskop oder eine
              andere pektinreiche Sorte), grob in Stücke geschnitten \\
      500 ml & \myindex{Wasser} \\
      700 ml & \myindex{Rotwein} \\
      ca. 700 g & \myindex{Zucker} (je nach Saftmenge) \\
    \end{zutaten}

    \bemerkung{ergibt ca. 700 g}

    \begin{zubereitung}
      Die Äpfel mit Wasser und Wein in einen großen Topf geben. Aufkochen, dann
      die Hitze reduzieren und 30~Minuten köcheln, bis die Äpfel sehr weich
      sind. Durch ein Seihtuch entsaften. \\
      Wenn der Saft vollständig durchgetropft ist, die Menge abmessen und
      zurück in den ausgespülten Topf gießen. Auf 500~ml Flüssigkeit 450~g
      Zucker geben. Bei geringer Hitze ständig rühren, bis der Zucker sich
      vollständig aufgelöst hat. Aufkochen und 15~Minuten sprudelnd kochen,
      bzw. bis die Masse eindickt. \\
      Etwas abkühlen lassen, dann bei Bedarf den Schaum abschöpfen. In warme,
      sterilisierte Gläser füllen und mit Pergamentpapier abdecken. Nach dem
      Abkühlen mit Zellophan oder Schraubdeckeln verschließen, beschriften und
      kühl aufbewahren. \\
      Tip: Weingelee kann man auch mit trockenem Weiß- oder Ros\'ewein
      zubereiten. \\
    \end{zubereitung}

  \mynewsection{Kiwi-Orangen-Marmelade}

    % aus Marmeladen & Eingemachtes ISBN 978-1-4075-1274-7

    \begin{zutaten}
      1\breh{} kg & ungespritzte \myindex{Orange}n aus biologischem Anbau,
              gründlich abgebürstet \\
      5 cm & Stück frische \myindex{Ingwer}wurzel, geschält und gehackt \\
      6 & \myindex{Kiwi}s, geschält und fein gewürfelt \\
      1,4 l & \myindex{Wasser} \\
      1,8 kg & \myindex{Zucker} \\
      1 Teelöffel & \myindex{Butter} \\
    \end{zutaten}

    \bemerkung{ergibt ca. 1 kg}

    \begin{zubereitung}
      Die Orangen sehr dünn schälen und die Schale zur Seite stellen. Die
      Orangen halbieren und auspressen. Alle Kerne mit dem Ingwer in ein
      Mullsäckchen füllen. Die Orangenschale in feine Streifen schneiden und in
      einen großen Topf geben. Kiwis, Wasser, Orangensaft und die Kerne
      zufügen. \\
      Aufkochen, dann die Hitze reduzieren und 1~Stunde köcheln, bzw. bis die
      Schale und Früchte sehr weich sind. Das Mullsäckchen herausnehmen, den
      Inhalt wegwerfen. \\
      Den Zucker zugeben und bei geringer Hitze ständig rühren, bis er sich
      vollständig aufgelöst hat. Die Butter einrühren. Aufkochen und 15~Minuten
      sprudelnd kochen, bzw. bis die Masse eindickt. \\
      Leicht abkühlen lassen, dann in warme, sterilisierte Gläser füllen und
      mit Pergamentpapier abdecken. Nach dem Abkühlen mit Zellophan oder
      Schraubdeckeln verschließen, beschriften und kühl aufbewahren. \\
      Tip: Freunde von Ingwer können nach dem Eindicken noch etwas klein
      geschnittenen eingelegten Ingwer unter die Marmelade rühren. \\
    \end{zubereitung}

  \mynewsection{Apple Butter (Apfelgelee)}

    % aus Marmeladen & Eingemachtes ISBN 978-1-4075-1274-7

    \begin{zutaten}
      1\breh{} kg & \myindex{Kochäpfel}\index{Aepfel=Äpfel>Koch-} (Boskop oder
                    eine andere pektinreiche Sorte) \\
      1 l & lieblicher \myindex{Cidre} oder
                       \myindex{Apfelsaft}\index{Saft>Apfel-} \\
      125 ml & \myindex{Wasser} \\
      125 ml & \myindex{Zitrone}nsaft\index{Saft>Zitronen-} \\
      2 Eßlöffel & abgeriebene \myindex{Zitrone}nschale \\
      2 & \myindex{Zimt}stangen, leicht zersplittert \\
      ca. 500 g & \myindex{Zucker} (je nach Musmenge) \\
    \end{zutaten}

    \bemerkung{ergibt ca. 1 kg}

    \begin{zubereitung}
      Die Äpfel in kleine Stücke schneiden (inklusive Schale und Kerngehäuse).
      Druckstellen herausschneiden. In einen großen Topf geben, dann Cidre
      (oder Apfelsaft), Wasser, Zitronensaft, Zitronenschale und Zimt
      zufügen. Aufkochen, dann die Hitze reduzieren und 30~Minuten köcheln,
      bis die Äpfel zerkocht sind. Zwischendurch umrühren. Die Zimtstangen
      herausnehmen und wegwerfen. Die Apfelmischung passieren, dann abmessen
      und wieder in den ausgespülten Topf füllen. Auf 500~ml Mus 300~g Zucker
      geben. Bei geringer Hitze ständig rühren, bis der Zucker sich vollständig
      aufgelöst hat. Weitere 15--20~Minuten kochen, bzw. bis die Masse cremig
      ist. \\
      In warme, sterilisierte Gläser füllen und mit Pergamentpapier abdecken.
      Nach dem Abkühlen mit Zellophan oder Schraubdeckeln verschließen,
      beschriften und kühl aufbewahren. \\
      Tip: An einem kühlen dunklen Ort bis zu 3~Monate haltbar. Nach dem Öffnen
      im Kühlschrank lagern und innerhalb von 10~Tagen verzehren. Statt
      Kochäpfel können Holzäpfel oder Fallobst verwendet werden. Holzäpfel
      müssen jedoch 1~Stunde köcheln. Zimt kann man durch frische Ingwerwurzel
      ersetzen. \\
    \end{zubereitung}

  \mynewsection{Pflaumenmus}

    % aus Marmeladen & Eingemachtes ISBN 978-1-4075-1274-7

    \begin{zutaten}
      1\breh{} kg & reife \myindex{Pflaumen}, gewaschen, entsteint und
                   halbiert \\
      500 ml & \myindex{Wasser} \\
      2 Eßlöffel & abgeriebene \myindex{Orange}nschale \\
      250 ml & frisch gepreßter \myindex{Orange}nsaft\index{Saft>Orangen-} \\
      1\breh{} Teelöffel & gemahlener \myindex{Zimt} \\
      ca. 1 kg & \myindex{Zucker} (je nach Musmenge) \\
    \end{zutaten}

    \bemerkung{ergibt ca. 700 g}

    \begin{zubereitung}
      Pflaumen und Wasser in einem großen Topf aufkochen. Die Hitze reduzieren
      und 40--50~Minuten köcheln, bis die Pflaumen sehr weich sind und
      zusammenfallen. Etwas abkühlen lassen, dann passieren und die Musmenge
      abmessen. Wieder in den ausgespülten Topf füllen. \\
      Orangenschale, Saft und Zimt zufügen und 10~Minuten leicht kochen. Auf
      500~ml Mus 300~g Zucker geben. Bei geringer Hitze häufig rühren, bis der
      Zucker sich vollständig aufgelöst hat. Aufkochen und weiterköcheln, bis
      die Masse eindickt und cremig ist. \\
      In warme, sterilisierte Gläser füllen und mit Pergamentpapier abdecken.
      Nach dem Abkühlen verschließen und beschriften. Dieses Pflaumenmus ist
      an einem kühlen dunklen Ort bis zu 3~Monate haltbar. Nach dem Öffnen im
      Kühlschrank lagern und innerhalb von 2~Wochen verzehren. \\
      Tip: Beim ersten Kochen können Sie einige Zweige frischen Rosmarin,
      Thymian oder Salbei zu den Pflaumen geben. Vor dem Passieren entfernen.
      So paßt das Mus noch besser zu kalten und warmen Fleischgerichten. \\
    \end{zubereitung}

  \mynewsection{Omas Pflaumenmus}

    % Rezept aus Kaiserslautern

    \begin{zutaten}
      5 kg & \myindex{Pflaumen} \\
      1 kg & \myindex{Zucker} \\
      100 ml & \myindex{Weinessig} \\
      6 & ganze \myindex{Sternanis} \\
      10 g & \myindex{Zimt} \\
      \breh{} Päckchen & \myindex{Pflaumenmusgewürz} \\
    \end{zutaten}

    \begin{zubereitung}
      Pflaumen waschen, entsteinen. Mit Zucker und Weinessig vermischen und
      etwa 12~Stunden Saft ziehen lassen. \\
      In einen backofengeeigneten Bräter geben und im Ofen ca. 10~Stunden
      bei \grad{150} eindicken lassen. \\
      Eine halbe Stunde vor Garende die Gewürze zugeben. Rühren ist nicht
      notwendig. \\
      Wer die Pflaumenschale nicht mag, kann das Mus am Garende mit dem
      Pürierstab zerkleinern (vorher Sternanis entfernen). Pflaumenmus sofort
      in vorbereitete Gläser abfüllen und verschließen. \\
    \end{zubereitung}

  % \mynewsection{Text}

    % \begin{zutaten}
    % \end{zutaten}

    % \begin{zubereitung}
    % \end{zubereitung}
