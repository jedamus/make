
% created Montag, 10. Dezember 2012 16:21 (C) 2012 by Leander Jedamus
% modifiziert Dienstag, 23. Juni 2015 19:01 von Leander Jedamus
% modifiziert Mittwoch, 11. März 2015 17:16 von Leander Jedamus
% modifiziert Montag, 09. März 2015 14:22 von Leander Jedamus
% modified Donnerstag, 27. Dezember 2012 15:24 by Leander Jedamus
% modified Montag, 10. Dezember 2012 16:30 by Leander Jedamus

  \mynewchapter{Kuchen}

    \mynewsection{Mürbteig}\glossary{Teig>Mürb-}\label{muerbteig}

      \begin{zutaten}
        250 g & \myindex{Mehl} \\
        1 Teelöffel & \myindex{Backpulver} \\
        80 g & \myindex{Zucker} \\
        1 & \myindex{Vanillezucker}\index{Zucker>Vanille-} \\
        1 Prise & \myindex{Salz} \\
        1 & \myindex{Ei} \\
        100 g & \myindex{Margarine} \\
      \end{zutaten}

      \begin{zubereitung}
        Das Mehl wird mit dem Backpulver gesiebt und mit den geschmackgebenden
	Zutaten vermischt. In der Mitte macht man eine Vertiefung und gibt das
	Ei hinein. Am Rand verteilt man die Butterflöckchen. Alle Zutaten
	werden zu einem Teig verknetet. Der Teig wird kalt gestellt. \\
      \end{zubereitung}

    \mynewsection{Hefeteig}\glossary{Teig>Hefe-}\label{hefeteig}

      \begin{zutaten}
        500 g & \myindex{Mehl} \\
        20--30 g & \myindex{Hefe} \\
        etwa \brev{} l & \myindex{Milch} \\
        80--100 g & \myindex{Fett} \\
        80 g & \myindex{Zucker} \\
        \breh{} Teelöffel & \myindex{Salz} \\
        1--2 & \myindex{Ei}er \\
        1 & \myindex{Vanillezucker}\index{Zucker>Vanille-} oder
	    \myindex{Aroma} oder abgeriebene \myindex{Zitrone} \\
      \end{zutaten}

      \begin{zubereitung}
        Das Mehl wird in eine Schüssel gesiebt und mit Salz, Zucker und
	Vanillezucker vermischt. In der Mitte des Mehles macht man eine
	Vertiefung. Die Hefe wird zerbröckelt und in die Vertiefung gegeben.
	Ein wenig Mehl und lauwarme Milch wird dazu gerührt. Das Hefestück läßt
	man gehen, bis es Blasen schlägt. Die übrigen Zutaten werden
	hinzugegeben und der Teig wird gut geschlagen oder geknetet. Man läßt
	ihn nochmals gehen und knetet ihn nochmals durch und verarbeitet ihn
	weiter nach Belieben. \\
      \end{zubereitung}

    \mynewsection{Quarkblätterteig}\glossary{Teig>Quarkblätter-}

      \begin{zutaten}
        250 g & \myindex{Mehl} \\
        \breh{} & \myindex{Backpulver} \\
        1 & \myindex{Vanillezucker}\index{Zucker>Vanille-} \\
        1 Prise & \myindex{Salz} \\
        \breh{} & \myindex{Quark} \\
        150--200 g & \myindex{Margarine} \\
        2--3 Eßlöffel & \myindex{Konfitüre} \\
        1 & \myindex{Eigelb} \\
        & \myindex{Puderzucker}\index{Zucker>Puder-} \\
      \end{zutaten}

      \begin{zubereitung}
        Das Mehl wird mit dem Backpulver auf ein Backbrett gesiebt. In der
	Mitte macht man eine Vertiefung und gibt den durch ein Sieb
	gestrichenen Quark hinein. Die in Stücke geschnittene Margarine
	verteilt man am Rand und verknetet alles zu einem Teig. Man rollt ihn
	aus und teilt ihn mit dem Kuchenrädchen in Vierecke, füllt sie mit
	Konfitüre und formt sie beliebig zu Dreiecken, Hörnchen, Taschen,
	Windmühlen und dergleichen. Man bestreut sie mit Eigelb und bäckt sie
	ab. Nach dem Backen überstäubt man sie mit Puderzucker oder überzieht
	sie mit Zuckerguß. \\
      \end{zubereitung}

    \mynewsection{Quarkölteig}\glossary{Teig>Quarköl-}\label{quarkoelteig}

      \begin{zutaten}
        200 g & abgetropfter \myindex{Quark} \\
	7 Eßlöffel & Öl\index{Oel=Öl} (oder 150 g \myindex{Butter}) \\
	\brev{} Teelöffel & \myindex{Salz} \\
	1 & \myindex{Ei} \\
	100 g & \myindex{Zucker} \\
	1 & \myindex{Zitrone} (Schale abreiben) \\
        400 g & \myindex{Mehl} \\
	1 Päckchen & \myindex{Backpulver} \\
      \end{zutaten}

      \begin{zubereitung}
        In einer Rührschüssel Öl (oder cremige Butter) mit Salz, Ei, Zucker,
	abgeriebener Zitronenschale und Quark verrühren. Mehl und Backpulver
	sieben. Die Hälfte dieser Zutaten unterrühren, zweite Hälfte auf dem
	Backbrett unterkneten. In der Maschine geht es aber auch, nur 2--3~Mal
	die festsitzenden Rückstände bei ausgeschalteter Maschine runterschaben
	und zur Mitte schieben. \\
	Danach für 1--2~Stunden kühlstellen (mit Folie abdecken oder in Folie
	wickeln). Anschließend auswalzen für das Blech, für eine runde
	Kuchenform reicht die halbe Menge. \\
	Belag: Obst, Quark- und Streuselbelag. \\
      \end{zubereitung}

    \mynewsection{Strudelteig}\label{strudelteig}\glossary{Teig>Strudel-} 

      \begin{zutaten}
        \brea{} l & lauwarmes \myindex{Wasser} \\
	\breh{} Teelöffel & \myindex{Salz} \\
	250 g & \myindex{Mehl} \\
	1 & \myindex{Ei} \\
	1 Eßlöffel & Öl\index{Oel=Öl} \\
      \end{zutaten}

      \begin{zubereitung}
        Das Mehl gibt man auf ein großes Nudelbrett. Dann verquirlt man gut
	das Wasser, Öl, Ei und Salz und gießt es nach und nach zum Mehl. Jetzt
	bereitet man einen nicht zu festen Teig, den man so lange abarbeitet,
	bis er Blasen wirft. 1~Stunde mit einer angewärmten Schüssel zugedeckt
	gehen lassen. Nun schneidet man 2--3~Teile, rollt sie auf einem
	bemehlten Tuch messerrückendick aus und zieht dann den Teig mit den
	Fingern ganz dünn aus. \\
	Ein guter Strudelteig fordert viel Zeit, Geduld und Geschick. Der
	tiefgefrorene fertige Strudelteig erspart der eiligen Hausfrau viel
	Arbeit! \\
      \end{zubereitung}

    \mynewsection{Zwetschenkuchen (Pflaumenkuchen)}\glossary{Kuchen>Zwetschen-}%
      \glossary{Kuchen>Pflaumen-}

      \begin{zutaten}
        1 & \myindex{Hefeteig} wie auf Seite \pageref{hefeteig} \\
        1 \brev{} kg & \myindex{Pflaumen} \\
        & \myindex{Streusel} \\
        200 g & \myindex{Mehl} \\
        100 g & \myindex{Zucker} \\
        1 & \myindex{Vanillezucker}\index{Zucker>Vanille-} \\
        100 g & \myindex{Butter} \\
      \end{zutaten}

      \garzeit{30--40}

      \begin{zubereitung}
        Die Pflaumen werden gewaschen, halbiert, entsteint und auf den
	ausgerollten Teig gegeben. Beliebig gibt man Streusel darauf und bäckt
	den Kuchen ab (\grad{220} mittlere Schiene 30--40~Minuten). \\
      \end{zubereitung}

    \mynewsection{Käsekuchen}\glossary{Kuchen>Käse-}

      \begin{zutaten}
      \end{zutaten}
      \begin{zutat}{Backpulverteig}
        80 g & \myindex{Butter}, zimmerwarm, in Flöckchen \\
        80 g & \myindex{Zucker} \\
        1 & \myindex{Ei} \\
        1 Prise & \myindex{Salz} \\
        etwas & abgeriebene \myindex{Zitrone}nschale oder \breh{} Päckchen \myindex{Vanillezucker}\index{Zucker>Vanille-} \\
        200 g & \myindex{Mehl} \\
        1 \breh{} Teelöffel & \myindex{Backpulver} \\
        3--4 Eßlöffel & kalte \myindex{Milch} \\
        2 Eßlöffel & \myindex{Rum} \\
      \end{zutat}
      \begin{zutat}{Belag}
        2x 500 g & \myindex{Schichtkäse} \\
        120 g & \myindex{Zucker} \\
        4 & \myindex{Ei}er \\
        \breh{} & abgeriebene \myindex{Zitrone}nschale \\
        4 Eßlöffel & \myindex{Zitrone}nsaft \\
        1 Päckchen & \myindex{Vanillezucker}\index{Zucker>Vanille-} \\
        1 Teelöffel & \myindex{Stärkemehl} \\
        1 Teelöffel & \myindex{Backpulver} \\
      \end{zutat}

      \garzeit{15--20 + 40--50}

      \begin{zubereitung}
        Butter schaumig rühren, Zucker und Ei zugeben. Schaumig rühren. Salz,
	Zitrone oder Vanillezucker untermischen. Mehl und löffelweise Milch und
	Alkohol unter die Schaummasse rühren. Teig muß sich gerade noch mit dem
	Löffel bearbeiten lassen. Springform fetten und ausbröseln. Teig mit
	4~cm Rand mit Backpapier und Hülsenfrüchten bedecken. Blind-Backen bei
	\grad{190} etwa 15--20~Minuten. Schichtkäse durch ein Sieb streichen,
	Zucker, Eier, Zitronenschale, Vanillezucker etc. gut verrühren
	(Mixstab) und auf abgebackenen Kuchenboden geben, glätten. Bei
	\grad{190} etwa 40--50~Minuten backen. Nach dem Backen 10--15~Minuten
	in der Form abkühlen lassen. \\
      \end{zubereitung}

    \mynewsection{Klassischer Käsekuchen}\glossary{Kuchen>Käse-}

      \begin{einleitung}
        Dieser Kuchen fällt \underline{nicht} zusammen!\\
	\begin{enumerate}
	  \item Kein Teigrand.
	  \item Eischnee nur cremig schlagen, nicht zu steif.
	  \item Eischnee nur mit Zucker schlagen.
	  \item Ofen auf \grad{200} Ober- / Unterhitze vorheizen.
	  \item Kuchen auf mittlerer Schiene 15--20~Minuten backen (kocht,
	        geht stark in die Höhe).
	  \item Kuchen rausnehmen, oben Kante einschneiden.
	  \item Ofen auf \grad{160} stellen.
	  \item Nach 10~Minuten Kuchen wieder reinschieben.
	  \item Weitere 50~Minuten backen.
	  \item In der Form abkühlen lassen.
	  \item Rand abnehmen, mit Puderzucker bestreuen.
	\end{enumerate}
      \end{einleitung}

      \begin{zutaten}
      \end{zutaten}
      \begin{zutat}{Mürbteig}
        200 g & \myindex{Butter} \\
        100 g & \myindex{Zucker} \\
        1 Prise & \myindex{Salz} \\
        etwas & \myindex{Vanillearoma} \\
        1 & \myindex{Ei}gelb (mittelgroßes Ei) \\
        1 Messerspitze & \myindex{Backpulver} \\
        300 g & \myindex{Mehl} \\
      \end{zutat}
      \begin{zutat}{Belag}
        665 g & \myindex{Schichtkäse}, klumpenfrei gerührt \\
        1 Prise & \myindex{Salz} \\
        etwas & \myindex{Vanillearoma} \\
        & Abrieb von \breh{} unbehandelter \myindex{Zitrone} \\
        330 g & \myindex{saure Sahne}\index{Sahne>sauer} \\
        4--6 & \myindex{Ei}gelbe (mittelgroße Eier) \\
        330 ml & \myindex{Milch} \\
        52 g & \myindex{Vanillepudding}pulver (ersatzweise
	       \myindex{Speisestärke}) \\
        8 & \myindex{Ei}weiß \\
        210 g & \myindex{Zucker} \\
      \end{zutat}

      \begin{zubereitung}
        Mürbteig: Alles zügig verkneten, 1~Stunde in den Kühlschrank. Die Form
	sollte 26--28~cm~\durchmesser{} haben. \\
	Flach ausrollen und auf gefetteten Boden einpassen. \underline{Keinen}
	Rand hochziehen. \\
	\\
	Belag: Die letzten beiden Zutaten (8~Eiweiße und den Zucker) werden
	in einer Extra-Schüssel cremig geschlagen. \\
	Zutaten wie angegeben (\underline{Reihenfolge} soll man einhalten!)
	rühren. Den Eischnee nicht zu stark schlagen, nur cremig, daher der
	Zucker. Dann den Eischnee mit dem Schneebesen von außen nach innen
	einarbeiten in die Schichtkäsemasse. Masse auf der Teigplatte in die
	Form füllen. \\
	Backofen auf \grad{200}~Ober- und Unterhitze vorheizen. Auf mittlerer
	Schiene 15--20~Minuten backen, rausnehmen. Ofen auf \grad{160} stellen.
	Mit einem kleinem Messer ganz oben am Rand entlang ca. 2--3~cm~tief
	ringsrum einschneiden.
	Nach 10~Minuten des Abkühlens weiterbacken für 50~Minuten. Abkühlen
	lassen. Dann erst aus der Form nehmen. Mit Puderzucker bestäuben. \\
	Man sollte vermeiden, zu schnell und zu heiß zu backen! \\
      \end{zubereitung}

    \mynewsection{Käsekuchen ohne Boden}\glossary{Kuchen>Käse- ohne Boden}
      
      \begin{zutaten}
        225 g & \myindex{Margarine} \\
	200 g & \myindex{Zucker} \\
	500 g & \myindex{Magerquark}\index{Quark>Mager-} \\
	500 g & \myindex{Sahnequark}\index{Quark>Sahne-} (Fettstufe 10--40\%,
	        nach Geschmack)\\
	2 Päckchen & \myindex{Vanillezucker}\index{Zucker>Vanille-} \\
	3 Päckchen & \myindex{Vanillepudding}\index{Pudding>Vanille-} \\
	6 & \myindex{Ei}er \\
      \end{zutaten}

      \begin{zubereitung}
        Alle Zutaten verrühren und in eine gefettete Springform geben
	(28~cm~\durchmesser{}). Backofen auf \grad{170} vorheizen und ca.
	60~Minuten bei Ober- und Unterhitze backen. \\
      \end{zubereitung}

    \mynewsection{Klassischer Käsekuchen (norddeutsch)}

      \begin{zutaten}
      \end{zutaten}

      \begin{zutat}{Teig}
        150 g & \myindex{Mehl} \\
	1 Teelöffel & \myindex{Backpulver} \\
	75 g & \myindex{Butter} \\
	75 g & \myindex{Fett} \\
	\breh{} Teelöffel & \myindex{Zitrone}nschale \\
	2 & \myindex{Ei}gelb \\
      \end{zutat}

      \begin{zutat}{Belag}
        \breh{} l & \myindex{Milch} \\
	200 g & \myindex{Zucker} \\
	2 Päckchen & \myindex{Vanillepudding}pulver \\
	\breh{} Fläschchen & \myindex{Backöl} ,,Zitrone`` \\
	750 g & \myindex{Quark} 40 \% \\
	3 & \myindex{Ei}weiß \\
      \end{zutat}

      \begin{zutat}{Zum Bestreichen}
        1 & \myindex{Ei}gelb \\
	1 Eßlöffel & \myindex{Milch} \\
      \end{zutat}

      \begin{zubereitung}
        Teig als Mürbteig verarbeiten und in eine Springform
	(\durchmesser{}~28~cm) drücken. Backen bei \grad{180--200}~Ober- und
	Unterhitze ca. 10--12~Minuten vorbacken. \\
	\brzd{} der Milch in einen Topf geben mit \brzd{} des Zuckers und dem
	Backöl aufkochen. In der restlichen Milch das Puddingpulver anrühren,
	dann in die kochende Milch geben und aufkochen lassen. Dauernd rühren.
	Den Quark dazugeben und unter Rühren (ständig!) nochmals aufkochen. Es
	dürfen \underline{keine} Klumpen im Belag sein. \\
	Eiweiß mit restlichem Zucker zu einem steifen Eischnee schlagen und
	unter die Quarkmasse heben. Dann Schüsselinhalt auf vorgebackenem
	Mürbteig geben, glattstreichen. Das Eigelb mit der Milch verquirlen und
	mit einem Pinsel den Kuchen bestreichen (damit er schön gelb wird). \\
	Bei \grad{~150--160} Ober- und Unterhitze ca. 55--60~Minuten backen.
	Herd ausschalten und Kuchen noch 15~Minuten bei geschlossener Tür
	drinlassen, so fällt er nicht zusammen. \\
	Man soll die Tür beim Backen zulassen. \\
	Den Kuchen mit einem Tortenteiler markieren bzw. stanzen und servieren.
	\\
	Manche geben Rosinen, Kirschen und andere Früchte in den Teig. \\

      \end{zubereitung}

    \mynewsection{Apfelkuchen}\glossary{Kuchen>Apfel-}

      \begin{zutaten}
        & \myindex{Mürbteig} (siehe Seite \pageref{muerbteig})
	  aus 1 Pfund Mehl machen \\
        2--3 kg & saure geschälte Äpfel\index{Aepfel=Äpfel} in Spalten geschnitten \\
        \brev{}--\breh{} Pfund & blaue \myindex{Rosinen} oder
	                         \myindex{Korinthen} (heiß abgespült) \\
        \brev{} l & \myindex{saure Sahne}\index{Sahne>sauer} \\
        & \myindex{Zucker} \\
        1--2 & \myindex{Eigelb}e \\
      \end{zutaten}

      \garzeit{45}

      \begin{zubereitung}
        Die Äpfel mit Rosinen oder Korinthen mischen, vorsichtig dünsten mit
	Zucker nach Geschmack. Den Teig auf 1--2~Bleche rollen, dicker Rand
	ringsum. In den Ofen bis der Teig ca. 30~Minuten gebacken hat. Dann
	dicht mit den Äpfeln belegen und übergießen mit der Mischung aus saurer
	Sahne, Zucker und Eigelben. 10--15~Minuten weiterbacken. Jetzt sollte
	der Guß goldgelb sein. \\
      \end{zubereitung}

    \mynewsection{Apfelkuchen mit Quarkölteig}%
      \glossary{Kuchen>Apfel- mit Quarkölteig}

      \begin{zutaten}
        & \myindex{Quarkölteig} (siehe Seite \pageref{quarkoelteig}) \\
	ca. 2 kg & Äpfel\index{Aepfel=Äpfel} geschält und zerteilt \\
	2 Becher & \myindex{Schmand} (\'a 200 g) \\
	1 Handvoll & \myindex{Rosinen} oder \myindex{Sultaninen} gewaschen \\
	1--2 Eßlöffel & \myindex{Zucker} \\
	\breh{} Teelöffel & \myindex{Zimt} \\
      \end{zutaten}

      \garzeit{50}

      \begin{zubereitung}
        Äpfel auf den Teig schütten. Schmand mit restlichen Zutaten mischen und
	über die Äpfel geben (Man könnte auch die Äpfel mit Schmand und
	restlichen Zutaten mischen). \\
	Backen 50~Minuten bei \grad{180} auf der 2. Schiene von unten. \\
	Schmeckt am besten am gleichen Tag wie gebacken. \\
	Falls Äpfel vom Belag übrig sind: Apfelmus machen. Dazu wenig Zucker
	und Wasser dazu, aufkochen lassen und auf der Kochplatte garziehen
	lassen. \\
      \end{zubereitung}

    \mynewsection{Versunkener Apfelkuchen}%
      \glossary{Kuchen>Apfel- versunken}

      \begin{zutaten}
        125 g & \myindex{Margarine} \\
	125 g & \myindex{Zucker} \\
	3 & \myindex{Ei}er \\
	5--6 & Äpfel\index{Aepfel=Äpfel} \\
	250 g & \myindex{Mehl} \\
	\breh{} Päckchen & \myindex{Backpulver} \\
      \end{zutaten}

      \begin{zubereitung}
        Backofen auf \grad{200} vorheizen. \\
	Butter und Zucker schaumig rühren. Eier zugeben und verrühren. Mehl
	und Backpulver sieben und über die Masse geben. Teig in eine gefettete
	Springform (26~cm~\durchmesser{}) geben. Äpfel schälen, vierteln und
	einritzen, dann in den Teig drücken. \\
	Bei \grad{200} ca. 25--40~Minuten backen, weitere 5--10~Minuten bei
	\grad{70}. \\
      \end{zubereitung}

    \mynewsection{Dosenaprikosenkuchen}\glossary{Kuchen>Aprikosen-}

      \begin{zutaten}
      \end{zutaten}
      \begin{zutat}{Teig}
        1 & \myindex{Ei} \\
        100 g & \myindex{Margarine} \\
        1 Päckchen & \myindex{Vanillezucker}\index{Zucker>Vanille-} \\
        200 g & \myindex{Mehl} \\
        75 g & \myindex{Zucker} \\
        2 Teelöffel & \myindex{Backpulver} \\
        & \myindex{Salz} \\
      \end{zutat}
      \begin{zutat}{Belag}
        100 g & blättrige \myindex{Mandel}n \\
        100 g & \myindex{Zucker} \\
        1 & \myindex{Ei} \\
        \brea{} l & \myindex{Sahne} oder \myindex{Dosenmilch} \\
        1 große Dose & \myindex{Aprikosen} gut abgetropft \\
      \end{zutat}

      \garzeit{30}

      \begin{zubereitung}
        Teig in Springform geben, Rand hochziehen. Aprikosen verteilen. Aus den
	restlichen Zutaten eine dickflüssige Soße machen und über die Aprikosen
	gießen. Bei \grad{200} ca. 30~Minuten backen. \\
      \end{zubereitung}

    \mynewsection{Aprikosenkuchen mit Zitronenguß}

      \begin{zutaten}
      \end{zutaten}
      \begin{zutat}{Mürbeteig}
        200 g & \myindex{Mehl} \\
	75 g & \myindex{Butter} \\
	50 g & \myindex{Zucker} \\
	1 & \myindex{Eiweiß} \\
	1 Prise & \myindex{Salz} \\
      \end{zutat}
      \begin{zutat}{Belag}
        ca. 1,2 kg & \myindex{Aprikosen} \\
	50 g & \myindex{Haselnüsse}\index{Nuß>Hasel-} \\
	30 g & \myindex{brauner Zucker}\index{Zucker>braun} \\
	& \myindex{Butter} für die Form \\
      \end{zutat}
      \begin{zutat}{Zitronenguß}
        2 & \myindex{Ei}er (sowie das vom Teig übrig gebliebene Eigelb) \\
	4 Eßlöffel & \myindex{Zucker} \\
	2 & Saft von etwa 2 \myindex{Zitrone}n \\
      \end{zutat}

      \begin{zubereitung}
        Aus den Zutaten einen Teig kneten, zu einer Kugel gerollt und in einem
	Klarsichtbeutel \breh{}~Stunde kalt stellen. \\
	Unterdessen die Aprikosen waschen und entsteinen, dabei halbieren, sehr
	große Früchte sogar vierteln. Die Haselnüsse im \grad{200} heißen Ofen
	auf einem Blech etwa 10~Minuten rösten, bis sie duften. Mit dem
	Zucker im Zerhacker zermahlen. \\
	Den Teig auf einem Stück Folie ausrollen (am besten dafür einfach den
	Klarsichtbeutel aufschlitzen), mit Hilfe der Folie in die ausgebutterte
	Form heben und diese damit auskleiden. Dabei den Rand schön hochziehen.
	Die Nuß-Zucker-Mischung hineinschütten und gleichmäßig verteilen. \\
	Die entsteinten Aprikosen, mit Wölbung nach unten dicht an dicht, auf
	dieser Fläche anordnen. Lücken können mit kleingeschnittenen Aprikosen
	gefüllt werden. Im \grad{200} heißen Ofen etwa 15~Minuten vorbacken
	--- dabei die Form möglichst weit unten einschieben (am besten sogar
	direkt auf einem Rost auf den Boden des Backofens stellen) und, wenn
	möglich, zusätzlich Unterhitze einschalten. Der Boden soll jetzt
	deutlich bräunen. \\
	Erst dann die Zutaten für den Guß miteinander verquirlen. Je nach Säure
	der Zitronen mit Zucker abschmecken, über den Kuchen gießen und
	nochmals 10--15~Minuten backen, bis alles gestockt ist, dabei die Hitze
	auf \grad{180} herunterschalten. Ein wenig auskühlen lassen, bevor der
	Kuchen angeschnitten wird, damit sich die Zitronencreme festigen kann.
	\\
	Getränk: Kaffee oder Espresso. Wenn der Kuchen als Dessert serviert
	wird, paßt auch ein Moscato d'Asti oder lieblicher Prosecco.
	Verblüffend gut paßt dazu auch ein junger, frisch-fruchtiger Rotwein.
	Diesen sollte man kellerkalt servieren. \\
      \end{zubereitung}

    \mynewsection{Apfelstrudel}

      \begin{zutaten}
        1 & \myindex{Strudelteig} (siehe Seite \pageref{strudelteig}) \\
	8--10 & \myindex{Kochäpfel}\index{Aepfel=Äpfel>Koch-} \\
	\brev{} l & \myindex{saurer Rahm} \\
	80 g & \myindex{Weinbeeren} \\
	125 g & \myindex{Butter} \\
	100 g & \myindex{Zucker} \\
	\brev{} l & \myindex{Milch} \\
      \end{zutaten}

      \begin{zubereitung}
        Der ausgezogene Strudelteig wird mit zerlassener Butter und dickem
	saurem Rahm bestrichen, darauf kommen die feingeschnittenen, gut mit
	Zucker bestreuten Äpfel und Weinbeeren. Den Strudel zusammenrollen und
	in eine Bratreine geben, in der man ca. 50~g Butter und etwas heiße
	Milch zerlassen hat. Je nach Größe der Bratreine setzt man 2--3~Strudel
	nebeneinander, berstreicht sie noch gut mit Butter und bäckt den
	Strudel im Rohr ungefähr 50~Minuten, bis er eine schöne, hellbraune
	Farbe hat. \\
      \end{zubereitung}

    \mynewsection{Rhabarberkuchen mit Mandelbaiser}

      \begin{zutaten}
        700 g & \myindex{Rhabarber} \\
	250 g & \myindex{Zucker} \\
	250 g & \myindex{Mehl} \\
	125 g & \myindex{Butter} oder \myindex{Margarine} \\
	50 g & gemahlene \myindex{Mandel}n \\
	50 g & \myindex{Mandelblättchen} \\
	4 & \myindex{Ei}er \\
	1 Päckchen & \myindex{Vanillezucker}\index{Zucker>Vanille-} \\
	\breh{} Päckchen & \myindex{Backpulver} \\
	1 Eßlöffel & \myindex{Zitrone}nsaft \\
	eventuell & \myindex{Milch}, falls der Teig zu trocken ist \\
	1 Prise & \myindex{Salz} \\
      \end{zutaten}

      \begin{zubereitung}
        Rhabarber putzen und in Stücke schneiden. \\
	3~Eier trennen. Fett, 100~g Zucker, Vanillezucker, Salz, 1~ganzes Ei
	und 3~Eigelb schaumig rühren. Nach und nach Mehl, Backpulver und
	eventuell Milch zufügen. \\
	Auf ein gefettetes Backblech streichen. Rhabarber drauflegen, etwas
	eindrücken. \\
	Im vorgeheizten Backofen bei \grad{180} mit Ober- und Unterhitze ca.
	20~Minuten backen. \\
	3~Eiweiß steif schlagen, 150~g Zucker einrieseln lassen, Zitronensaft
	und gemahlene Mandeln unterziehen. \\
	Auf den Kuchen streichen, mit den Mandelblättchen bestreuen und weitere
	20~Minuten bei \grad{180} Hitze backen. \\
      \end{zubereitung}

    \mynewsection{Aprikosenkuchen mit frischen Aprikosen}

      \begin{zutaten}
      \end{zutaten}

      \begin{zutat}{Rührteig}
	150 g & \myindex{Butter} \\
	150 g & \myindex{Zucker} \\
	150 g & \myindex{Mehl} \\
	50 g & \myindex{Stärke} \\
	2 Eßlöffel & \myindex{Orangenlikör} oder \myindex{Marillenbrand} \\
	2 Teelöffel & \myindex{Backpulver} \\
	& abgeriebene \myindex{Zitrone}nschale \\
	3 & \myindex{Ei}er \\
	1 Tütchen & \myindex{Vanillezucker}\index{Zucker>Vanille-} \\
	1 Prise & \myindex{Salz} \\
      \end{zutat}

      \begin{zutat}{Belag}
        1 kg & \myindex{Aprikosen} \\
	& \myindex{Puderzucker} zum Bestäuben \\
      \end{zutat}

      \begin{zubereitung}
        Butter bei Zimmertemperatur weich werden lassen, Form
	(24~cm~\durchmesser{}) einfetten und mit Mehl ausstäuben. Aprikosen
	entsteinen, halbieren. \\
	Die weiche Butter mit dem Handrührer cremig schlagen, Zucker einrieseln
	lassen und nacheinander die Eier zufügen. Wenn alles cremig verbunden
	ist, mit Zitronenschale, Orangenlikör, Salz, Vanillezucker würzen.
	Mehl, Stärke, Backpulver sieben und hinzufügen und einarbeiten.
	Teigmasse in Form streichen. \\
	Aprikosenhälften senkrecht (Stielansatz unten) dicht an dicht
	nebeneinander in den Teig stecken. \\
	Bei \grad{200} Ober- + Unterhitze 55--60~Minuten backen. \\
	Holzstäbchenprobe: Senkrecht an der dicksten Teigstelle einstechen.
	Das Holz muß trocken sein und sich warm an der Lippe anfühlen. \\
	5~Minuten in der Form auskühlen lassen, dann den Rand der Form
	abnehmen. Tortenheber verwenden, um Kuchen aus der Form auf die
	Tortenplatte zu transportieren, Kuchen zerbricht sehr leicht. \\
	Vor dem Servieren mit Puderzucker bestäuben. \\
	Dazu paßt Aprikosen-, Orangen- oder Zitronenlikör. \\
      \end{zubereitung}

    \mynewsection{Kirschstrudel mit Schokoladeneis und Chili}

      \begin{zutaten}
      \end{zutaten}
      \begin{zutat}{Füllung}
        600 g & ensteinte \myindex{Sauerkirschen}\index{Kirschen>Sauer-}
	        aus dem Glas \\
        2 cl & \myindex{Kirschwasser} \\
	20 g & \myindex{Semmelbrösel} \\
	100 g & gemahlene \myindex{Mandel}n \\
	1 & \myindex{Ei} \\
	2 cl & \myindex{Mandellikör} \\
	2 cl & \myindex{Orangenlikör} \\
	50 g & \myindex{Marzipan} \\
	50 g & \myindex{Butter} \\
	50 g & \myindex{Puderzucker}\index{Zucker>Puder-} \\
	1 & unbehandelte \myindex{Zitrone} \\
	1 & unbehandelte \myindex{Orange} \\
	\breh{} & \myindex{Vanilleschote} \\
	2 Eßlöffel & \myindex{Zartbitterschokolade}%
	             \index{Schokolade>Zartbitter-} \\
	& \myindex{Zimt} \\
	& \myindex{Salz} \\
      \end{zutat}

      \begin{zutat}{Strudel}
        2 & \myindex{Strudelteig}blätter 20 x 30 cm Seitenlänge \\
	60 g & flüssige \myindex{Butter} \\
      \end{zutat}

      \begin{zutat}{Pralinensoße}
	125 g & \myindex{Zartbitterschokolade}%
	        \index{Schokolade>Zartbitter-} \\
	125 g & \myindex{Haselnußnougat} \\
	250 g & \myindex{Sahne} \\
	2 cl & \myindex{Mandellikör} \\
	2 cl & \myindex{Cognac} \\
	1 & unbehandelte \myindex{Orange} \\
	& \myindex{Chili} \\
      \end{zutat}

      \begin{zubereitung}
        Füllung: Kirschen absieben und mit Kirschwasser beträufeln. Die
	Semmelbrösel mit den Mandeln mischen und auf einem Blech im Backofen
	nach Sicht etwa 10~Minuten hell rösten. Dazwischen mehrmals umrühren,
	damit die Nüsse gleichmäßig bräunen. Anschließend abkühlen lassen.
	Das Ei trennen. Das Marzipan mit dem Eigelb, Mandellikör und
	Orangenlikör cremig rühren. Etwas Schale der Zitrone und der Orange
	abreiben. Die Butter mit dem Puderzucker, je \breh{}~Teelöffel Zitrone
	und Orangenabrieb, Vanillemark, 1~Prise Zimt und Salz schaumig schlagen.
	Das Marzipan mit dem Eiweiß hinzufügen und einige Minuten weiter
	schlagen. Die Kirschen mit der Marzipanmasse, der Mandelmischung und
	gehackter Zartbitterschokolade vermengen. \\
	\\
	Strudel: Backofen auf \grad{200} vorheizen. Muffinblech mit etwas
	flüssiger Butter bestreichen. Den Strudelteig in 15~cm Quadrate
	schneiden und mit der restlichen flüssigen Butter bestreichen.
	Quadrate in Muffinblech legen. 1--2~Eßlöffel der Füllung in die Mitte
	setzen und die Enden übereinanderschlagen. \\
	Im Backofen 20~Minuten hell backen. \\
	\\
	Pralinensoße: Schokolade und Nougat grob würfeln. Sahne erhitzen und
	mit der Schokolade und den Nougatwürfeln zu einer glatten Soße rühren.
	Ca. 1~Messerspitze Orangenschale abreiben. \\
	Abschmecken mit Mandellikör, Cognac, Orangenabrieb und etwas Chili. \\
	\\
	Servieren: Pralinensoße auf einen Teller geben und eventuell noch etwas
	Vanillesoße dazugeben und ein Muster daraus machen. \\
	Strudelteigtaschen auf die Mitte des Tellers setzen. Mit Puderzucker
	bestreuen. \\
      \end{zubereitung}

    \mynewsection{Taunus-Tiramisu}\glossary{Tiramisu>Taunus-}

      \begin{zutaten}
	1 & \myindex{Ei} \\
	60 g & \myindex{Zucker} \\
	2 Päckchen & \myindex{Vanillezucker}\index{Zucker>Vanille-} \\
	1 Päckchen & \myindex{Sahnesteif} \\
	60 g & \myindex{Mehl} \\
	1 Teelöffel & \myindex{Backpulver} \\
	1 Teelöffel & \myindex{Kakao}pulver \\
	30 ml & \myindex{Olivenöl}\index{Oel=Öl>Oliven-} \\
	40 ml & \myindex{Orangenlimonade} \\
	2 Eßlöffel & \myindex{Milch} \\
	2 Eßlöffel & \myindex{Orangensaft} \\
	300 g & \myindex{Pflaumenmus} \\
	125 ml & \myindex{Sahne} \\
	150 g & \myindex{Schmand} \\
	50 g & \myindex{Margarine} \\
      \end{zutaten}

      \personen{2}

      \begin{zubereitung}
        Backofen auf \grad{200} vorheizen. \\
	Ei, Zucker und 1~Päckchen Vanillezucker schaumig schlagen, dann
	Olivenöl und Orangenlimonade unterrühren. Mehl mit Backpulver mischen,
	sieben und protionsweise unter die Masse rühren. \\
	4~Dessertringe mit Margarine ausfetten und auf 1~Backblech setzen.
	Teig 1\breh{}~cm hoch einfüllen und 20~Minuten im Ofen ausbacken. \\
	Sahne mit Sahnesteif und der Hälfte des restlichen Vanillezuckers steif
	schlagen, Schmand glatt rühren und die Sahne unterheben. Zuletzt die
	Milch einrühren. \\
	Biskuitböden aus dem Ofen nehmen und auf je 2~Tellern anrichten. \\
	Das Pflaumenmus mit Orangensaft verrühren, zentimeterhoch auf die
	Biskuitböden streichen und die Schmand-Sahne-Masse auftragen. Diesen
	Vorgang 1~x wiederholen. \\
	Mit Kakao bestreuen und servieren. \\
      \end{zubereitung}

    \mynewsection{Käsekuchen mit Mürbteig}

      \begin{zutaten}
      \end{zutaten}
      \begin{zutat}{Mürbteig}
        150 g & \myindex{Zucker} \\
	100 g & kalte \myindex{Butter} \\
	250 g & \myindex{Mehl} \\
	1 & \myindex{Ei} \\
	1 & \myindex{Eigelb} \\
	1 & \myindex{Vanilleschote} \\
	\breh{} Päckchen & \myindex{Backpulver} \\
	2 Eßlöffel & dunkler \myindex{Kakao} (möglichst gekühlt) \\
      \end{zutat}
      \begin{zutat}{Füllung}
        1,2 kg & \myindex{Quark} \\
	150 g & flüssige \myindex{Butter} \\
	1 Päckchen & \myindex{Vanillepudding}pulver zum Aufkochen \\
	300 g & \myindex{Zucker} \\
	6 & \myindex{Ei}er \\
	3 & \myindex{Eigelb} \\
	40 g & \myindex{Grieß} \\
      \end{zutat}

      \begin{zubereitung}
        Zubereitung Teig: Alle Zutaten bis auf die Vanilleschote in eine
	große Schüssel geben. Vanilleschote der Länge nach aufschneiden, das
	Mark herausschaben und zu den anderen Zutaten geben. Alles zu einem
	glänzenden, glatten Teig verkneten, mit den Händen zu einer Kugel
	formen und eine halbe Stunde in den Kühlschrank (in Frischhaltefolie
	gewickelt) legen. \\
	Den gekühlten Teig in zwei Drittel für den Boden und ein Drittel für
	den oberen Abschluß teilen. Den Boden einer Springform
	(28~cm~\durchmesser{}) mit Backpapier auslegen, die zwei Drittel des
	Teiges grob ausrollen und in die Form drücken. Bei \grad{200}
	im vorgeheizten Backofen 10~Minuten vorbacken. Dann das Backpapier
	abziehen und alles auskühlen lassen. \\
	Zubereitung Füllung: Alle Zutaten in einer großen Schüssel verrühren.
	Den vorgebackenen Teig zurück in die Springform geben. Quarkmasse
	darauf gießen und anschließend mit einer groben Reibe den restlichen
	Mürbteig darüber reiben. Bei \grad{190} etwa 70~Minuten backen.
	Anschließend einen Tag lang auskühlen und fest werden lassen. \\
	Tip: Restliche Vanilleschote mit ca. 200~g Zucker in ein
	Schraubverschlußglas geben. Der Zucker hat nach einigen Tagen das
	Vanillearoma angenommen. \\
      \end{zubereitung}

    \mynewsection{Rhabarber-Käsekuchen}

      \begin{zutaten}
      \end{zutaten}
      \begin{zutat}{Teig}
        250 g & \myindex{Mehl} \\
	125 g & weiche \myindex{Butter} \\
	60 g & \myindex{Zucker} \\
	1 & \myindex{Ei} \\
	1 Prise & \myindex{Salz} \\
	& abgeriebene Schale einer unbehandelten \myindex{Zitrone} \\
      \end{zutat}
      \begin{zutat}{Belag}
        100 g & \myindex{Butter} \\
	150 g & \myindex{Zucker} \\
	2 Päckchen & \myindex{Vanillezucker}\index{Zucker>Vanille-} \\
	2 & \myindex{Ei}er \\
	250 g & \myindex{Speisequark}\index{Quark>Speise-} mit 20~\% Fett \\
	250 g & \myindex{Speisequark}\index{Quark>Speise-} mit 40~\% Fett \\
	80 g & \myindex{Speisestärke} \\
	2 Teelöffel & \myindex{Backpulver} \\
	500 g & geputzter \myindex{Rhabarber}, also mehr einkaufen \\
      \end{zutat}

      \begin{zubereitung}
        Backofen auf \grad{200} vorheizen. Mürbteig zubereiten, eine
	\breh{}~Stunde kalt stellen. Springform (26~cm~\durchmesser{}) mit
	weicher Butter einfetten. Teig auslegen und 3~cm Rand formen. Den
	Boden 10--15~Minuten vorbacken. Butter, Zucker, Vanillezucker und
	Eier schaumig rühren, Quark, Speisestärke und Backpulver unterrühren.
	Rhabarber in 2~cm Stücke schneiden und unterheben. \\
	Quarkmasse auf den vorgebackenen Mürbteigboden verteilen und ca.
	60~Minuten backen. Am Ende eventuell mit Alufolie abdecken. \\
      \end{zubereitung}

    \mynewsection{Zerkrümelter Apfel mit Zabaione}

      \begin{zutaten}
        500 g & Äpfel\index{Aepfel=Äpfel} \\
	100 g & \myindex{Zucker} \\
	1 & \myindex{Zimt}stange, sehr fein zerkleinert \\
	100 g & \myindex{Weizenmehl}\index{Mehl>Weizen-} \\
	60 g & \myindex{Butter} \\
	1 Prise & \myindex{Salz} \\
      \end{zutaten}

      \begin{zutat}{Zabaione}
        5 & \myindex{Ei}gelb \\
	2 & \myindex{Nelke}n \\
	ca. 75 g & feiner \myindex{Zucker} \\
	100 ml & \myindex{Riesling}\index{Wein>Riesling} \\
      \end{zutat}

      \begin{zubereitung}
        Die Äpfel schälen, in Stücke schneiden, in 2 Eßlöffel des Zuckers und
	dem Zimt wälzen und in eine eingefettete Auflaufform geben. Dann das
	Mehl in eine Schüssel geben und mit der Butter zu Bröckchen verkneten,
	den restlichen Zucker und die Prise Salz dazu mischen und alles auf
	die Äpfel bröseln. Bei \grad{200} im Backofen 30--35~Minuten backen,
	bis die Streusel Farbe angenommen haben. \\
	Für die Zabaione die Eigelbe und Nelkenköpfchen in eine Rührschüssel
	geben. Zucker und Wein dazugeben und alles mit dem Schneebesen im
	Wasserbad aufschlagen, bis die Masse zur Creme wird. Vor dem
	Servieren die Nelken entfernen. Als Nachtisch auf einem Teller
	anrichten, die Apfelstücke mit Zabaione umgießen. \\
      \end{zubereitung}

    \mynewsection{Margeritenkuchen}

      % von Martina Meuth und Bernd Neuner-Duttenhofer 
      % Sendung vom 06.04.2009
      % (ES beim WDR am 09.03.2007)

      \begin{zutaten}
        250 g & \myindex{Butter} \\
	250 g & \myindex{Zucker} \\
	8 & \myindex{Ei}er \\
	1 & \myindex{Zitrone} \\
	3 Eßlöffel & \myindex{Maraschino} oder \myindex{Zitronenlikör} \\
	250 g & gesiebtes \myindex{Mehl} \\
	1 gehäufter Teelöffel & \myindex{Backpulver} \\
      \end{zutaten}

      \begin{zutat}{Außerdem}
        250 g & gesiebter \myindex{Puderzucker}\index{Zucker>Puder-} \\
	2--3 Eßlöffel & \myindex{Zitrone}nsaft für den Guß \\
      \end{zutat}

      \begin{zubereitung}
        Die Kastenform\footnote{für eine Kastenform von 1\breh{} Litern
	Inhalt}\footnote{Länge 30~cm, Breite 16~cm, Höhe 10~cm, max.
	\grad{230}, 10~Minuten Abkühlzeit} mit
	Backpapier über den Rand hinaus auslegen. Zimmerwarme Butter und Zucker
	mit dem Handrührer oder in der Küchenmaschine zu einer dicken Creme
	schlagen, dann die Eier hinzufügen und geduldig weiter schlagen. Mit
	abgeriebener Zitronenschale, Zitronensaft und Maraschino (oder
	Zitronenlikör, zum Beispiel Limoncello) würzen. Rasch Mehl und
	Backpulver einarbeiten. In die Form füllen. Bei \grad{200} bei Ober-
	und Unterhitze (etwa 10~Minuten) anbacken, bei \grad{180} schließlich
	30--35~Minuten fertig backen. Es ist etwas schwierig, den weichen
	Kuchen ohne Verluste aus der Form zu bringen. Den lauwarmen Kuchen mit
	Zitronenguß überziehen. Dafür den Puderzucker mit Zitronensaft
	anrühren. Den Guß mit einem breiten Messer oder einer Palette
	gleichmäßig verstreichen. \\
      \end{zubereitung}

    \mynewsection{Mandelkrustenkuchen mit Apfelfüllung}

      % von Martina Meuth und Bernd Neuner-Duttenhofer 
      % Sendung vom 06.04.2009
      % (ES beim WDR am 09.03.2007)

      \begin{zutaten}
        50 g & \myindex{Butter} \\
	100 g & \myindex{Mandelblättchen} \\
	3 & Äpfel\index{Aepfel=Äpfel} \\
	2 & \myindex{Zitrone}n \\
	4 & \myindex{Eiweiß}e \\
	1 & Prise \myindex{Salz}\\
	6 Eßlöffel & \myindex{Zucker} \\
	1 Tütchen & \myindex{Vanillezucker}\index{Zucker>Vanille-} \\
	125 g & gesiebter \myindex{Puderzucker}\index{Zucker>Puder-} \\
	250 g & gemahlene \myindex{Mandel}n \\
      \end{zutaten}

      \begin{zubereitung}
        Das Backpapier anhand der Springform%
	\footnote{für eine Springform von 24 Zentimetern Durchmesser}
	3~cm breiter ausschneiden und um den Boden umklappen. Dann den Ring
	aufsetzen und schließen. Das Papier und den Springformrand mit Butter
	einstreichen. Zwei Eßlöffel des Zuckers auf dem Boden gleichmäßig
	verstreuen, darauf die Mandelblättchen verteilen --- sie sollen dicht
	nebeneinander liegen und eine gleichmäßige Fläche bilden. Die Äpfel
	schälen, vierteln, vom Kerngehäuse befreien. In Scheiben schneiden.
	Damit sie sich nicht verfärben, mit dem Saft einer halben Zitrone
	marinieren. Dann hübsch akkurat übereinander greifend (flach legen) auf
	den Mandelblättchen auslegen. \\
	Inzwischen die Eiweiße mit den Schaumschlägern der Küchenmaschine steif
	schlagen, dabei die Prise Salz hinzufügen und langsam 2 Eßlöffel Zucker
	samt dem Vanillezucker einrieseln lassen. Der Schnee darf nicht wolkig
	oder flockig sein, sondern sollte dicht und cremig wirken.
	Mit der Küchenmaschine dauert das etwa 20~Minuten. \\
	Schließlich die abgeriebene Zitronenschale, den restlichen Zitronensaft
	sowie die mit Puderzucker vermischten gemahlenen Mandeln löffelweise
	unterrühren. Diese Masse auf die Äpfel kippen und glatt streichen. Bei
	\grad{160} im vorgeheizten Ofen (Ober- und Unterhitze!) circa
	45~Minuten backen. In der Form eine halbe Stunde abkühlen lassen, erst
	dann auf eine mit Backpapier oder Alufolie belegte Platte stürzen. Das
	Papier, das sich jetzt auf der Oberfläche befindet, vorsichtig
	abziehen. Falls es sich nicht gut löst, ein wenig anfeuchten. Die
	Mandeloberfläche mit weicher Butter einstreichen und mit 2~Eßlöffel
	Zucker bestreuen. Nochmals für 8--10~Minuten in den \grad{200} heißen
	Ofen (diesmal Oberhitze! Einfach oberste Schiene nehmen) stellen, bis
	die Mandelschicht schön gebräunt ist. \\
      \end{zubereitung}

    \mynewsection{Martina's Käsekuchen}

      % von Martina Meuth und Bernd Neuner-Duttenhofer 
      % Sendung vom 06.04.2009
      % (ES beim WDR am 09.03.2007)

      \begin{zutaten}
      \end{zutaten}

      \begin{zutat}{für den Boden}
        200 g & trockene \myindex{Kekse}
	        (\myindex{Butterkekse}\index{Kekse>Butter-},
		 \myindex{Löffelbiskuit}s oder
		 \myindex{Mürbeteigplätzchen}) \\
        75 g & \myindex{Butter} \\
	75 g & \myindex{Zucker} \\
      \end{zutat}

      \begin{zutat}{für die Füllung}
        750 g & \myindex{Schichtkäse}%
	        \footnote{einen Tag lang in einem Sieb abtropfen lassen}
		durchs Sieb streichen \\
        200 g & \myindex{Zucker} (abzüglich 2 Eßlöffel für den Eischnee) \\
	1 gehäufter Eßlöffel & gesiebter \myindex{Vanillezucker}%
	                                 \index{Zucker>Vanille-} \\
        1 & Prise \myindex{Salz} \\
	2 Eßlöffel & \myindex{Zitronenlikör} oder \myindex{Rum} \\
	& Schale und Saft einer ganzen \myindex{Zitrone} \\
	7 & \myindex{Ei}gelbe \\
	2 gehäufte Eßlöffel & gesiebte \myindex{Speisestärke} \\
	7 & \myindex{Ei}weiße \\
      \end{zutat}

      \begin{zubereitung}
        Zunächst den Boden vorbereiten%
	\footnote{Auf einem Konditorblech machen! Der Kuchen läßt sich schwer
	umbetten, da der Boden klebt und reißt. Zur Not nimmt man dafür
	Backpapier, das über den Rand hinausreicht.}:
	Die Kekse in eine Plastiktüte stecken, mit dem Nudelholz darüberfahren,
	bis alles klein zerkrümelt ist. Die Butter in einem Topf zerlassen, die
	Krümel damit mischen, auch den Zucker hinzufügen und einige Minuten
	quellen lassen. Diese Masse schließlich auf dem Boden der Springform
	verteilen, mit den Fingern gut fest und flach drücken. Die Form in den
	\grad{150} vorgeheizten Backofen stellen, den Boden etwa 10--15~Minuten
	backen. Danach auf einem Gitter abkühlen lassen. \\
	Inzwischen den Schichtkäse in einer Rührschüssel mit dem Handrührer
	glatt rühren, dabei Zucker (zwei Eßlöffel davon für den Eischnee
	beiseite stellen), außerdem Vanillezucker, Salz, Zitronenlikör (oder
	Rum), Zitronenschale und -saft einarbeiten. Schließlich nacheinander
	die Eigelbe und am Ende die Stärke unter die Quarkmasse rühren. Die
	Eiweiße geduldig steif schlagen (dauert mit der Küchenmaschine ca.
	20~Minuten), auf kleiner Stufe, damit viel Luft eingearbeitet und der
	Schnee schön dicht wird. Dabei eine Prise Salz und 2~Löffel Zucker
	hineinrieseln lassen. Zuerst ein Drittel davon unter die Käsemasse
	ziehen, erst dann in zwei Portionen den Rest. In die Springform auf den
	inzwischen abgekühlten Boden gießen. In den \grad{200} heißen Ofen
	stellen (mittlere Schiene), bereits nach 15~Minuten jedoch auf
	\grad{160} herunterschalten. Den Käsekuchen noch einmal 45~Minuten
	backen. Falls dabei die Oberfläche zu dunkel wird, mit einem Stück
	Alufolie abdecken. \\
	Tip: Ein Trick, damit der Kuchen schön hoch bleibt: Nach der halben
	Backzeit (30~Minuten) herausholen, mit einem Messer zwischen Füllung
	und Rand entlangfahren, sie so regelrecht vom Formrand schneiden. Nach
	5~Minuten wieder in den Ofen stellen. Diesen Vorgang 10~Minuten vor
	Ablauf der Backzeit wiederholen. \\
	Also nochmals zum Ablauf: Bei \grad{200} 15~Minuten, bei \grad{160}
	nochmal 15~Minuten. Kuchen raus und Rand beschneiden. 5~Minuten warten.
	Wieder für 20~Minuten in den Ofen stellen. Nochmal rausholen und
	beschneiden. 5~Minuten warten. 10~Minuten zu Ende backen. \\
      \end{zubereitung}

    \mynewsection{Rhabarberkuchen mit Baiser}

      % ARD 14.05.2010

      \begin{zutaten}
      \end{zutaten}

      \begin{zutat}{Belag}
        1 kg & \myindex{Rhabarber} \\
	1 Tütchen & \myindex{Vanillezucker}\index{Zucker>Vanille-} \\
	150 g & \myindex{Zucker} \\
	3 & \myindex{Ei}gelbe \\
	2 Eßlöffel & \myindex{Sahne} \\
	etwas & \myindex{Zucker} \\
      \end{zutat}

      \begin{zutat}{Teig}
        200 g & \myindex{Mehl} \\
	1 Teelöffel & \myindex{Backpulver} \\
	70 g & \myindex{Zucker} \\
	125 g & weiches \myindex{Fett} \\
	1 & \myindex{Ei} \\
	1 Prise & \myindex{Salz} \\
      \end{zutat}

      \begin{zutat}{Schicht auf dem Teig}
	& \myindex{Semmelbrösel} \\
	& gemahlene \myindex{Mandel}n oder \myindex{Haselnüsse} \\
      \end{zutat}

      \begin{zutat}{Baiser}
        3 & \myindex{Ei}weiße \\
	1 Tütchen & \myindex{Vanillezucker}\index{Zucker>Vanille-} \\
	150 g & \myindex{Zucker} \\
      \end{zutat}

      \begin{zubereitung}
        Rhabarber schälen, zuckern und Vanillezucker dazu. 2~Stunden ziehen
	lassen, danach Saft abgießen. \\
	Teig kalt mit einem Messer bereiten, geschmeidigen Mürbteig daraus
	machen. In einer Frischhaltefolie wickeln und 1~Stunde im Kühlschrank
	ruhen lassen. \\
	Teig ausrollen auf einem gefetteten Boden. den Rand der Form fetten
	und die Form schließen. Semmelbrösel und Mandeln oder Haselnüsse
	aufstreuen, bis der Teig bedeckt ist. \\
	Backofen auf \grad{200} ca. 20~Minuten backen, bis der Teig leicht
	braun ist. 3 Eier trennen. \\
	Die Form herausnehmen. Rhabarber (gut abgetropft) darauf verteilen, bis
	der Untergrund nicht mehr sichtbar ist. 3 Eigelbe mit der Sahne und dem
	Zucker verkleppern und gleichmäßig über den Rhabarber gießen. Alles
	wieder in den Backofen bei \grad{200}. \\
	Eiweiße mit 150~g Zucker und 1 Vanillezucker cremig-steif schlagen.
	Form aus dem Ofen, Baiser darauf verteilen. \\
	Backofen auf \grad{150} und ca. 30~Minuten weiterbacken, bis der Baiser
	hart und fest ist (als Probe mit dem Löffel darauf klopfen). \\
	Kuchen rausnehmen und abkühlen lassen. Dann die Form öffnen. Mit Sahne
	servieren. \\
      \end{zubereitung}

    \mynewsection{Zwetschgendatschi}

      % BR Querbeet 03.08.2010

      \begin{zutaten}
        ca. 1\breh{}--2 kg & \myindex{Zwetschgen}%
	                     \footnote{möglichst noch etwas unreife, feste
			     Zwetschgen!} (ergibt 1--1\breh{} kg fertiger
			     Belag) \\
      \end{zutaten}

      \begin{zutat}{Mürbteig}
        80 g & \myindex{Zucker} \\
	160 g & \myindex{Butter} \\
	240 g & \myindex{Mehl} Typ 550 \\
	1 Prise & \myindex{Salz} \\
	\breh{} & \myindex{Vanilleschote} \\
	\breh{} & \myindex{Zitrone}nschale fein abgerieben von einer halben
	          Zitrone \\
      \end{zutat}

      \begin{zutat}{Hefeteig}
        30 g & \myindex{Zucker} \\
	30 g & \myindex{Butter} \\
	300 g & \myindex{Mehl} Typ 550 \\
	5 Prisen & \myindex{Salz} \\
	\breh{} & \myindex{Vanilleschote} \\
	\breh{} & \myindex{Zitrone}nschale fein abgerieben von einer halben
	          Zitrone \\
        30 g & \myindex{Hefe} \\
	ca. 120 ml & \myindex{Milch} \\
	2 & \myindex{Ei}gelb \\
      \end{zutat}

      \begin{zutat}{Franchipanmasse}
        50 g & \myindex{Zucker} \\
	50 g & \myindex{Marzipan} \\
	50 g & \myindex{Butter} \\
	1 & \myindex{Ei} \\
	25 g & \myindex{Mehl} Typ 550 \\
      \end{zutat}

      \begin{zubereitung}
        Bemerkung: Für den Mürbteig sollten alle Zutaten kalt, für den
	Hefeteig zimmerwarm sein! \\
	Als erstes den Hefeteig kneten, damit er kurz gehen kann. Dann den
	Mürbteig zubereiten und die beiden Teige zusammenkneten. \\
	Jetzt ein Backblech (60x40~cm) nicht zu sparsam mit wachsweicher
	Butter bis in alle Ecken gründlich einstreichen. Der Teig nimmt die
	Butter während des Backens auf und der Kuchen bekommt so ein
	besonderes Aroma. \\
	Nun den Teig (gleichmäßige Dicke!) ausrollen und auf das Backblech
	legen. Gut an die Ecken und Ränder hinarbeiten, am besten mit einem
	dünnen Rollholz oder Holzstab. \\
	Die Franchipanmasse wie einen Sandkuchen zubereiten und als Bett für 
	die Zwetschgen mit einer Palette auf den Teig aufstreichen. \\
	Damit der Teig beim Backen nicht zu viel Feuchtigkeit saugt, siebt man
	ihn mit Semmelbröseln ab. \\
	Jetzt die Früchte dachziegelartig auf den Teig legen. Bei \grad{180}
	für 55~Minuten in den vorgeheizten Backofen geben. \\
	Nach der Backzeit streut man Zimtzucker über den Kuchen (je unreifer
	der Belag, desto mehr Zucker). Dann glasiert man die Zwetschgen, d.h.
	streicht mit einem Pinsel Gelee/Tortengußmasse in einer Legerichtung
	der Zwetschgen. \\
      \end{zubereitung}

    \mynewsection{Apfelkuchen}

    % von Martina Meuth und Bernd Neuner-Duttenhofer WDR März 2011
    % gebacken 27.03.2011

      \begin{einleitung}
        Kennen Sie das: einen plötzlichen, unüberwindbaren Hunger, eine
	unbezwingbare Lust nach etwas Süßem? Sogar für solche Bedürfnisse und
	Wünsche haben wir ein Rezept. Alle Zutaten könnten Sie im Haus haben
	--- und wenn nicht, gibt es sie im Supermarkt um die Ecke. \\
	Zutaten für eine Form von 28--30~cm~\durchmesser{}. \\
      \end{einleitung}

      \begin{zutaten}
      \end{zutaten}

      \begin{zutat}{Für den Mürbteig}
        280 g & \myindex{Mehl} \\
	150 g & \myindex{Butter} \\
	100 g & \myindex{Zucker} \\
	1 Prise & \myindex{Salz} \\
	2 & \myindex{Ei}weiße \\
	2 Eßlöffel & lauwarmes \myindex{Wasser} (eventuell) \\
      \end{zutat}

      \begin{zutat}{Für den Belag}
        ca. 1 kg & feste, säuerliche Äpfel\index{Aepfel=Äpfel} (z.B. Golden Delicious,
	           Elstar, Rubinette oder Boskoop) \\
	1 & \myindex{Zitrone} \\
	1 Becher & \myindex{Ricotta}\index{Käse>Ricotta} (200 g) \\
	2 & \myindex{Ei}gelbe \\
	2--3 Eßlöffel & \myindex{Zucker} \\
	& Schale einer halben \myindex{Zitrone} \\
	1 Teelöffel & \myindex{Vanillezucker} \\
	& brauner \myindex{Zucker} zum Bestreuen \\
	30 g & \myindex{Butter} zum Einfetten der Form und für Flöckchen \\
      \end{zutat}

      \begin{zubereitung}
        Aus der zimmerwarmen Butter, Mehl, Zucker und Eiweiß schnell einen
	festen Mürbeteig kneten, eine Prise Salz hinzugeben. Falls sich die
	Zutaten nicht gut mischen, weil die Butter noch zu fest ist, etwas
	warmes Wasser hinzufügen. In einem Folienbeutel eine \breh{}~Stunde
	ruhen lassen, damit der Weizenkleber aufquellen kann und der Teig
	schön mürbe wird. \\
	In der Zwischenzeit die Äpfel schälen, vierteln und das Kerngehäuse
	aus den Vierteln schneiden. Die Äpfel in einer Schüssel in
	Zitronensaft wenden, damit sie nicht braun werden. In einer kleineren
	Schüssel Ricotta, Zucker und Eigelbe rasch zu einer Creme verrühren,
	mit Vanillezucker und geriebener Zitronenschale würzen. \\
	Den Teig ausrollen --- dafür am besten einen großen Folienbeutel
	benutzen: An zwei Seiten mit einem Messer aufschlitzen, ausbreiten, den
	Teig darauf ausrollen und jetzt mithilfe dieser Folie zum Blech
	bugsieren, stürzen und die gebutterte Form damit auskleiden. Vor allem
	den Rand gut andrücken, und dafür sorgen, daß der Boden gleichmäßig
	dünn ist und keine Löcher bildet. Äußerst praktisch ist übrigens eine
	Tarte- oder Quicheform mit Hebeboden, der Kuchen läßt sich nach dem
	Backen mit dem Boden herausnehmen. \\
	Die Ricottacreme auf dem Teigboden verteilen und glatt streichen. Die
	Apfelviertel mit dem Gemüsehobel in gleichmäßig dünne Scheibchen
	schneiden und diese dann dicht an dicht auf der Creme von außen nach
	innen schön akkurat anordnen. Braunen Zucker auf die Apfelscheibchen
	streuen, Butterflöckchen obendrauf und schließlich im heißen Ofen
	backen: \grad{200-220} Ober- und Unterhitze. Wenn möglich, die
	Unterhitze verstärken, damit der Boden knusprig wird, oder die
	Entfeuchtungsstufe einschalten. Nach etwa einer \breh{}~Stunde sollte
	man auf den Apfelscheiben dunkle Karamellspuren sehen können, und der
	Boden sollte hübsch gebräunt sein. Bald, vorzugsweise noch lauwarm,
	essen, zusammen mit einem Klecks kühler, halbsteif geschlagener Sahne.
	Dazu schmeckt ein Tässchen Espresso oder Kaffee. \\
      \end{zubereitung}

    \mynewsection{Orangenkuchen}

      % 25.02.2012 WDR Lokalzeit Münster gebacken am 02.04.2012 sehr lecker

      \begin{zutaten}
        250 g & \myindex{Butter} oder \myindex{Margarine} \\
	200 g & \myindex{Zucker} \\
	350 g & gesiebtes \myindex{Mehl} \\
	4--5 & \myindex{Orange}n, eine davon unbehandelt \\
	1 & unbehandelte \myindex{Zitrone} \\
	1 Päckchen & \myindex{Vanillezucker}\index{Zucker>Vanille-} \\
	1 Päckchen & \myindex{Backpulver} \\
	4 & \myindex{Ei}er \\
	& Saft einer \myindex{Orange} (ca.75 ml) für den Teig \\
      \end{zutaten}

      \begin{zutat}{Glasur}
        4--5 & \myindex{Orange}n, entsaftet (ca.350 ml) \\
	1 Prise & \myindex{Salz} \\
	150 g & \myindex{Puderzucker}\index{Zucker>Puder-} \\
      \end{zutat}

      \begin{zubereitung}
        Butter in einer Schüssel schaumig rühren. Dann Eier, Zucker,
	Vanillezucker zufügen. Eine Orange auspressen und den Saft zu der Masse
	geben. Schalen von unbehandelter Zitrone und Orange abreiben, zufügen.
	Nach und nach gesiebtes Mehl, Backpulver und eine Prise Salz
	unterrühren. Alles gut verrühren. \\
	Backofen auf \grad{200} vorheizen. Eine Kranzform \durchmesser{}~26~cm
	einfetten und mit Paniermehl ausbröseln. Teig hineingießen und
	glattstreichen. 45~Minuten backen. \\
	Restliche Orangen auspressen, Saft beiseite stellen. \\
	Heißen Kuchen stürzen. Mit einer dicken Stricknadel Löcher in den
	Kuchen stechen. Puderzucker mit Orangensaft mischen und langsam über
	den Kuchen und in die Löcher träufeln. \\
	Kuchen abkühlen lassen und mit Schlagsahne servieren, wahlweise mit
	einem Klecks Orangenmarmelade. \\
	Zur Dekoration mit Puderzucker bestäuben und/oder frische Minzeblätter
	obenauf legen. \\
	Tip: Am besten schmeckt der Orangenkuchen, wenn man ihn ein oder zwei
	Tage vor dem Servieren zubereitet, weil er dann gut durchgezogen und
	sehr saftig ist. \\
      \end{zubereitung}

    \mynewsection{Sauerkirsch-Baiser-Torte}

      % NDR 11.04.2012 Mein Nachmittag

      \begin{zutaten}
      \end{zutaten}

      \begin{zutat}{Teig}
        125 g & \myindex{Butter} \\
	125 g & \myindex{Zucker} \\
	1 Päckchen & \myindex{Vanillezucker}\index{Zucker>Vanille-} \\
	5 & \myindex{Ei}gelbe (Eier sauber trennen!) \\
	150 g & \myindex{Mehl} \\
	2 Teelöffel & \myindex{Backpulver} \\
      \end{zutat}

      \begin{zutat}{Baiser}
        5 & \myindex{Ei}weiße (alles fettfrei halten!) \\
	250 g & \myindex{Zucker} \\
	2 Eßlöffel & gehobelte \myindex{Mandeln} \\
      \end{zutat}

      \begin{zutat}{Füllung}
        1 Glas & \myindex{Sauerkirschen} \\
	2 Eßlöffel & \myindex{Zucker} \\
	1 Päckchen & \myindex{Vanillepudding}pulver \\
	\breh{} Liter & \myindex{Sahne} \\
      \end{zutat}

      \begin{zubereitung}
        Teig: Zimmerwarme Butter in eine Rührschüssel geben und mit restlichen
	Zutaten zum Rührteig verarbeiten. 2~Formen \durchmesser{}~28~cm mit
	Backpapier auf dem Boden bereitstellen. Teig auf beide Formen verteilen.
	\\
	Baiser: Küchenmaschine mit fettfreiem Behälter: Eiweiße rein und
	schlagen, erst zum Schluß den Zucker einrieseln lassen. Warum? Das
	Baiser wird sonst zu hart zum schneiden. Das Baiser aufteilen. Etwas
	weniger als die Hälfte bis ein Drittel auf den unteren Boden
	aufstreichen mit einer Teigkarte (Schaber). Den größeren Teil in einen
	Spritzbeutel mit Tülle füllen und Rosetten auf den anderen Boden
	spritzen. Gehobelte Mandeln auf beide Böden aufstreuen, in den Backofen
	bei \grad{190} für 30--35~Minuten. \\
	Füllung: Sauerkirschen mit dem Saft und Zucker in einen Topf geben. Das
	Puddingpulver in kleiner Schüssel mit etwas Saft glatt rühren.
	Vorsichtig (die Kirschen müssen unbedingt ganz bleiben, nicht
	zermatschen) an die Kirschen geben, umrühren und aufkochen lassen.
	Abkühlen bzw. kalt werden lassen. Sahne steif schlagen. \\
	Unteren Boden mit Metallmanschette (Konditorring) versehen und Kirschen
	darauf verteilen, dann die Sahne darüber. Den oberen Boden auf die
	Sahne setzen (die Böden kann man auch am Vortag bereiten, die Füllung
	sollte jedoch am selben Tag bereitet werden und die Torte kann in
	den Kühlschrank). \\
      \end{zubereitung}

    \mynewsection{Ostpreußische Glumstorte}

      % Martina Kömpel, Servicezeit WDR 13.02.2015

      \begin{zutaten}
        200 g & zimmerwarme \myindex{Butter} \\
        5 & \myindex{Ei}gelbe \\
        300 g & \myindex{Zucker} \\
        6 Eßlöffel & \myindex{Weichweizengrieß}\index{Grieß>Weichweizen-} \\
        1 kg & \myindex{Magerquark}\index{Quark>Mager-} \\
        1 & unbehandelte \myindex{Zitrone} \\
        1 Päckchen & \myindex{Backpulver} \\
        eventuell & \myindex{Puderzucker}\index{Zucker>Puder-} \\
      \end{zutaten}

      \begin{zubereitung}
        Benötigt wird eine Springform \durchmesser{26 cm}. \\
	Backofen auf \grad{175} vorheizen. \\
	Butter glattrühren und die Eigelbe und den Zucker nach und nach zugeben
	und schaumig aufschlagen. Grieß und Backpulver unterheben, den Quark
	löffelweise hinzugeben und zum Schluß Zitronenschale und -saft
	zufügen.\\
	Den Boden einer Springform mit Backpapier auslegen und den Rand
	aufsetzen. Die Masse in die Springform füllen und glattstreichen. \\
	Im vorgeheizten Ofen 60~Minuten backen. \\
	Wer mag, kann die Torte nach dem Abkühlen mit Puderzucker bestäuben. \\
	Am Samstag, 13.06.2015 abends gebacken, Sonntag gegessen: Extrem
	lecker!! \\
      \end{zubereitung}

    % \mynewsection{Text}

      % \begin{zutaten}
      % \end{zutaten}

      % \begin{zubereitung}
      % \end{zubereitung}
