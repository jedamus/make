
% created Montag, 10. Dezember 2012 16:17 (C) 2012 by Leander Jedamus
% modified Montag, 10. Dezember 2012 16:29 by Leander Jedamus

  \mynewchapter{Auberginen}

    \mynewsection{Auberginen gebraten in Tomatensoße}

      \begin{zutaten}
        1 & mittelgroße \myindex{Aubergine} ca. 12--14~cm lang \\
        1--2 & \myindex{Knoblauchzehe}n \\
        1 kleine Dose & \myindex{Tomate}n \\
        2 Eßlöffel & \myindex{Olivenöl}\index{Oel=Öl>Oliven-} \\
        & \myindex{Salz}, \myindex{Pfeffer} aus der Mühle, \myindex{Oregano} \\
      \end{zutaten}

      \personen{1--2}

      \garzeit{ca. 20}

      \begin{zubereitung}
        Die Aubergine auf der Oberseite und auf der Unterseite dünn schälen,
	Stengel und Blatt abschneiden. Die Aubergine in Längsscheiben schneiden
	(ca. 0,5~cm dick, ergibt etwa 4~Scheiben), jede Scheibe auf Ober- und
	Unterseite gut salzen, die Scheiben aufeinanderlegen. Eventuell
	2~Stapel machen. Die Scheiben auf ein Küchenbrett oder einen flachen
	Teller legen und schräg stellen. Am besten legt man etwas zum
	Beschweren drauf. Das Salzen, schrägstellen und beschweren hilft, den
	bitteren Saft aus der Aubergine zu bekommen. 30~Minuten so stehen
	lassen. Danach die Scheiben gut drücken und trocken machen (Küchenkrepp
	oder sauberes Geschirrtuch). \\
        Dose Tomaten öffnen, Deckel nur leicht anheben, Saft ablaufen lassen,
	danach den Deckel ganz öffnen. Olivenöl und Gewürze bereitstellen.
	Nudelwasser aufsetzen, Pfanne erhitzen und mit Salz ausstreuen (damit
	es nicht spritzt), Knoblauch schälen und in Stifte schneiden. Öl in die
	Pfanne geben, heiß werden lassen, aber es soll möglichst nicht rauchen,
	dann Pfanne auf 1~\breh{}~runterschalten und die trockenen
	Auberginen-Scheiben nebeneinander in die Pfanne geben. Die Scheiben
	sachte, d.h. langsam, goldbraun werden lassen, Knoblauch-Stifte
	dazugeben und mitbraten, alles pfeffern. \\
        (Nudeln ins kochende Salzwasser geben). \\
        Nachsehen, ob die Scheiben ausreichend gebräunt sind, dann wenden,
	pfeffern. Wenn die zweite Seite der Scheiben ebenfalls gebräunt ist,
	über die Scheiben breit Oregano streuen (nicht zuviel) und die Dose
	Tomate darübergeben. Die Tomaten mit der Gabel zerteilen/zerdrücken,
	dabei die Auberginen-Scheiben ganz lassen. Alles langsam köcheln
	lassen, Platte rechtzeitig abschalten (wenn die Nudeln fast gar sind
	z.B.) \\
        Wenn die Nudeln vor dem Gemüse fertig sind: Nudeln abgießen, dann
	zurück in den heißen Topf, Deckel drauf, aber nicht auf die heiße
	Platte stellen. \\
        Wenn die Auberginen vorher gar sind, auf der Platte in der Pfanne
	abkühlen lassen. Schmecken lauwarm auch sehr gut. \\
        Nudeln mit Parmesan mischen, Auberginen-Scheiben daneben legen, darüber
	Tomatensoße geben. \\
      \end{zubereitung}

    \mynewsection{Auberginen gefüllt mit Mozzarella und Tomate}

      \begin{zutaten}
        2 größere & runde \myindex{Aubergine}n \\
        300--400 g & \myindex{Kirschtomaten} (kleine Tomaten) \\
        1 Handvoll & \myindex{grüne Oliven}\index{Oliven>grün} ohne Füllung \\
        2 Beutel & \myindex{Mozzarella}\index{Käse>Mozzarella} (2 Kugeln) \\
        1 Handvoll & frisches \myindex{Basilikum} \\
        ca. 1 Teelöffel & \myindex{Oregano} \\
        & \myindex{Salz} \\
        & \myindex{Pfeffer} \\
        1 & \myindex{Knoblauchzehe} \\
        & \myindex{Olivenöl}\index{Oel=Öl>Oliven-} \\
      \end{zutaten}

      \personen{2}

      \garzeit{20--30}

      \begin{zubereitung}
        Aubergine dünn abschälen und in Olivenöl sehr heiß ringsrum braun
	anbraten. Abgekühlte Auberginen mittig durchschneiden, aushöhlen,
	0,5--1~cm Rand stehen lassen. In geölte Auflaufform legen und mit wenig
	Oregano, Salz und Pfeffer bestreuen. \\
        \emph{Füllung}: Das Ausgehöhlte in Würfel schneiden, in der Pfanne mit
	Olivenöl braten. Gehackten Knoblauch, Salz, Pfeffer und Oregano dazu
	geben. Tomaten waschen, halbieren. Mozzarella in Würfel schneiden,
	Basilikumblätter waschen, in Streifen schneiden, die Oliven in
	Scheiben. Alle Zutaten in einer Schüssel mischen, Auberginen damit
	füllen. \\
        Bei \grad{200} ca. 20--30~Minuten im Backofen garen. \\
      \end{zubereitung}

    \mynewsection{Aubergine-Medaillons mit Tomaten und Mozzarella überbacken}

      \begin{zutaten}
        1 & \myindex{Aubergine} Viola (kreiselförmig) \\
        & \myindex{Salz} \\
        & \myindex{Olivenöl} \\
        & \myindex{Balsamico-Essig}\index{Essig>Balsamico-} \\
        & \myindex{Tomate}n \\
        & \myindex{Mozzarella}\index{Käse>Mozzarella} \\
      \end{zutaten}
      \begin{zutat}{Tomatensoße}
        \breh{} & \myindex{Zwiebel} \\
        1 & \myindex{Knoblauchzehe} \\
        1 Teelöffel & \myindex{Kapern} \\
        & \myindex{Weißwein} \\
        250 ml & fertige \myindex{Tomatensoße} \\
        & \myindex{Basilikum} \\
        & \myindex{Oregano} \\
        & \myindex{Petersilie} \\
        & \myindex{Salz} \\
        & \myindex{Pfeffer} \\
      \end{zutat}

      \personen{4}

      \garzeit{20}

      \begin{zubereitung}
        Aubergine in ca. 1~cm dicke Scheiben schneiden, die Scheiben salzen,
	mit Olivenöl und Balsamico-Essig bestreichen, kurz anbraten oder
	grillen. \\
        Tomatensoße: Zwiebel klein würfeln, Knoblauch pressen, zusammen mit
	den Kapern in Olivenöl anbraten --- wenn die Zwiebeln glasig sind, mit
	Weißwein ablöschen. Päckchen Tomatensoße hinzufügen, mit Basilikum,
	Oregano und Petersilie würzen und alles 20--25~Minuten auf kleiner
	Flamme kochen lassen. Mit Salz und Pfeffer abschmecken. \\
        Angebratene Auberginen-Scheiben jeweils mit eine Scheibe Tomate und
	eine Scheibe Mozzarella belegen, dann mit Tomatensoße bedecken. Alles
	in einer feuerfesten Form 20~Minuten im Ofen (\grad{180}, vorgeheizt)
	überbacken. \\
      \end{zubereitung}

    \mynewsection{Farfalle mit Pesto alla calabrese und gebratenen Auberginen}

      \begin{zutaten}
        350 g & \myindex{Farfalle} \\
        2 & \myindex{rote Paprika}\index{Paprika>rot}schoten \\
        2 & \myindex{Aubergine}n \\
        30 g & \myindex{Mandel}n \\
        60 g & frischer \myindex{Ricotta}\index{Käse>Ricotta} \\
        \breh{} & \myindex{Zwiebel} \\
        4 Eßlöffel & \myindex{Olivenöl} (extra vergine) \\
        20 g & geriebener \myindex{Parmesan}\index{Käse>Parmesan} \\
        1 Bund & \myindex{Petersilie} \\
        & \myindex{Salz} \\
        & \myindex{Pfeffer} \\
      \end{zutaten}

      \personen{4}

      \garzeit{20}

      \begin{zubereitung}
        Die Zwiebel hacken und mit 1~Eßlöffel Olivenöl leicht andünsten.
	Die Haut der Paprikaschoten abziehen, die Schoten in Stücke schneiden
	und zur Zwiebel geben. Circa 6~Minuten kochen und einen Schöpflöffel
	Wasser dazugeben. Die Auberginen würfeln und separat 2~Minuten in
	2~Eßlöffel Olivenöl anbraten. Das überschüssige Öl dann mit Küchenkrepp
	entfernen. Die Paprikaschoten, die Mandeln, den Parmesankäse, den
	Ricotta-Käse und das restliche Öl im Mixer pürieren und mit Salz und
	Pfeffer abschmecken. Die Farfalle in reichlich Salzwasser kochen,
	abgießen und mit der Soße anrichten. Mit den gebratenen
	Auberginenwürfeln und der gehackten Petersilie garnieren. \\
      \end{zubereitung}

    \mynewsection{Auberginenscheiben gebraten}

      \begin{zutaten}
        2 mittelgroße & \myindex{Aubergine}n \\
        & \myindex{Olivenöl}\index{Oel=Öl>Oliven-} \\
        & \myindex{Salz} \\
        & \myindex{Pfeffer} \\
        & \myindex{Knoblauch}, klein gehackt oder dünne Scheiben \\
        & \myindex{Oregano} \\
      \end{zutaten}

      \personen{2}

      \begin{zubereitung}
        Aubergine in 1~cm dicke Stücke längs aufschneiden. Die äußeren Schalen
	eventuell wegwerfen. Stücke salzen und auf einen schräggestellten
	Teller legen für 30~Minuten. Dann abspülen und trocknen (Krepp oder
	Geschirrtuch). In heißem Olivenöl anbraten. Die gewendete Scheibe
	würzen mit Salz, Pfeffer, Oregano und Knoblauch. Zum Servieren in
	Tomatensoße (siehe Seite \pageref{tomatensosse}) legen. \\
      \end{zubereitung}

    \mynewsection{Auberginen mit Kräutersoße}

      \begin{zutaten}
        500 g & \myindex{Aubergine}n \\
	100 ml & \myindex{Weißwein}\index{Wein>weiß} oder
	         verdünnter \myindex{Zitrone}nsaft \\
        \breh{} Bund & \myindex{Basilikum} \\
	& \myindex{Salz} \\
	& Öl\index{Oel=Öl} zum Braten \\
	1 Becher & \myindex{\cremefraiche{}} (150 g) \\
	\breh{} Bund & \myindex{glatte Petersilie}\index{Petersilie>glatt} \\
	\breh{} Zitrone \\
	1 Eßlöffel & \myindex{Weinbrand} (eventuell) \\
      \end{zutaten}

      \personen{4}
      \kalorien{320}

      \begin{zubereitung}
        Auberginen in 2~Zentimeter dicke Scheiben schneiden. Wein mit einem
	Eßlöffel gehackten Basilikumblättern und Salz mischen und über die
	Auberginenscheiben gießen. Etwa 30~Minuten stehenlassen, dabei einmal
	wenden. Abgetropfte Auberginenscheiben trockentupfen und in heißem Öl
	bei mittlerer Hitze von beiden Seiten goldbraun braten. \\
	Für die Soße \cremefraiche{} und restliche Kräuter im Mixer oder mit
	dem Schneidstab des Handrührers verschlagen. Mit Zitronensaft, Salz
	und Weinbrand abschmecken. Soße zu den warmen oder kalten
	Auberginenscheiben servieren. \\
      \end{zubereitung}

    \mynewsection{Auberginencreme}

      \begin{einleitung}
        Sie läßt sich wunderbar auf Vorrat zubereiten --- in Schraubgläser
        gefüllt, hält sie sich im Kühlschrank mehr als eine Woche,
	vorausgesetzt, die Oberfläche ist immer von einem Ölfilm bedeckt, der
	vor Luftkontakt schützt. Bei \grad{90} \breh{}~Stunde lang
	sterilisiert, hält sich die Creme wie eine Konserve. \\
      \end{einleitung}

      \begin{zutaten}
        2 & \myindex{Aubergine}n\footnote{nur kleine Auberginen nehmen! Große
	                                  haben zuviel Wasser.} \\
        3--4 & rote \myindex{Chilischote}n \\
	1 & feste \myindex{Tomate} \\
	3 & \myindex{Knoblauchzehe}n \\
	\brea{} l & \myindex{Olivenöl}\index{Oel=Öl>Oliven-} \\
	& \myindex{Salz} \\
	& \myindex{Pfeffer} \\
	1--2 Eßlöffel & \myindex{Zitrone}nsaft \\
	1 Prise & \myindex{Zucker} \\
      \end{zutaten}

      \personen{6}

      \begin{zubereitung}
        Die Auberginen auf einem Blatt Alufolie in den \grad{200} heißen
	Backofen legen und 20--30~Minuten backen, bis sich die Haut rundum sehr
	dunkel gefärbt hat und die Frucht auf Fingerdruck nachgibt. Immer
	wieder drehen, damit die Hitze gleichmäßig einwirken kann. \\
	Inzwischen die Chilis an der Herdflamme --- wer kein Gas hat, nimmt den
	Gasbrenner, den man auch für die Cr\`eme brul\'ee braucht, notfalls tut
	es auch der Feueranzünder --- rundum rösten, bis die Haut schwarz wird
	und sich löst. Sie läßt sich jetzt ganz leicht unter kaltem Wasser
	abreiben. Chilis entkernen und in den Mixer füllen. Die gehäutete
	Tomate ebenfalls hinzufügen, auch die geschälten Knoblauchzehen. \\
	Schließlich die Auberginen aus dem Ofen holen, die Früchte aufschlitzen
	und mit einem Löffel das weiche Fruchtfleisch herausholen. Ebenfalls in
	den Mixer füllen, jetzt auch Öl, Salz und Pfeffer hinzufügen und alles
	auf höchster Stufe zerkleinern und zu einer glatten Creme mixen.
	Abschmecken, vor allem mit Zitronensaft, eventuell eine Prise Zucker
	hinzufügen, falls die Chilis sehr scharf waren. \\
	Tip: Diese Grillsoße schmeckt nicht nur auf Brot, sondern ebenso gut
	zu Würstchen oder zu Grillfleisch. \\
      \end{zubereitung}

    \mynewsection{Bandnudeln mit Auberginen-Tomaten}

      \begin{zutaten}
        1 & \myindex{Aubergine} (150 g) \\
        1 große & \myindex{Tomate} (150 g) \\
        1 Eßlöffel & Öl\index{Oel=Öl} \\
        3 & frische \myindex{Salbeiblätter} \\
        & \myindex{Pfeffer} aus der Mühle \\
        60 g & \myindex{Bandnudeln}\index{Nudeln>Band-} (Tagliatelle) \\
      \end{zutaten}

      \kalorien{400}

      \begin{zubereitung}
        Eine Aubergine in dünne Scheiben schneiden, mit Salz bestreuen und
	20~Minuten ziehen lassen. Dann abspülen und tockentupfen. Öl in der
	Pfanne erhitzen und die Auberginenscheiben hellbraun braten. Die
	kleingeschnittene Tomate und Salbei dazugeben. Leicht pfeffern. Die
	bißfest gekochten Nudeln mit dem Gemüse anrichten. \\
      \end{zubereitung}

    \mynewsection{Sizilianische Bandnudeln mit Auberginen}

      \begin{zutaten}
        1 & \myindex{Aubergine} \\
        2 & \myindex{Knoblauchzehe}n \\
        & \myindex{Jodsalz}\index{Salz>Jod-} \\
        4 Eßlöffel & natives \myindex{Olivenöl}\index{Oel=Öl>Oliven-} extra \\
        500 g & \myindex{passierte Tomate}\index{Tomate>passiert}n \\
        & \myindex{Pfeffer} \\
        500 g & \myindex{Bandnudeln}\index{Nudeln>Band-}, gewalzt \\
        120 g & \myindex{Pinienkerne} \\
        1 Topf & frisches \myindex{Basilikum} \\
      \end{zutaten}

      \personen{4}
      \garzeit{20}

      \begin{zubereitung}
        Aubergine waschen, abtrocknen, längs in 1~cm breite Scheiben schneiden.
	Diese wiederum längs in dünne Streifen schneiden. Knoblauch schälen und
	fein hacken. Reichlich Wasser mit einer Prise Salz in einem großen Topf
	aufkochen. Olivenöl in einem zweiten Topf erhitzen, Auberginenstreifen
	darin 5~Minuten braten. Knoblauch und passierte Tomaten dazugeben und
	bei mittlerer Hitze 5--7~Minuten kochen. Mit Salz und Pfeffer würzen.
	\\
	Währenddessen die Bandnudeln nach Packungsanweisung kochen. Pinienkerne
	in einer beschichteten Pfanne ohne Fett rösten, bis sie hellbraun sind.
	Auf einem Teller abkühlen lassen. \\
	Die Bandnudeln in ein Nudelsieb gießen und abtropfen lassen.
	Anschließend Nudeln behutsam mit der Auberginen-Tomaten-Soße
	vermischen, auf Tellern verteilen. Mit Pinienkernen bestreuen und mit
	gehackten Basilikumblättchen dekorieren. \\
      \end{zubereitung}

    \mynewsection{Auberginen in Öl (Melanzane sott'olio)}

      \begin{zutaten}
	6 & \myindex{Aubergine}n (1,2~kg) \\
	& \myindex{Salz} \\
        \brev{} l & \myindex{Olivenöl}\index{Oel=Öl>Oliven-} zum Braten \\
	& frisch gemahlener \myindex{Pfeffer} \\
	2 & zerdrückte \myindex{Knoblauchzehe}n \\
	& frisches \myindex{Basilikum} \\
	& frische \myindex{Minze} \\
        \brea{} l & italienischer \myindex{Rotweinessig}
	                          \index{Essig>Rotwein-} \\
        \brea{} l & \myindex{Olivenöl}\index{Oel=Öl>Oliven-} \\
      \end{zutaten}

      \begin{zubereitung}
        Auberginen waschen, in Scheiben schneiden und mit Salz bestreuen.
	20~Minuten stehen lassen, dann abtupfen. In heißem Öl von beiden Seiten
	leicht braun braten, dabei auf jeder Seite pfeffern. In eine flache
	Schale schichten. Jede Schicht mit Knoblauch, gehackten oder ganzen
	Kräutern und Essig würzen. Zum Schluß Öl darüber träufeln. Über Nacht
	bechweren und kühl stellen. \\
	Dazu ein Ros\'e aus Apulien ,,Salice Salentino``, samtig reif und
	trocken. \\
      \end{zubereitung}

    \mynewsection{Auberginen-Carpaccio}

      % Tim Mälzer kocht 14.08.2010

      \begin{zutaten}
        1 & \myindex{Aubergine} (ca. 250 g, am besten die violette) \\
	& \myindex{Salz} \\
	2--3 Eßlöffel & \myindex{Zitrone}nsaft \\
	1--2 Eßlöffel & flüssiger \myindex{Honig} \\
	6 Eßlöffel & gutes \myindex{Olivenöl}\index{Oel=Öl>Oliven-} \\
	& \myindex{Pfeffer} aus der Mühle \\
	50 g & \myindex{Rauke}\index{Rucola} \\
	30 g & \myindex{Parmesan} \\
      \end{zutaten}

      \personen{4}

      \begin{zubereitung}
        Aubergine mit einer Gabel rundum einstechen. Die Aubergine in
	kochendem Salzwasser 4--5~Minuten kochen, herausnehmen und gut
	abtropfen lassen. Im Gefrierfach 10--15~Minuten kühlen. \\
	Zitronensaft, Honig, Olivenöl und etwas Salz kräftig verrühren.
	Aubergine in dünne Scheiben schneiden, auf eine gekühlte Platte legen
	und für weitere 10~Minuten ins Gefrierfach stellen. \\
	Vor dem Servieren mit der Vinaigrette beträufeln, salzen und pfeffern.
	Rauke grob schneiden und über den Auberginenscheiben verteilen. Mit
	gehobeltem Parmesan bestreuen. \\
      \end{zubereitung}


    % \mynewsection{Text}

      % \begin{zutaten}
      % \end{zutaten}

      % \begin{zubereitung}
      % \end{zubereitung}
