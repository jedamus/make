
% created Montag, 10. Dezember 2012 16:23 (C) 2012 by Leander Jedamus
% modified Montag, 10. Dezember 2012 16:30 by Leander Jedamus

  \mynewchapter{Speisekarte Deutschland}

    \mynewsection{Kohl und Pinkel}

      % NDR 14.10.2011 

      \begin{zutaten}
        500 g & \myindex{Grünkohl}\index{Kohl>Grün-} \\
	40 g & fetten \myindex{Speck} \\
	1 & \myindex{Pinkel} \\
	1 & \myindex{Kochwurst} \\
	50 g & \myindex{Zwiebel}n \\
	& \myindex{Hafergrütze} \\
	& \myindex{Kasseler} \\
	& \myindex{Kasseler-Knochen} \\
	& \myindex{Salz} \\
	& \myindex{Pfeffer} \\
	& \myindex{Senf} \\
	& getrocknete \myindex{Piment}körner \\
      \end{zutaten}

      \begin{zubereitung}
        \begin{enumerate}
	  \item Den Speck komplett auslassen. Grobe Zwiebelwürfel im Fett
	        glasig dünsten. Den Kohl zugeben und weiterdünsten.
		Den Kohl anschließend mit etwas Wasser und Salz köcheln.
		Nicht zu viel Wasser, lieber gegebenenfalls etwas Wasser
		nachgeben. Den Kasselerknochen mit in den Kohl geben und ca.
		1~Stunde bei kleiner Flamme mit dem Kohl ziehen lassen.
          \item Vorzugsweise kann etwas später auch Pinkel und Kochwurst mit
	        in den Kohl gegeben werden. Piment im Mörser zerreiben
		(nicht zerstoßen) und im Kohl noch mindestens 20~Minuten ziehen
		lassen. Mit Salz, Pfeffer und Senf nach belieben abschmecken.
		Zum Schluß die Hafergrütze darüberstreuen, nicht unterrühren
		(da der Kohl sonst zu leicht anbrennen kann), bis die
		Flüssigkeit leicht angebunden ist und nochmals abschmecken.
	\end{enumerate}
	Am besten gelingt der Kohl, wenn die Schritte unter 1. am Vortag
	ausgeführt werden und der Kohl am zweiten Tag erhitzt wird, um mit den
	Schritten unter 2. fortzufahren. \\
      \end{zubereitung}

    \mynewsection{Spargel mit Butter und Schinken}

      % NDR 14.10.2011

      \begin{zutaten}
        500 g & \myindex{Spargel} \\
	150 g & geschnittener \myindex{Landschinken}\index{Schinken>Land-} \\
	& \myindex{Butter} \\
	& \myindex{Salzkartoffel}\index{Kartoffel>Salz-}n \\
	& \myindex{Salz} \\
	& \myindex{Zucker} \\
	& \myindex{Zitrone}nsaft \\
      \end{zutaten}

      \begin{zubereitung}
        Den Spargel schälen und die Spitzen auslassen, die Enden abschneiden
	und bündeln. Gewürzt mit einer Prise Salz, Zucker, etwas Butter und
	einem Spritzer Zitrone in kochendes Wasser geben. Ca. 15--18~Minuten
	bißfest kochen. Mit zerlassener Butter und dem Landschinken servieren.
	Dazu Salzkartoffeln reichen. \\
     \end{zubereitung}

    \mynewsection{Matjes nach Hausfrauenart}

      % NDR 14.10.2011

      \begin{zutaten}
        12 & \myindex{Matjes}filets \\
	200 g & \myindex{Joghurt} \\
	200 g & \myindex{Schmand} \\
	2 Eßlöffel & \myindex{Mayonnaise} \\
	5 & Äpfel\index{Aepfel=Äpfel} (Elstar) \\
	1 & \myindex{Gemüsezwiebel}\index{Zwiebel>Gemüse-} \\
	2 kleine & \myindex{rote Zwiebel}\index{Zwiebel>rot}n \\
	3 & \myindex{Gewürzgurke}\index{Gurke>Gewürz-}n \\
	2 & \myindex{Lorbeer}blätter \\
	1 Tasse & \myindex{Birnensaft} \\
	1 Eßlöffel & \myindex{Worcestershiresoße} \\
	& \myindex{Salz} \\
	& \myindex{Pfeffer} \\
	& \myindex{Zucker} \\
      \end{zutaten}

      \personen{4}

      \begin{zubereitung}
        Den Joghurt mit dem Schmand glatt rühren und mit der Mayonnaise binden.
        Mit Salz, Zucker und Pfeffer abschmecken. Die Zwiebel, Gewürzgurke
        geschält und die entkernten Äpfel in gleichmäßig feine Spalten und
        Streifen schneiden. Worcester, Lorbeer und den Birnensaft dazugeben,
        vermengen und gut durchziehen lassen. Danach die Soße unterheben und
        mit dem Matjes anrichten. Dazu passen frische Pellkartoffeln,
        Bratkartoffeln oder auch Schwarzbrot.
      \end{zubereitung}

    \mynewsection{Labskaus mit Spiegelei}

      % NDR 14.10.2011

      \begin{zutaten}
        1 kg & gepökelte \myindex{Rinderbrust} \\
	300 g & \myindex{Zwiebel}n \\
	500 g & mehlige \myindex{Kartoffel}n \\
	10 & \myindex{Ei}er \\
	1 & \myindex{Salzgurke}\index{Gurke>Salz-} \\
	& \myindex{Rote Beete} \\
	& geschroteter \myindex{Pfeffer} \\
      \end{zutaten}

      \begin{zubereitung}
        Die Rinderbrust mit den Zwiebeln, geschält und geviertelt, und den
        Pfeffer in einem mit 2~Liter Wasser gefüllten Topf geben und ca.
        2~Stunden lang kochen. Ist danach die Rinderbrust bißfest und weich,
        den gesamten Inhalt aus dem Topf nehmen und warm stellen. Nur die Brühe
        bleibt. Nun die Kartoffeln in der Brühe gar kochen und zerstampfen, so
        daß ein Brei entsteht. Überschüssige Brühe abgiessen. Dann den
	vorherigen Inhalt des Topfes, den wir warm gestellt haben, durch einen
	Fleischwolf, möglichst mit grober Scheibe, drehen und wieder zurück in
	den Topf mit den Stampfkartoffeln geben, leicht köcheln lassen, mit
	Hausgewürzen wie z.B. Boullionwürfel, Worcestershiresoße etc. nochmals
	abschmecken. \\
        Nun mit Rote Beete und Salzgurke flankiert und Spiegeleiern gekrönt, ist
        der hanseatische Gaumenschmaus perfekt. \\
     \end{zubereitung}

    \mynewsection{Pommerscher Gänsebraten}

      % NDR 14.10.2011

      \begin{zutaten}
        1 & junge \myindex{Gans} \\
	4--6 & säuerliche Äpfel\index{Aepfel=Äpfel} \\
	1 & \myindex{Quitte} \\
	\breh{} l & \myindex{Gemüsebrühe} \\
	& \myindex{Beifuß} \\
	& \myindex{Salz} \\
	& \myindex{Pfeffer} \\
	& \myindex{Rosinen}, in Rotwein eingeweicht \\
	& \myindex{Backpflaumen}\index{Pflaumen>Back-} \\
	& \myindex{Pumpernickel} \\
      \end{zutaten}

      \begin{zubereitung}
        Der Gans die Innereien und dickes Fett entnehmen, die Flügel und den
        Hals abschneiden. Die Innereien in Würfel schneiden. Die gewaschenen
        Äpfel vierteln und entkernen. Innereien, Rosinen, Backpflaumen,
        Schwarzbrot, Quitte, etwas geschnittenen Beifuß und gemahlenen Pfeffer
        mit den Äpfeln vermengen. Die Gans innen mit Salz würzen und die
        Füllung einstopfen. Die Gans vernähen, Keulen und Flügel so binden, daß
        die Gänsebrust hochgewölbt wird. Die Gans von außen kräftig mit Salz
        abreiben und mit der Brust nach unten in den Bräter setzen. Das
        Gänsefett, die Obst- und Fleischabschnitte sowie etwas Beifuß in den
        Bräter geben und mit reichlich Wasser angießen. Den Gänsebraten 1~Stunde
        bei \grad{180} backen, dann die Gans auf den Rücken legen und für ca.
        1\breh{}--2~Stunden weiter backen. Regelmäßig mit dem Bratenfond
        übergießen. Die Gans ist fertig, wenn beim Anstechen der Keulen klare
        Flüssigkeit austritt. \\
        Den fertigen Braten aus dem Bräter nehmen und den Bratensatz mit dem
        Fond ablöschen. Den Fond abseihen und entfetten. Den entfetteten Fond
        kräftig einkochen und als Soße verwenden. \\
      \end{zubereitung}

    \mynewsection{Schweinsbraten mit Kartoffelknödel und Biersoß'}

      % 06.10.2011 Wir in Bayern BR Platz 1

      \begin{zutaten}
      \end{zutaten}

      \begin{zutat}{Fleisch}
        2 kg & dicke \myindex{Schweineschulterstück} mit Schwarte \\
	& \myindex{Salz} \\
	& \myindex{schwarzer Pfeffer}\index{Pfeffer>schwarz} gemahlen \\
	1 Eßlöffel & \myindex{Kümmel} \\
	2 Zehen & \myindex{Knoblauch} fein gehackt \\
      \end{zutat}

      \begin{zutat}{Soße}
        1 kg & \myindex{Schweinsknochen} vom Metzger klein gehackt \\
	2 mittelgroße & \myindex{Zwiebel}n mit Schale geviertelt \\
	ca. 200 ml & \myindex{Bier} \\
	ca. 500 ml & \myindex{Bratengrundsoße} \\
	& \myindex{Wasser} nach Bedarf \\
      \end{zutat}

      \personen{6}

      \begin{zubereitung}
        Knochen und Zwiebeln in einem großen Bräter (Reine) im Ofen bei starker
        Hitze in etwas Fett anbraten. Fleisch dann kräftig mit Salz, Pfeffer,
        Kümmel und Knoblauch würzen, einreiben. Mit der Schwarte nach unten auf
        Knochen und Zwiebeln setzen, etwa 2~cm hoch. Wasser angießen. Für
        30--40~Minuten bei \grad{130} in den Ofen schieben, bis die Schwarte
        ,,geschmeidig`` ist. Jetzt die Schwarte einschneiden, etwa alle
	\breh~cm, dann läßt sich der Braten besser aufschneiden. Nun den Braten
	im Ofen, bei \grad{120--130} (wenn möglich keine Heißluft verwenden,
	die trocknet zu sehr aus), behutsam fertig garen, das dauert noch mal
        1\breh{}--2~Stunden. Dabei öfter mit etwas Bier und Wasser übergießen.
        Der Braten muß immer feucht sein. Das Fleisch ist gar, wenn beim
        Einstechen mit einer Nadel fast klarer Fleischsaft austritt. Falls Sie
        ein Fleischthermometer besitzen: die Kerntemperatur sollte \grad{70}
        sein. Dann aus dem Ofen nehmen und mindestens 20~Minuten ruhen lassen.\\
        Soße: Bratensatz mit der Grundsoße auffüllen und, während das Fleisch
        sich ausruht, köcheln lassen. Soße passieren, gegebenenfalls noch
        einkochen und mit Bier, Salz und Pfeffer würzig abschmecken. \\
        Braten krusten: Braten auf ein Gitter setzen und zum Krusten für etwa
        20~Minuten bei starker Hitze, \grad{230} in den Ofen schieben. Ein
        Reindl mit etwas Wasser unterstellen. Erst jetzt entsteht die gewünschte
        und begehrte Kruste und der Braten wird nochmal richtig heiß. \\
        Fertigstellen und Anrichten: Fleisch zwischen der Kruste aufschneiden
        und mit der Biersoße und Kartoffelknödel anrichten und servieren. \\
        Dazu gibt es in Bayern Sauer- oder Blaukraut oder Salat! \\
      \end{zubereitung}

      \begin{zutaten}
      \end{zutaten}

      \begin{zutat}{Kartoffelknödel}
        1 kg & \myindex{Pellkartoffel}n vom Vortag, geschält \\
	ca. 100 g & \myindex{Mehl} \\
	80 g & \myindex{Grieß} \\
	ca. 50 g & \myindex{Kartoffelstärke} (Kartoffelmehl) \\
	1 Teelöffel & \myindex{Salz} \\
	1 & \myindex{Ei} \\
	2 & \myindex{Ei}gelb \\
	1 & \myindex{Brötchen} (Semmel) vom Vortag \\
	50 g & \myindex{Butter} \\
	1 Eßlöffel & \myindex{Kartoffelstärke} (Kartoffelmehl) \\
	30 g & \myindex{Semmelbrösel} \\
	50 g & \myindex{Butter} \\
      \end{zutat}

      \personen{6--8}

      \begin{zubereitung}
        Semmel (Brötchen) in etwa 1~cm große Würfel schneiden, diese in Butter
        knusprig braun braten, Kartoffeln fein reiben. Semmelbrösel in Butter
        leicht bräunen. Geriebene Kartoffeln mit Grieß, Mehl, Kartoffelstärke,
        Salz, Ei und Eigelb gut vermengen --- bis ein geschmeidiger, griffiger
        Teig entsteht. Knödel garen: Etwa 2\breh{}~Liter Salzwasser mit
        angerührter Kartoffelstärke leicht binden. Am besten zunächst einen
        kleinen Probeknödel machen, ist er zu weich, dann noch etwas
        Kartoffelmehl zugeben. Jetzt Knödel in gewünschter Größe formen, diese
        in der Mitte mit einigen gerösteten Semmelwürfeln füllen. Knödel in 
        das kochende Wasser geben und am Siedepunkt, je nach Größe der Knödel
        15--25~Minuten gar ziehen lassen. Abschließend Knödel anrichten und mit
        Butterbröseln überziehen. \\
      \end{zubereitung}

    \mynewsection{Allgäuer Kasspatzn}

      % 05.10.2011 BR Wir in Bayern Platz 2

      \begin{zutaten}
      \end{zutaten}

      \begin{zutat}{Spätzle}
        400 g & \myindex{Mehl} \\
	6 & \myindex{Ei}er \\
	& \myindex{Muskatnuß} oder \myindex{Macis} \\
	etwas & \myindex{Salz} \\
	etwas & \myindex{Wasser} \\
	200 g & \myindex{Emmentaler}\index{Käse>Emmentaler} \\
	100 g & \myindex{Bergkäse}\index{Käse>Berg-} \\
	100 ml & \myindex{Sahne} \\
	1 Eßlöffel & geschnittene \myindex{Petersilie} \\
	& Röstzwiebeln\index{Zwiebel} \\
	etwas & \myindex{Butter} \\
	2 größere & \myindex{Zwiebel}n \\
	& wahlweise \myindex{Butterschmalz} oder geklärte \myindex{Butter} \\
      \end{zutat}

      \begin{zutat}{Salat}
        1 & \myindex{Endivie} \\
	1 Bund & \myindex{Schnittlauch} \\
	& \myindex{Senf} \\
	& \myindex{Zucker} \\
	& \myindex{Salz} \\
	& \myindex{Pfeffer} \\
	& \myindex{Essig} \\
      \end{zutat}

      \begin{zubereitung}
        Spätzle: Einen großen Topf Wasser zum Kochen bringen und salzen. Aus dem
        Mehl, den Eiern, Muskat und dem Salz einen Teig schlagen. Den Teig dann
        nach Belieben entweder mit dem Spätzlehobel oder vom Brett schaben. \\
        Teig in den Spätzlehobel oder aufs Brett geben, in das kochende Wasser
        drücken oder schaben. Wenn die Spätzle an die Oberfläche steigen,
        abschöpfen und ins Sieb geben. Trocken in die feuerfeste Form geben. \\
        Dann Sahne in einem Topf erwärmen, mit Muskat, Salz und Pfeffer
        abschmecken, den Emmentaler dazu geben und diese Sahnemasse über die
        Spätzle gießen. Alles gut durchrühren, den geriebenen Bergkäse darauf
        streuen und das Ganze dann im vorgeheizten Ofen bei \grad{170}
        15--20~Minuten backen. \\
        Röstzwiebeln: Butterschmalz schmelzen, Zwiebeln in Ringe schneiden, dazu
        geben, leicht ansalzen und bei mittlerer Temperatur bräunen. \\
        Anrichten: Zum Schluß die Pfanne aus dem Ofen nehmen, Zwiebeln und
        Petersilie darüber streuen und auf die Teller verteilen. \\
        Endiviensalat mit süßem Dressing: 1 Endivie in feine Streifen schneiden.
        Dann 1 Bund Schnittlauch mit der Schere klein schneiden. Eine
	Vinaigrette aus körnigem Senf herstellen: Zucker, Salz, Pfeffer, Essig
	und Öl und einem Spritzer Wasser. Alles emulgieren und über den Salat
	geben. \\
      \end{zubereitung}

    \mynewsection{Semmelknödel mit Schwammerlsoß'}

      % 04.10.2011 Wir in Bayern BR Platz 3

      \begin{zutaten}
      \end{zutaten}

      \begin{zutat}{Semmelknödel}
        500 g & \myindex{Knödelbrot} \\
	3 & \myindex{Ei}er \\
	250 ml & \myindex{Milch} \\
	& \myindex{Salz} \\
	& \myindex{Pfeffer} aus der Mühle \\
	& \myindex{Muskatnuß} \\
	& eventuell \myindex{Semmelbrösel} \\
	1 kleiner Bund & \myindex{Petersilie} \\
      \end{zutat}

      \begin{zutat}{Schwammerlsoß'}
        400 g & \myindex{Reherl}\index{Pilze>Reherl}
	        (\myindex{Pfifferlinge} gewaschen) \\
	400 g & \myindex{Steinpilz}\index{Pilze>Stein-}e \\
	400 g & \myindex{Champignon}\index{Pilze>Champignon}s \\
	2 Eßlöffel & \myindex{Butter} \\
	& \myindex{Salz} \\
	& \myindex{Pfeffer} \\
	1 Stamperl & \myindex{Cognac} \\
	400 ml & \myindex{Sahne} \\
	1 Zehe & \myindex{Knoblauch} geschält \\
      \end{zutat}

      \begin{zubereitung}
        Semmelknödel: Milch erwärmen, über das Knödelbrot gießen und einweichen.
        Dann Eier und Gewürze zugeben und vermischen. Petersilie waschen, zupfen
        und hacken. Knödel abdrehen und im Salzwasser kochen. \\
        Schwammerlsoß': Schwammerl in beliebige Stücke schneiden. Butter
        zerlaufen lassen. Schwammerl anschwitzen, würzen, Knoblauch zugeben.
        Mit Cognac ablöschen, Sahne zugeben und beliebig reduzieren.
        Knoblauch entfernen, abschmecken. \\
        Anrichten: Schwammerl in einem tiefen Teller geben, Knödel aus dem
	Wasser heben, in gehackter Petersilie wälzen und auf den Schwammerln
        anrichten. \\
      \end{zubereitung}

    \mynewsection{Rheinischer Sauerbraten mit Rotkohl und Klößen}

      % WDR 15.10.2011 daheim & unterwegs

      \begin{zutaten}
      \end{zutaten}

      \begin{zutat}{Sauerbraten}
        2 kg & \myindex{Rindfleisch} (Semmerrolle aus der Keule) \\
	\brev{} & \myindex{Sellerie}knolle \\
	1 & \myindex{Möhre} \\
	2 & \myindex{Zwiebel}n \\
	1 Scheibe & \myindex{Pumpernickel} \\
	20--30 & \myindex{Rosinen} \\
	\brdz{} Liter & \myindex{Rinderbrühe} \\
	\brez{} Liter & \myindex{Kräuteressig}\index{Essig>Kräuter-} \\
	\brez{} Liter & \myindex{Rotweinessig}\index{Essig>Rotwein-} \\
	\brsz{} Liter & trockener \myindex{Rotwein}\index{Wein>rot} (Typ Bordeaux) \\
	\brzz{} Liter & \myindex{Zuckerrübensirup} \\
	30 g & \myindex{Butterschmalz} \\
	4 & \myindex{Wacholderbeeren} \\
	10 & \myindex{Pfeffer}körner \\
	2 & \myindex{Lorbeer}blätter \\
	1 Zweiglein & \myindex{Thymian} \\
	etwas & \myindex{Salz} \\
	etwas & \myindex{Pfeffer}, frisch gemahlen \\
      \end{zutat}

      \begin{zutat}{Rotkohl}
        \breh{} Kopf & \myindex{Rotkohl} (frisch) \\
	2 & \myindex{Boskopp}\index{Aepfel=Äpfel>Boskopp-}äpfel \\
	1 & \myindex{Zwiebel} \\
	2 & \myindex{Gewürznelken} \\
	2 & \myindex{Wacholderbeeren} \\
	1 Blatt & \myindex{Lorbeer} \\
	etwas & \myindex{Zucker} \\
	etwas & \myindex{Apfelessig}\index{Essig>Apfel-} \\
	1 Tasse & trockener \myindex{Rotwein}\index{Wein>rot} \\
	125 Gramm & \myindex{Schweineschmalz} \\
	etwas & \myindex{Salz} \\
	etwas & \myindex{Pfeffer} aus der Mühle \\
      \end{zutat}

      \begin{zutat}{Kartoffelklöße}
        1\breh{} kg & \myindex{Kartoffel}n \\
	170 Gramm & \myindex{Mehl} \\
	2 & \myindex{Ei}er \\
	etwas & \myindex{Salz} \\
      \end{zutat}

      \begin{zubereitung}
        Sauerbraten: Das Fleisch abwaschen und trocken tupfen, Gemüse
        kleinschneiden, alle Zutaten (außer Fett, Rosinen, Brühe und Brot) in
        eine verschließbare Schüssel geben und das Fleisch hinein geben.
        5--7~Tage marinieren lassen und täglich wenden. Am Tag der Zubereitung
        Fleisch herausnehmen und trocken tupfen. Die Marinade 30~Minuten lang
        kochen, danach abseihen, die Flüssigkeit auffangen. Das Fleisch von
        allen Seiten kräftig anbraten und würzen. Überflüssiges Fett entfernen
        und mit der Brühe und Marinade ablöschen. Etwa 100~Minuten köcheln
        lassen, danach das zerbröselte Brot, den Zuckerrübensirup und die
        Rosinen zugeben. Nochmals etwa 20~Minuten köcheln lassen. \\
        Apfelrotkohl (weiteres Rezept auf Seite \pageref{rotkohl}): Vom Rotkohl
        die Blätter entfernen, den Kohl waschen, vierteln, den Strunk entfernen,
        den Kohl sehr fein schneiden oder hobeln. Zwiebel abziehen, würfeln. Die
        Äpfel schälen, entkernen und würfeln. Das Schweineschmalz zerlassen, die
        Zwiebel darin hellgelb rösten, den Kohl dazugeben, andünsten, die Äpfel,
        1~Lorbeerblatt, die Gewürznelken, Wacholderbeeren, Zucker, 2~Eßlöffel
        Apfelessig, 1~Tasse Rotwein hinzufügen und gar dünsten lassen.
        1~Eßlöffel Weizenmehl mit 2~Eßlöffel Wasser anrühren, den Rotkohl damit
        binden, mit Salz, Pfeffer, Zucker und Essig abschmecken. Dünstzeit ca.
        2~Stunden. \\
        Kartoffelklöße: Die Kartoffeln schälen und waschen. Die Hälfte davon
        mit Wasser bedeckt in einer Schüssel beiseite stellen. Die übrigen in
        einem Topf mit gesalzenem Wasser bedeckt 20~Minuten kochen lassen. Das
        Wasser anschließend abgießen. Die Kartoffeln ausdämpfen und abkühlen
        lassen. Die rohen Kartoffeln auf ein Küchentuch reiben und ausdrücken.
        In eine Schüssel geben und die gekochten Kartoffeln drauf reiben. Etwas
        vom Mehl zum Wenden abnehmen und auf einen Teller geben. Das restliche
        Mehl unter die Kartoffelmasse mischen und die Eier unterrühren. Nach
        Bedarf salzen. 2\breh{}~Liter leicht gesalzenes Wasser in einem Topf
        aufkochen. Aus der Kartoffelmasse mit bemehlten Händen Klöße von
        jeweils 5~cm~Durchmesser formen und in dem abgenommenen Mehl wenden.
        In leicht siedendem Wasser ca.~20~Minuten ziehen lassen. Mit einem
        Schaumlöffel herausnehmen und abtropfen lassen. \\
      \end{zubereitung}

    \mynewsection{Entenbraten mit Rotkraut und Kastanien}

      % SWR 12.10.2011 kaffee-oder-tee Rheinland-Pfalz

      \begin{zutaten}
      \end{zutaten}

      \begin{zutat}{Rotkohl}
        1 kleiner & \myindex{Rotkohl} (ca. 1 kg) \\
	\brez{} Liter & \myindex{Rotweinessig}\index{Essig>Rotwein-} \\
	& \myindex{Salz} \\
	& \myindex{Zucker} \\
	1 & säuerlicher \myindex{Apfel} \\
	1 & \myindex{Zwiebel} \\
	2 Eßlöffel & \myindex{Schweineschmalz}\index{Schmalz>Schweine-}
	             oder \myindex{Gänseschmalz}\index{Schmalz>Gänse-} \\
	\brev{} Liter & \myindex{Rotwein}\index{Wein>rot} \\
	\brez{} Liter & \myindex{Fleischbrühe} \\
	\breh{} & \myindex{Zimtstange} \\
	1 & \myindex{Gewürznelke} \\
	1 kleines & \myindex{Lorbeer}blatt \\
      \end{zutat}

      \begin{zutat}{Ente und Soße}
	2 Zweige & \myindex{Thymian} \\
	3 & \myindex{Zwiebel}n \\
	2 & säuerliche Äpfel\index{Aepfel=Äpfel} \\
	\breh{} Teelöffel & \myindex{Beifuß} \\
	1 & \myindex{Bauernente}\index{Ente>Bauern-} (2\breh{}~kg; küchenfertig)
	    \\
	& \myindex{Salz} \\
	& \myindex{Pfeffer} aus der Mühle \\
	1 Bund & \myindex{Suppengemüse} \\
	1 Eßlöffel & Öl\index{Oel=Öl} \\
	1 Eßlöffel & \myindex{Tomatenmark} \\
	\brdz{} Liter & \myindex{Rotwein}\index{Wein>rot} \\
      \end{zutat}

      \begin{zutat}{Eßkastanien}
	1 & \myindex{Schalotte} \\
	3 Eßlöffel & \myindex{Butter} \\
	250 g & \myindex{Eßkastanien} (Maronen; vorgegart und geschält) \\
	2 Eßlöffel & \myindex{Puderzucker}\index{Zucker>Puder-} \\
	\brez{} Liter & \myindex{Hühnerbrühe} \\
	\brez{} Liter & \myindex{Apfelsaft} \\
	& \myindex{Salz} \\
      \end{zutat}

      \personen{4}

      \begin{zubereitung}
        Rotkohl: Den Kohl putzen, waschen, vierteln und in feine Streifen
        schneiden oder hobeln. Mit dem Essig, je 1~Teelöffel Salz und Zucker
        mischen. Zugedeckt mindestens 2~Stunden ziehen lassen. \\
        Für den Rotkohl den Apfel und die Zwiebeln schälen. Apfel entkernen, mit
        der Zwiebel reiben und in einem Topf im Schmalz andünsten. Mit
        1~Eßlöffel Zucker bestreuen, den Kohl und den entstandenen Saft
        dazugeben. Den Wein, die Brühe, Zimt, Nelke und Lorbeerblatt
        hinzufügen. Rotkohl zugedeckt bei schwacher Hitze 30~Minuten schmoren.
	\\
        Ente: Den Backofen auf \grad{180} vorheizen. Den Thymian waschen,
        trocken schütteln und die Blätter abzupfen. Die Zwiebeln schälen, die
        Äpfel waschen und entkernen, beides in Würfel schneiden und mit Thymian
        und Beifuß mischen. Die Ente waschen, trocken tupfen, die Flügelspitzen
        abtrennen, grob hacken und beiseitestellen. Die Ente innen und außen mit
        Salz und Pfeffer einreiben. Mit der Apfel-Zwiebel-Mischung füllen. Die
        Öffnung mit Küchengarn zunähen. \\
        Die Ente mit der Brustseite nach unten in einen Bräter legen.
        \breh{}~Liter heißes Wasser angießen. Die Ente im Ofen auf der mittleren
        Schiene etwa 2~Stunden goldbraun braten. Nach 1~Stunde Garzeit die Ente
        umdrehen und mehrfach mit dem Bratenfond übergießen. \\
        Das Suppengemüse putzen und waschen bzw. schälen und in kleine Würfel
        schneiden. Das Öl in einem Topf erhitzen. Die Flügelstücke und das
        Gemüse darin anbraten. Das Tomatenmark kurz mitrösten, mit dem Wein
        ablöschen und etwa 30~Minuten leicht köcheln lassen. \\
        Maronen: Die Schalotte schälen und in feine Würfel schneiden. Die Butter
        in einer Pfanne erhitzen, die Schalotte darin andünsten. Die Maronen
        dazugeben, mit Puderzucker bestäuben und karamellisieren. Mit Brühe und
        Apfelsaft ablöschen. Maronen etwa 10~Minuten köcheln lassen, bis die
        Soße fast eingekocht ist. Mit Salz würzen. \\
        Die Ente aus dem Ofen nehmen, zerlegen und die Entenstücke im
        ausgeschalteten Ofen warm halten. Den Entenfond aus dem Bräter
	entfetten, mit der Rotweinsoße durch ein Sieb in einen Topf gießen,
	dabei das Gemüse gut ausdrücken. Aufkochen, nach Belieben leicht binden
	und mit Salz und Pfeffer abschmecken. Den Rotkohl eventuell nachwürzen.
	Die Ententeile mit Rotkohl, Maronen und der Soße servieren. \\
      \end{zubereitung}

    \mynewsection{Zwiebelrostbraten}

      % SWR 05.10.2011 kaffee-oder-tee Baden-Württemberg

      \begin{zutaten}
        4 & dicke \myindex{Rumpsteak}s (\'a ca. 200 g) \\
	ca. 2 Eßlöffel & Öl\index{Oel=Öl} \\
	4 & \myindex{Zwiebel}n \\
	3 Eßlöffel & \myindex{Butter} \\
	1 Teelöffel & \myindex{Zucker} oder
	              \myindex{Puderzucker}\index{Zucker>Puder-} \\
        1 Zweig & \myindex{Thymian} \\
	1 Eßlöffel & \myindex{Tomatenmark} \\
	\brev{} Liter & \myindex{Rotwein}\index{Wein>rot} \\
	2 Eßlöffel & \myindex{Aceto balsamico} \\
	& \myindex{Salz} \\
	& \myindex{Pfeffer} aus der Mühle \\
	& getrockneter \myindex{Majoran} \\
	2 Eßlöffel & \myindex{Butterschmalz} \\
	\brea{} Liter & \myindex{Fleischbrühe} oder \myindex{Geflügelbrühe} \\
      \end{zutaten}

      \personen{4}

      \begin{zubereitung}
        Die Rumpsteaks waschen, trocken tupfen, mit dem Öl bestreichen und
	zugedeckt kühl stellen. \\
	Die Zwiebeln schälen und in feine Streifen oder dünne Spalten schneiden.
	In einer Pfanne 2~Eßlöffel Butter erhitzen, die Zwiebeln darin goldbraun
	braten und beiseitestellen. \\
	In einem Topf die restliche Butter erhitzen und den Zucker darin leicht
	karamellisieren. Den Thymianzweig waschen, trocken tupfen, mit dem
	Tomatenmark in den Topf geben und unter Rühren bei schwacher Hitze
	leicht andünsten. Mit dem Wein und dem Essig ablöschen und kurz
	köcheln lassen. Die gebratenen Zwiebeln hinzufügen und alles kurz
	aufkochen. Mit Salz, Pfeffer und 1~Prise Majoran würzen und warm halten.
	\\
	Den Fettrand der Rumpsteaks mehrmals einschneiden. Die Steaks
	vorsichtig leicht flach klopfen, mit Salz und Pfeffer würzen. Das
	Butterschmalz in einer großen Pfanne erhitzen. Die Steaks darin bei
	mittlerer bis starker Hitze auf beiden Seiten braun anbraten, aus der
	Pfanne nehmen, in Alufolie wickeln und warm halten. \\
	Den Bratenansatz in der Pfanne mit der Brühe ablöschen und etwa
	10~Minuten einkochen lassen. Die Rotwein-Zwiebeln (ohne den Thymian) in
	die Pfanne geben und aufkochen. Die Zwiebelsoße nach Belieben mit
	etwas kalt angerührter Speisestärke binden und mit Salz und Pfeffer
	würzen. \\
	Die Steaks mitsamt dem ausgetretenen Fleischsaft in die Zwiebelsoße
	geben und bei sehr schwacher Hitze weitere 10~Minuten ziehen lassen.
	Den Zwiebelrostbraten auf Tellern anrichten und nach Belieben mit
	Schnittlauchröllchen bestreuen. Dazu schmecken Spätzle oder
	Bratkartoffeln und Sauerkraut. \\
      \end{zubereitung}

    \mynewsection{Dibbelabbes}

      % 19.10.2011 SWR kaffee-oder-tee Saarland

      \begin{zutaten}
      2 kg & festkochende \myindex{Kartoffel}n \\
      2 & \myindex{Zwiebel}n \\
      1 & \myindex{Ei} \\
      100 g & \myindex{Speck}, gewürfelt \\
      1 Stange & \myindex{Lauch} \\
      80 g & \myindex{Mehl} \\
      & Öl\index{Oel=Öl} zum Anbraten (Sonnenblumenkern) \\
      & \myindex{Salz} \\
      & \myindex{Pfeffer} \\
      & \myindex{Cayennepfeffer}\index{Pfeffer>Cayenne-} \\
      & \myindex{Muskatnuß} \\
      1 Bund & Blatt-\myindex{Petersilie} (zum Anrichten) \\
      \end{zutaten}

      \personen{4}

      \begin{zubereitung}
        Kartoffeln waschen und schälen. Mit der groben Seite der Reibe die
	Kartoffeln reiben. Durch ein Sieb fest ausdrücken. Geriebene
	Kartoffeln in eine Schüssel geben. Lauch putzen, waschen und die zu
	grünen Blätter wegschneiden. In feine Streifen schneiden und der
	Kartoffelmasse zugeben. Speck, Mehl und Vollei der Masse zufügen. Mit
	Salz, Pfeffer, Cayenne und Muskatnuß gut abschmecken. \\
      \end{zubereitung}

    \mynewsection{Handkäse mit Musik}

      % 19.10.2011 hr service-trends Hessen

      \begin{zutaten}
        12 Stücke & \myindex{Handkäse} (mit oder ohne Kümmel) \\
	4 mittelgroße & \myindex{Zwiebel}n \\
	50 ml & \myindex{Sonnenblumenöl}\index{Oel=Öl>Sonnenblumen-} \\
	50 ml & \myindex{Apfelessig}\index{Essig>Apfel-} oder
	        \myindex{Weinessig}\index{Essig>Wein-} \\
	etwas & \myindex{Paprikapulver} (edelsüß) \\
	3 Spritzer & \myindex{Apfelwein}\index{Wein>Apfel-} \\
	& \myindex{Salz} \\
	& \myindex{Pfeffer} \\
      \end{zutaten}

      \personen{4}

      \begin{zubereitung}
        Zwiebeln in Würfel schneiden und mit Öl, Essig und Apfelwein vermengen.
	Mit Pfeffer und Salz (nach Geschmack) würzen. Die Marinade über den
	gereiften Handkäse geben und mit dem Paprikapulver bestäuben. \\
	Zum Handkäse mit Musik paßt ein kräftiges Schwarzbrot mit Butter und ein
	Glas Apfelwein. \\
      \end{zubereitung}

    \mynewsection{Lauchblatzen mit Ahler Worscht}

      % 19.10.2011 hr service-trends Hessen

      \begin{zutaten}
        \brev{} Liter & \myindex{Milch} \\
	500 g & \myindex{Mehl} \\
	30 g & \myindex{Hefe} \\
	75 g & \myindex{Butter} \\
	& \myindex{Zucker} \\
	5 & \myindex{Ei}er \\
	1 kg & \myindex{Lauch} \\
	350 g & \myindex{Ahle Worscht} (leicht geräuchert) \\
	\breh{} Liter & \myindex{Schmand} \\
	& \myindex{Salz} \\
	etwas & \myindex{Muskatnuß} \\
      \end{zutaten}

      \personen{4}

      \begin{zubereitung}
        Mehl in eine Schüssel geben und eine Mulde in die Mitte drücken, dann
	die Hefe hinein bröckeln. Die Hefe mit einem Teil des Mehls und mit
	etwas lauwarmer Milch verrühren. Den Teig unter einem Tuch etwa
	20~Minuten gehen lassen oder etwa 10~Minuten im Ofen bei \grad{45}
	erwärmen. Dann die restliche Milch, ein Ei, etwas Zucker und die
	geschmolzene, lauwarme Butter hinzugeben. Alles verkneten, bis sich der
	Teig von der Schüssel löst. Den Teig abdecken und 15~Minuten gehen
	lassen oder wieder ca. 10~Minuten im Ofen bei \grad{45} erwärmen.
	Anschließend auf einer bemehlten Arbeitsfläche zu gewünschter Größe
	ausrollen, auf ein gefettetes Backblech geben und den Teig an den
	Rändern hoch drücken. Nun den Lauch putzen und in Ringe schneiden. Die
	Ahle Worscht in Würfel schneiden und mit dem Lauch, Schmand und den
	4~Eiern verrühren. Mit Salz und geriebener Muskatnuß abschmecken. Die
	Masse auf den Teig geben und nochmals 15--20~Minuten gehen lassen. Den
	Blatzen in den auf \grad{200} vorgeheizten Ofen geben und etwa
	40~Minuten backen. \\
      \end{zubereitung}

    \mynewsection{Frankfurter Grüne Soße}

      \begin{zutaten}
        2 & \myindex{Grüne Soße}-Mischungen \\
	3 & \myindex{Ei}gelb \\
	\brea{} Liter & Öl\index{Oel=Öl} \\
	1 Teelöffel & \myindex{Senf} \\
	1 Eßlöffel & weißer \myindex{Balsamico-Essig}\index{Essig>Balsamico-} \\
	1 Spritzer & \myindex{Worcestershiresoße} \\
	400 g & \myindex{Joghurt} (3\breh{} \% Fett) \\
	3 & hartgekochte \myindex{Ei}er \\
	& \myindex{Salz} \\
	& \myindex{Pfeffer} \\
	& \myindex{Zitrone}nsaft \\
      \end{zutaten}

      \personen{4}

      \begin{zubereitung}
        Die Grüne Soße-Mischungen waschen, von groben Stielen befreien und fein
	hacken. Aus Eigelb, Senf, Balsamico-Essig, Worcestershiresoße, Salz,
	Pfeffer und Zitronensaft (nach Geschmack) eine Mayonnaise zubereiten und
	mit dem Joghurt verrühren. Die gehackten Kräuter dazugeben und mit Salz
	(nach Geschmack) und Pfeffer (nach Geschmack) abschmecken. Gekochte,
	gehackte Eier unterheben. Servieren mit Tafelspitz und Pellkartoffeln.\\
      \end{zubereitung}

    \mynewsection{Frischkäse-Parfait im Grüne Soße-Mantel}

      \begin{zutaten}
        2 Päckchen & frische \myindex{Grüne Soße} \\
	300 g & \myindex{Ziegenfrischkäse}\index{Käse>Ziegenfrisch-} \\
	300 g & \myindex{Frischkäse}\index{Käse>Frisch-} (Philadelphia o.ä.) \\
	10 Blatt & \myindex{Gelatine} \\
	50 ml & \myindex{Apfelwein}\index{Wein>Apfel-} \\
	& \myindex{Salz} \\
	& \myindex{Pfeffer} \\
      \end{zutaten}

      \personen{4}

      \begin{zubereitung}
        Kräuter waschen, eine Tasse Kräuter fein schneiden, den Rest grob
	zusammenschneiden. Die grobgeschnittenen Kräuter mit dem Entsafter
	entsaften. 200~ml des Saftes mit dem Apfelwein, Salz und Pfeffer
	abschmecken. Nun 4 Blatt in kaltem Wasser eingeweichte Gelatine
	ausdrücken und leicht erhitzen. Mit dem Kräutersaft mischen und auf ein
	kleines, mit Klarsichtfolie ausgelegtes Blech gießen. Im Kühlschrank
	erstarren lassen. Beide Frischkäse mit dem restlichen Kräutersaft
	mischen und mit Salz und Pfeffer abschmecken. 6 Blatt in kaltem Wasser
	eingeweichte Gelatine ausdrücken und leicht erhitzen und unter die
	Kräuter-Käsemasse ziehen. Eine Parfait- oder kleine Kuchenform mit dem
	erstarrten Kräutersaft auslegen. Darauf achten, daß die Folie an der
	Form anliegt. Die Frischkäsemasse einfüllen und im Kühlschrank
	mindestens 4~Stunden durchkühlen, bis das Parfait schnittfest ist. \\
      \end{zubereitung}

    \mynewsection{Grüne Soße-Buttermilchdrink}

      \begin{zutaten}
        2 Becher & \myindex{Buttermilch} \'a 500 g \\
	\breh{} Päckchen & frische Frankfurter \myindex{Grüne Soße} \\
	& \myindex{Salz} \\
	& \myindex{Pfeffer} \\
	2 Eßlöffel & gutes \myindex{Erdnußöl}\index{Oel=Öl>Erdnuß-} \\
      \end{zutaten}

      \personen{4}

      \begin{zubereitung}
        Kräuter waschen, große Stiele entfernen und grob hacken. Die Buttermilch
	in einen hohen Mixbecher geben, salzen, pfeffern und die Kräuter
	zugeben. Mit dem Stabmixer fein pürieren und zum Schluß das Erdnußöl
	einmixen. Kalt servieren. \\
      \end{zubereitung}

    \mynewsection{Frankfurter Grüne Soße-Salat}

      \begin{zutaten}
        1 Päckchen & frische Frankfurter \myindex{Grüne Soße} \\
	\breh{} Kopf & \myindex{Eisbergsalat}\index{Salat>Eisberg-} \\
	1 Tasse & geröstete, geschälte und gehackte \myindex{Erdnüsse} \\
	8 & \myindex{Radieschen} \\
	8 & weiße rohe \myindex{Champignon}s \\
	80 g & alter \myindex{Gouda}\index{Käse>Gouda} \\
      \end{zutaten}

      \begin{zutat}{Vinaigrette}
        & \myindex{Salz} \\
        & \myindex{Pfeffer} \\
        & \myindex{Zucker} \\
        & \myindex{Erdnußöl}\index{Oel=Öl>Erdnuß-} \\
        & \myindex{Rapsöl}\index{Oel=Öl>Raps-} \\
        & \myindex{Apfelbalsamessig}\index{Essig>Apfelbalsam-} \\
      \end{zutat}

      \personen{4}

      \begin{zubereitung}
        Kräuter waschen, große Stiele entfernen und grob zupfen. Eisbergsalat
	putzen und zupfen, Radieschen und Champignons in feine Scheiben
	schneiden. Aus den angegebenen Zutaten eine Salatsoße nach Geschmack
	rühren. Kräuter, Eisbergsalat, Radieschen und Champignons mischen und
	mit der Vinaigrette anmachen. Salat mit gehackten Erdnußkernen und grob
	gehobeltem Gouda bestreuen. \\
      \end{zubereitung}

    \mynewsection{Kürbiseintopf mit Handkäse}

      \begin{zutaten}
        500 g & \myindex{Gemüsekürbis}\index{Kürbis>Gemüse-} \\
	150 g & \myindex{Kartoffel}n \\
	1 kleine & \myindex{Gemüsezwiebel}\index{Zwiebel>Gemüse-} \\
	\brdv{} Liter & \myindex{Gemüsebrühe} \\
	150 g & \myindex{Möhren} \\
	150 g & \myindex{Sellerie} \\
	100 g & \myindex{Lauch} \\
	1 Stange & \myindex{Zimt} \\
	2 Stück & \myindex{Sternanis} \\
	100 g & gereifter \myindex{Handkäse} \\
	20 ml & Öl\index{Oel=Öl} \\
	& \myindex{Salz} \\
	& \myindex{Pfeffer} \\
      \end{zutaten}

      \personen{4}

      \begin{zubereitung}
        Gemüsekürbis, Kartoffeln, Möhren, Sellerie sowie Zwiebeln in Würfel
	schneiden, den Lauch in Streifen. Die Zwiebeln mit dem Öl leicht
	andünsten, das andere Gemüse dazugeben und weiter andünsten, ca.
	3--4~Minuten. Die Zimtstange und Sternanis ebenfalls dazugeben.
	Anschließend mit der Gemüsebrühe auffüllen, aufkochen lassen und bei
	mittlerer Stufe 25~Minuten simmern lassen. Mit Salz und Pfeffer würzen,
	den Sternanis und die Zimtstange wieder aus der Suppe nehmen. Den
	Handkäse in kleine Würfel schneiden und auf die angerichtete Suppe
	geben. \\
      \end{zubereitung}

    \mynewsection{Selleriepüree und Ahle Worscht}

      \begin{zutaten}
        400 g & \myindex{Knollensellerie}\index{Sellerie>Knollen-}, geputzt \\
	50 ml & \myindex{Wasser} \\
	& \myindex{Salz} \\
	& \myindex{Pfeffer} \\
	& \myindex{Muskatnuß} \\
	200 g & \myindex{Ahle Worscht} \\
	50 ml & \myindex{Leinöl}\index{Oel=Öl>Lein-} \\
      \end{zutaten}

      \personen{4}

      \begin{zubereitung}
        Den Sellerie in Würfel schneiden und mit dem Wasser in einem Topf
	langsam garen. Anschließend in einen Mixer geben und pürieren, mit
	Salz, Pfeffer und Muskat abschmecken. Die Ahle Worscht in feine Streifen
	schneiden und auf dem angerichteten Selleriepüree verteilen und mit
	dem Leinöl marinieren. Dazu paßt sehr gut Feldsalat. \\
      \end{zubereitung}

    \mynewsection{Karamellisierte Äpfel mit Frankfurter Grüne Soße}

      \begin{zutaten}
        8 & Äpfel\index{Aepfel=Äpfel} (Elstar) \\
	50 g & \myindex{Zucker} \\
	30 ml & \myindex{Apfelsaft} \\
	30 g & \myindex{Butter} \\
	1 Prise & \myindex{Salz} \\
	50 g & \myindex{Grüne Soße} \\
      \end{zutaten}

      \personen{4}

      \begin{zubereitung}
        Äpfel schälen und achteln. Den Zucker in einer Pfanne karamellisieren
	und mit dem Abfelsaft ablöschen. Butter, Salz und die gehackten
	Kräuter der Grünen Soße dazugeben und leicht einköcheln. Anschließend
	die Äpfel dazugeben und gut durchschwenken. Alles kurz aufköcheln, dann
	ist es fertig. Dazu schmeckt Vanilleeis sehr gut. \\
      \end{zubereitung}
